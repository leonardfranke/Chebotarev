\section{Spezielle Indexmengen}

\newcolumntype{C}{>{$}c<{$}}

In diesem Kapitel wollen wir spezielle Indexmengen betrachten und Bedingungen aufstellen, wann die Submatrix $M_{IJ}$ für ein $a \in \field{p}[n]$ invertierbar ist. Wir betrachten zunächst den allgemeinen Fall für $|I| = |J| = 3$. Seien $I = \{0, \alpha, \beta\}$ und $J = \{0, \gamma, \delta\}$. Dann besitzt $M_{IJ}$ die Form 
\begin{equation*}
    \begin{pmatrix}
        1 & 1 & 1 \\
        1 & x^{\alpha\gamma} & x^{\beta\gamma} \\
        1 & x^{\alpha\delta} & x^{\beta\delta}
    \end{pmatrix}.
\end{equation*}

Mithilfe von Zeilenumformungen und anschließender Spaltenentwicklung ergibt sich die Determinante wie folgt:

\begin{align*}
    \det \begin{pmatrix}
        1 & 1 & 1 \\
        1 & x^{\alpha\gamma} & x^{\beta\gamma} \\
        1 & x^{\alpha\delta} & x^{\beta\delta}
    \end{pmatrix} 
    &= \det \begin{pmatrix}
        1 & 1 & 1 \\
        0 & x^{\alpha\gamma} -1 & x^{\beta\gamma} -1 \\
        0 & x^{\alpha\delta} -1 & x^{\beta\delta} -1
    \end{pmatrix} \\
    &= \det \begin{pmatrix}
        x^{\alpha\gamma} -1 & x^{\beta\gamma} -1 \\
        x^{\alpha\delta} -1 & x^{\beta\delta} -1
    \end{pmatrix} \\
    &= (x^{\alpha\gamma} -1)(x^{\beta\delta} -1) - (x^{\alpha\delta} -1)(x^{\beta\gamma} -1)
\end{align*}

Für ein $a \in \field{p}[n]$ ist die Submatrix $M_{IJ}$ somit genau dann nicht invertierbar, falls $a$ die Gleichung $(x^{\alpha\gamma} -1)(x^{\beta\delta} -1) = (x^{\alpha\delta} -1)(x^{\beta\gamma} -1)$ löst. Da diese Gleichung noch sehr allgemein ist, setzen wir noch konkreter $I = \{0,1,3\}$. Dadurch reduziert sich die Bedingung zu

\begin{align}
        & (x^{\gamma} -1)(x^{3\delta} -1) = (x^{\delta} -1)(x^{3\gamma} -1) \nonumber \\
    \iff & \frac{(x^{3\delta} -1)}{(x^{\delta} -1)} = \frac{(x^{3\gamma} -1)}{(x^{\gamma} -1)} \nonumber \\
    \iff & {x^\delta}^2 + x^{\delta} + 1 = {x^{\gamma}}^2 + x^{\gamma} + 1 \nonumber \\
    \iff & {x^\delta}^2 + x^{\delta} = {x^{\gamma}}^2 + x^{\gamma}. \label{equation:013}
\end{align}

Anders formuliert sind nun zwei Elemente $v_1 = a^\delta$ und $v_2 = a^\gamma$ gesucht, die $x^2 + x = m$ für ein festes $m \in \field{p}[n]$ lösen. Abhängig von der Charakteristik des Körpers existieren zwei unterschiedliche Lösungsformeln.

\begin{equation*}
    v_{1,2} = \begin{cases}
        -2^{-1} \pm \sqrt{2^{-2} + m}                                   & \text{falls } \mathrm{char}(\field{p}[n]) \neq 2 \\  
        \sum_{k=1}^{n-1} m^{2^j}(\sum_{l=0}^{k-1} u^{2^k}),\quad v_1 + 1 & \text{sonst}
    \end{cases} 
\end{equation*}

Sofern $v_1,v_2 \neq 0$ sind, existieren für jeden Erzeuger $a \in \field{p}[n]$ Exponenten $\gamma$ und $\delta$ mit $v_1 = a^\gamma$ und $v_2 = a^\delta$. Für diese Erzeuger ist dann die \Cref{equation:013} erfüllt und $M_{IJ}$ nicht invertierbar. 

Im Folgenden betrachten wir ein Beispiel für den Körper $\field{11}$, wobei $-2^{-1} = -6 = 5$ und $2^{-2} = 3$ gilt. In \Cref{table:sol_013F11} sind für jedes 


{\renewcommand{\arraystretch}{1.5}
\begin{table}
    \centering
    \begin{tabular}{|C|C|C|C|C|C|C|C|C|C|C|C|C}
    \hline
    m          & 0    & 1   & 2   & 3 & 4 & 5 & 6   & 7  & 8   & 9   & 10 \\
    \hline
    \sqrt{3+m} & 5    & 2   & 4   & - & - & - & 3   & -  & 0   & 1   & -  \\
    \hline
    v_{1,2} = 5 \pm \sqrt{3+m}   & 10,0 & 3,7 & 9,1 & - & - & - & 8,2 & -  & 5,5 & 6,4 & -  \\
    \hline
    \end{tabular}
    \caption{Lösungen $v_{1,2}$ zu $x^2 + x = m$ für festes $m \in \field{11}$.} \label{table:sol_013F11}
\end{table}
}

\begin{table}
    \centering
    \begin{tabular}{|C|C|C|C|C|C|C|C|C|C|C|C|}
    \hline
    \text{\backslashbox{a}{i}} & 0 & 1  & 2 & 3 & 4 & 5  & 6 & 7 & 8 & 9 & 10 \\ \hline
    1  & 1 & 1  &   &   &   &    &   &   &   &   &   \\ \hline
    2  & 1 & 2  & 4 & 8 & 5 & 10 & 9 & 7 & 3 & 6 & 1 \\ \hline
    3  & 1 & 3  & 9 & 5 & 4 & 1  &   &   &   &   &   \\ \hline
    4  & 1 & 4  & 5 & 9 & 3 & 1  &   &   &   &   &   \\ \hline
    5  & 1 & 5  & 3 & 4 & 9 & 1  &   &   &   &   &   \\ \hline
    6  & 1 & 6  & 3 & 7 & 9 & 10 & 5 & 8 & 4 & 2 & 1 \\ \hline
    7  & 1 & 7  & 5 & 2 & 3 & 10 & 4 & 6 & 9 & 8 & 1 \\ \hline
    8  & 1 & 8  & 9 & 6 & 4 & 10 & 3 & 2 & 5 & 7 & 1 \\ \hline
    9  & 1 & 9  & 4 & 3 & 5 & 1  &   &   &   &   &   \\ \hline
    10 & 1 & 10 & 1 &   &   &    &   &   &   &   &   \\ \hline
    \end{tabular}
    \caption{Alle von einem $a \in \field{11}$ erzeugten Untergruppen.} \label{table:subgroupsF11}
\end{table}

{\renewcommand{\arraystretch}{1.5}
\begin{table}
    \centering
    \begin{tabular}{|C|C|C|C|C|C|C|C|C|}
    \hline
    m               & 0 & 1 & o & o + 1 & o^2 & o^2 + 1 & o^2 + o  & o^2 + o + 1 \\
    \hline
    \mathrm{Tr}(m)  & 0 & 1 & 0 & 1     & 0   & 1       & 0        & 1 \\
    \hline
    v_1         & 0 & - & o^2 & - & o^2 + o & - & o & - \\
    \hline
    \end{tabular}
    \caption{Lösungen $v_1$ zu $x^2 + x = m$ für festes $m \in \field{2}[3]$.} \label{table:sol_013F2_3}
\end{table}
}

\begin{table}
    \centering
    \begin{tabular}{|C|C|C|C|C|C|C|C|C|C|C|C|}
    \hline
    \text{\backslashbox{i}{a}}& 1 & o       & o+1     & o^2     & o^2 + 1 & o^2 + o & o^2 + o + 1 \\ \hline
    0 & 1 & 1       & 1       & 1       & 1       & 1       & 1           \\ \hline
    1 & 1 & o       & o+1     & o^2     & o^2+1   & o^2+o   & o^2+o+1     \\ \hline
    2 &   & o^2     & o^2+1   & o^2+o   & o^2+o+1 & o       & o+1         \\ \hline
    3 &   & o+1     & o^2     & o^2+1   & o^2+o   & o^2+o+1 & o           \\ \hline
    4 &   & o^2+o   & o^2+o+1 & o       & o+1     & o^2     & o^2+1       \\ \hline
    5 &   & o^2+o+1 & o       & o+1     & o^2     & o^2+1   & o^2+o       \\ \hline
    6 &   & o^2+1   & o^2+o   & o^2+o+1 & o       & o+1     & o^2         \\ \hline
    7 &   & 1       & 1       & 1       & 1       & 1       & 1           \\ \hline
    \end{tabular}
    \caption{Alle von einem $a \in \field{2}[3]$ erzeugten Untergruppen.} \label{table:subgroupsF2_3}
\end{table}
