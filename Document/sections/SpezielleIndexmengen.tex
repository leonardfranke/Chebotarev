\section{Spezielle Indexmengen}


In diesem Kapitel wollen wir spezielle Indexmengen betrachten und Bedingungen aufstellen, wann die Submatrix $M_{IJ}$ für ein $a \in \field{p}[n]$ invertierbar ist. 

Wir betrachten zunächst allgemeine $I,J \subseteq \mathbb{Z}$ mit $|I| > \ord{a}$. Dann existieren stets zwei Elemente $i \ne i' \in I$ mit $i \equiv i' \pmod{o}$. Da alle Bedingungen von \Cref{lemma:equal-columns} erfüllt sind, folgt direkt $(\det M_{IJ})(a) = 0$. Es genügt somit Indexmengen zu untersuchen, deren Mächtigkeit kleiner als eine gewählte (Gruppen-)Ordnung ist.

Sei als nächstes $I = \{0,\dots,k-1\}$ und $J$ wieder beliebig. Die Matrix $M_{IJ}$ entspricht dann der Vandermondematrix $V_{(x^{j_1},\dots,x^{j_k})}$, deren Determinanten
\begin{equation*}
    \det V_{(x^{j_1},\dots,x^{j_k})} = \prod_{1\leq v<w\leq k} (x^{j_v} - x^{j_w})
\end{equation*}
bekannt ist. Damit $(\det M_{IJ})(a) = 0$ gilt, müssen wieder zwei Indizes existieren, deren Differenz ein Vielfaches der Ordnung von $a$ entspricht. Falls die Mächtigkeit von $J$ größer als diese Ordnung ist, müssen, wie vorher erläutert, diese Elemente existieren.

Als nächstes betrachten wir den speziellen Fall für $|I| = |J| = 3$. Seien $I = \{0, \alpha, \beta\}$ und $J = \{0, \gamma, \delta\}$. Dann besitzt $M_{IJ}$ die Form 
\begin{equation*}
    \begin{pmatrix}
        1 & 1 & 1 \\
        1 & x^{\alpha\gamma} & x^{\beta\gamma} \\
        1 & x^{\alpha\delta} & x^{\beta\delta}
    \end{pmatrix}.
\end{equation*}

Mithilfe von Zeilenumformungen und anschließender Spaltenentwicklung ergibt sich die Determinante wie folgt:

\begin{align*}
    \det \begin{pmatrix}
        1 & 1 & 1 \\
        1 & x^{\alpha\gamma} & x^{\beta\gamma} \\
        1 & x^{\alpha\delta} & x^{\beta\delta}
    \end{pmatrix} 
    &= \det \begin{pmatrix}
        1 & 1 & 1 \\
        0 & x^{\alpha\gamma} -1 & x^{\beta\gamma} -1 \\
        0 & x^{\alpha\delta} -1 & x^{\beta\delta} -1
    \end{pmatrix} \\
    &= \det \begin{pmatrix}
        x^{\alpha\gamma} -1 & x^{\beta\gamma} -1 \\
        x^{\alpha\delta} -1 & x^{\beta\delta} -1
    \end{pmatrix} \\
    &= (x^{\alpha\gamma} -1)(x^{\beta\delta} -1) - (x^{\alpha\delta} -1)(x^{\beta\gamma} -1)
\end{align*}

Für ein $a \in \field{p}[n]$ ist die Submatrix $M_{IJ}$ somit genau dann nicht invertierbar, falls $a$ die Gleichung $(x^{\alpha\gamma} -1)(x^{\beta\delta} -1) = (x^{\alpha\delta} -1)(x^{\beta\gamma} -1)$ löst. Da diese Gleichung noch sehr allgemein ist, setzen wir noch konkreter $I = \{0,1,3\}$. Dadurch reduziert sich die Bedingung zu

\begin{align}
        & (x^{\gamma} -1)(x^{3\delta} -1) = (x^{\delta} -1)(x^{3\gamma} -1) \nonumber \\
    \iff & \frac{(x^{3\delta} -1)}{(x^{\delta} -1)} = \frac{(x^{3\gamma} -1)}{(x^{\gamma} -1)} \nonumber \\
    \iff & {x^\delta}^2 + x^{\delta} + 1 = {x^{\gamma}}^2 + x^{\gamma} + 1 \nonumber \\
    \iff & {x^\delta}^2 + x^{\delta} = {x^{\gamma}}^2 + x^{\gamma}. \label{equation:013}
\end{align}

Anders formuliert sind nun zwei Elemente $v_1 = a^\delta$ und $v_2 = a^\gamma$ gesucht, die $x^2 + x = m$ für ein festes $m \in \field{p}[n]$ lösen. Abhängig von der Charakteristik des Körpers existieren zwei unterschiedliche Lösungsformeln.

\begin{lemma}
    Sei $\charac(\field{p}[n]) \neq 2$. Die quadratische Gleichung $x^2 + x = m$ ist genau dann lösbar, wenn die Wurzel $\sqrt{2^{-2} + m}$ existiert. Die Lösungen lauten dann
    \begin{equation*}
        v_{1,2} = -2^{-1} \pm \sqrt{2^{-2} + m}.
    \end{equation*}
\end{lemma}

\begin{lemma}[{\cite{QuadEquChar2}}]
    Sei $\charac(\field{p}[n]) = 2$. Die quadratische Gleichung $x^2 + x = m$ ist genau dann lösbar, wenn $\Tr(m) = 0$ gilt. Die erste Lösung lautet dann
    \begin{equation*}
        v_1 = 1 + \sum_{k=1}^{n-1} m^{2^k}(\sum_{l=0}^{k-1} u^{2^l})
    \end{equation*}
     für ein beliebiges $u \in \field{p}[n]$ mit $\Tr(u) = 1$. Die zweite Lösung ist $v_2 = v_1 + 1$.
\end{lemma}

Sofern $v_1,v_2 \neq 0$ sind, existieren für jeden Erzeuger $a \in \field{p}[n]$ Exponenten $\gamma$ und $\delta$ mit $v_1 = a^\gamma$ und $v_2 = a^\delta$. Für diese Erzeuger ist dann die \Cref{equation:013} erfüllt und $M_{IJ}$ nicht invertierbar. 

Im Folgenden betrachten wir ein Beispiel für den Körper $\field{11}$, wobei $-2^{-1} = -6 = 5$ und $2^{-2} = 3$ gilt. In \Cref{table:sol_013F11} ist für jedes $m$ die Wurzel $\sqrt{3 + m}$ angegeben. Im Allgemeinen existieren diese, falls $p \neq 2$ gilt, für $\frac{p^n-1}{2} + 1$ Körperelemente \cite{RootsFiniteFields}. In diesem Beispiel somit für 6 der 11 Elemente. Die dritte Zeile enthält die sich daraus ergebenden Lösungen $v_{1,2} = 5 \pm \sqrt{3+m}$. Aus diesen Lösungen und einem passenden Erzeuger lassen sich die Exponenten $\gamma$ und $\delta$ herleiten.

In \Cref{table:subgroupsF11} sind die Untergruppen des $\field{11}$ aufgelistet. Wir wählen $v_1 = 3$ und $v_2 = 7$. Sei zum Beispiel der Erzeuger $a = 8$. Dann ist $3 = 8^6$ und $7 = 8^9$. Somit ist die \Cref{equation:013} durch
\begin{align*}
    {8^6}^2 + 8^6 &= {8^{9}}^2 + 8^9 \\
    3^2 + 3 &= 7^2 + 7 \\
    1 &= 1
\end{align*}
erfüllt und $(\det M_{IJ})(8) = 0$ für $J = \{0,6,9\}$. 

Die Fälle, bei denen $v_1 \neq v_2 = 0$ ist, müssen hier nicht weiter betrachtet werden, da diese Werte nicht in der selben Untergruppe auftauchen können. Falls $v_1$ oder $v_2$ gleich $1$ ist, muss einer der Exponenten $\delta$ oder $\gamma$ gleich $0$ sein. Weil dieser Exponent nicht doppelt in $J$ vorkommen kann, ist dieser Fall ebenfalls nicht relevant. Gleiches gilt für $v_1 = v_2$. Alle zulässigen Lösungen $v_1,v_2$ führen zu folgenden Indexmengen:

{\renewcommand{\arraystretch}{1.5}
\begin{table}
    \centering
    \begin{tabular}{|C|C|C|C|C|C|C|C|C|C|C|C|C}
    \hline
    m          & 0    & 1   & 2   & 3 & 4 & 5 & 6   & 7  & 8   & 9   & 10 \\
    \hline
    \sqrt{3+m} & 5    & 2   & 4   & - & - & - & 3   & -  & 0   & 1   & -  \\
    \hline
    v_{1,2} = 5 \pm \sqrt{3+m}   & 10,0 & 3,7 & 9,1 & - & - & - & 8,2 & -  & 5,5 & 6,4 & -  \\
    \hline
    \end{tabular}
    \caption{Lösungen $v_{1,2}$ zu $x^2 + x = m$ für festes $m \in \field{11}$.} \label{table:sol_013F11}
\end{table}
}

\begin{table}[]
    \centering
    \begin{tabular}{|C|C|C|C|C|C|C|C|C|C|}
    \hline
    \text{\backslashbox{i}{a}} & 2  & 3 & 4 & 5 & 6  & 7  & 8  & 9 & 10 \\ \hline
    0 & 1  & 1 & 1 & 1 & 1  & 1  & 1  & 1 & 1  \\ \hline
    1 & 2  & 3 & 4 & 5 & 6  & 7  & 8  & 9 & 10 \\ \hline
    2 & 4  & 9 & 5 & 3 & 3  & 5  & 9  & 4 & 1  \\ \hline
    3 & 8  & 5 & 9 & 4 & 7  & 2  & 6  & 3 &    \\ \hline
    4 & 5  & 4 & 3 & 9 & 9  & 3  & 4  & 5 &    \\ \hline
    5 & 10 & 1 & 1 & 1 & 10 & 10 & 10 & 1 &    \\ \hline
    6 & 9  &   &   &   & 5  & 4  & 3  &   &    \\ \hline
    7 & 7  &   &   &   & 8  & 6  & 2  &   &    \\ \hline
    8 & 3  &   &   &   & 4  & 9  & 5  &   &    \\ \hline
    9 & 6  &   &   &   & 2  & 8  & 7  &   &    \\ \hline
    10 & 1  &   &   &   & 1  & 1  & 1  &   &    \\ \hline
    \end{tabular}
    \caption{Alle von einem $a \in \field{11}$ erzeugten Untergruppen.} \label{table:subgroupsF11}
\end{table}

{\renewcommand{\arraystretch}{1.5}
\begin{table}
    \centering
    \begin{tabular}{|C|C|C|C|C|C|C|C|C|}
    \hline
    m               & 0 & 1 & o & o + 1 & o^2 & o^2 + 1 & o^2 + o  & o^2 + o + 1 \\
    \hline
    \mathrm{Tr}(m)  & 0 & 1 & 0 & 1     & 0   & 1       & 0        & 1 \\
    \hline
    v_1         & 0 & - & o^2 & - & o^2 + o & - & o & - \\
    \hline
    \end{tabular}
    \caption{Lösungen $v_1$ zu $x^2 + x = m$ für festes $m \in \field{2}[3]$.} \label{table:sol_013F2_3}
\end{table}
}

\begin{table}
    \centering
    \begin{tabular}{|C|C|C|C|C|C|C|C|C|C|C|}
    \hline
    \text{\backslashbox{i}{a}}& o       & o+1     & o^2     & o^2 + 1 & o^2 + o & o^2 + o + 1 \\ \hline
    0 & 1       & 1       & 1       & 1       & 1       & 1           \\ \hline
    1 & o       & o+1     & o^2     & o^2+1   & o^2+o   & o^2+o+1     \\ \hline
    2 & o^2     & o^2+1   & o^2+o   & o^2+o+1 & o       & o+1         \\ \hline
    3 & o+1     & o^2     & o^2+1   & o^2+o   & o^2+o+1 & o           \\ \hline
    4 & o^2+o   & o^2+o+1 & o       & o+1     & o^2     & o^2+1       \\ \hline
    5 & o^2+o+1 & o       & o+1     & o^2     & o^2+1   & o^2+o       \\ \hline
    6 & o^2+1   & o^2+o   & o^2+o+1 & o       & o+1     & o^2         \\ \hline
    7 & 1       & 1       & 1       & 1       & 1       & 1           \\ \hline
    \end{tabular}
    \caption{Alle von einem $a \in \field{2}[3]$ erzeugten Untergruppen.} \label{table:subgroupsF2_3}
\end{table}