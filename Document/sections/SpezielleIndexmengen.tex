\section{Spezielle Indexmengen} \label{sec:spezielleIndexmengen}

Im vorherigen Abschnitten wurden hinreichende Bedingungen zur Invertierbarkeit vorgestellt, die sich grundlegend auf die Struktur der Indexmengen und auf die Ordnungen der Körperelemente bezogen. In diesem Kapitel werden weitere quadratische Submatrizen untersucht, deren Invertierbarkeit sich nicht über die gezeigten Sätze herleiten lässt.

Zuerst werden Indexmengen mit zusammenhängenden Indizes betrachtet. Sei ohne Beschränkung der Allgemeinheit $I = \{0,\dots,k-1\}$ und $J$ beliebig. Die Matrix $M_{IJ}$ entspricht dann der Vandermondematrix $V_{(x^{j_1},\dots,x^{j_k})}$, deren Determinante bekannt ist. \cite{VandermondeDet}
\begin{equation*}
    \det V_{(x^{j_1},\dots,x^{j_k})} = \prod_{1\leq v<w\leq k} (x^{j_v} - x^{j_w})
\end{equation*}


Damit $(\det M_{IJ})(a) = 0$ gilt, müssen wieder zwei Indizes existieren, deren Differenz ein Vielfaches der Ordnung von $a$ entspricht. Dies ist unter anderem der Fall, wenn die Mächtigkeit der Indexmengen größer als die Ordnung ist.

\begin{sloppypar}
    Als Nächstes werden spezielle Indexmengen mit $|I| = |J| = 3$ untersucht. Seien ${I = \{0, 1, 3\}}$ und $J = \{0, \alpha, \beta\}$ für beliebige $\alpha,\beta \in \mathbb{Z}$. Dann besitzt $M_{IJ}$ diese Form.
\end{sloppypar}
\begin{equation*}
    \begin{pmatrix}
        1 & 1 & 1 \\
        1 & x^{\alpha} & x^{\beta} \\
        1 & x^{3\alpha} & x^{3\beta}
    \end{pmatrix}
\end{equation*}

Mithilfe von Zeilenumformungen und anschließender Spaltenentwicklung ergibt sich die Determinante wie folgt.

\begin{align*}
    \det \begin{pmatrix}
        1 & 1 & 1 \\
        1 & x^{\alpha} & x^{\beta} \\
        1 & x^{3\alpha} & x^{3\beta}
    \end{pmatrix}
    &= \det \begin{pmatrix}
        1 & 1 & 1 \\
        0 & x^{\alpha} -1 & x^{\beta} -1 \\
        0 & x^{3\alpha} -1 & x^{3\beta} -1
    \end{pmatrix} \\
    &= \det \begin{pmatrix}
        x^{\alpha} -1 & x^{\beta} -1 \\
        x^{3\alpha} -1 & x^{3\beta} -1
    \end{pmatrix} \\
    &= (x^{\alpha} -1)(x^{3\beta} -1) - (x^{3\alpha} -1)(x^{\beta} -1)
\end{align*}

Für ein $a \in \field{p}[n]^\times$ ist die Submatrix $M_{IJ}$ somit genau dann nicht invertierbar, falls $a$ die Gleichung $(x^{\alpha} -1)(x^{3\beta} -1) = (x^{3\alpha} -1)(x^{\beta} -1)$ löst. Diese Bedingung reduziert sich weiter zu:

\begin{align}
        & (x^{\alpha} -1)(x^{3\beta} -1) = (x^{3\alpha} -1)(x^{\beta} -1) \nonumber \\
    \iff & \frac{(x^{3\beta} -1)}{(x^{\beta} -1)} = \frac{(x^{3\alpha} -1)}{(x^{\alpha} -1)} \nonumber \\
    \iff & {x^\beta}^2 + x^{\beta} + 1 = {x^{\alpha}}^2 + x^{\alpha} + 1 \nonumber \\
    \iff & {x^\beta}^2 + x^{\beta} = {x^{\alpha}}^2 + x^{\alpha} \label{equation:013}
\end{align}

Anders formuliert sind nun zwei Elemente $v_1 = a^\alpha$ und $v_2 = a^\beta$ gesucht, die $x^2 + x = m$ für ein festes $m \in \field{p}[n]$ lösen. Abhängig von der Charakteristik des Körpers existieren zwei unterschiedliche Lösungsformeln.

\begin{lemma}[{\cite[S. 4]{QuadEquCharNot2}}]
    Sei $\charac(\field{p}[n]) \neq 2$. Die quadratische Gleichung $x^2 + x = m$ ist genau dann lösbar, wenn die Wurzel $\sqrt{2^{-2} + m}$ existiert. Die Lösungen lauten dann
    \begin{equation*}
        v_{1,2} = -2^{-1} \pm \sqrt{2^{-2} + m}.
    \end{equation*}
\end{lemma}

\begin{lemma}[{\cite[S. 140]{QuadEquChar2}}]
    Sei $\charac(\field{p}[n]) = 2$. Die quadratische Gleichung $x^2 + x = m$ ist genau dann lösbar, wenn $\Tr(m) = 0$ gilt. Die erste Lösung lautet dann
    \begin{equation*}
        v_1 = 1 + \sum_{k=1}^{n-1} m^{2^k}(\sum_{l=0}^{k-1} u^{2^l})
    \end{equation*}
     für ein beliebiges $u \in \field{p}[n]$ mit $\Tr(u) = 1$. Die zweite Lösung ist $v_2 = v_1 + 1$.
\end{lemma}

Sofern die Lösungen $v_1$ und $v_2$ existieren, liefert ein gemeinsamer Erzeuger $a$ die gesuchten Exponenten $\alpha$ und $\beta$. Da die Potenzen $v_1 = a^\alpha$ und $v_2 = a^\beta$ \Cref{equation:013} lösen, ist $M_{IJ}$ mit $J = \{0,1,3\}$ und $J = \{0,\alpha,\beta\}$ für dieses $a$ nicht invertierbar.

\sloppy Im Folgenden wird ein Beispiel für $\field{11}$ betrachtet. In diesem Körper gelten ${-2^{-1} = -6 = 5}$ und $2^{-2} = 3$. \Cref{table:sol_013F11} gibt für jedes $m$ die Wurzel $\sqrt{3 + m}$ an. Im Allgemeinen existieren diese, falls $p \neq 2$ gilt, für $\frac{p^n-1}{2} + 1$ Körperelemente \cite{RootsFiniteFields}. In diesem Beispiel somit für sechs der elf Elemente. Die dritte Zeile enthält die sich daraus ergebenden Lösungen $v_{1,2} = 5 \pm \sqrt{3+m}$.
In \Cref{table:subgroupsF11} sind die Untergruppen von $\field{11}$ aufgelistet. Für die Wahl von $m=1$ ergeben sich zum Beispiel $v_1 = 3$ und $v_2 = 7$. Ein möglicher Erzeuger ist $a = 8$ mit $v_1 = 8^6$ und $v_2 = 8^9$. Somit ist die \Cref{equation:013} erfüllt und $(\det M_{IJ})(8) = 0$ für $I = \{0,1,3\}$ und $J = \{0,6,9\}$.

\begin{align*}
    {8^6}^2 + 8^6 &= {8^{9}}^2 + 8^9 \\
    3^2 + 3 &= 7^2 + 7 \\
    9 + 3 &= 5 + 7
\end{align*}

Die Fälle, bei denen $v_1$ oder $v_2$ gleich 0 sind, müssen hier nicht weiter betrachtet werden, da diese Werte nicht in der selben Untergruppe auftauchen können. Falls $v_1$ oder $v_2$ gleich $1$ sind, muss einer der Exponenten $\alpha$ oder $\beta$ gleich $0$ sein. Weil dieser Exponent nicht doppelt in der Indexmenge $J$ vorkommen soll, ist dieser Fall ebenfalls nicht relevant. Gleiches gilt für $v_1 = v_2$. Alle zulässigen Lösungen $v_1$ und $v_2$ führen zu den folgenden Indexmengen.

\begin{align*}
    &a = 6:\quad J=\{0,2,3\}, J=\{0,7,9\} \text{ und } J=\{0,1,8\} \\
    &a = 7:\quad J=\{0,1,4\}, J=\{0,3,9\} \text{ und } J=\{0,6,7\} \\
    &a = 8:\quad J=\{0,6,9\}, J=\{0,1,7\} \text{ und } J=\{0,3,4\}     
\end{align*}

{\renewcommand{\arraystretch}{1.5}
\begin{table}
    \centering
    \begin{tabular}{|C|C|C|C|C|C|C|C|C|C|C|C|C}
    \hline
    m          & 0    & 1   & 2   & 3 & 4 & 5 & 6   & 7  & 8   & 9   & 10 \\
    \hline
    \sqrt{3+m} & 5    & 2   & 4   & - & - & - & 3   & -  & 0   & 1   & -  \\
    \hline
    v_{1,2} = 5 \pm \sqrt{3+m}   & 10,0 & 3,7 & 9,1 & - & - & - & 8,2 & -  & 5,5 & 6,4 & -  \\
    \hline
    \end{tabular}
    \caption{Lösungen $v_{1,2}$ zu $x^2 + x = m$ für festes $m \in \field{11}$} \label{table:sol_013F11}
\end{table}
}

\begin{table}[]
    \centering
    \begin{tabular}{|C|C|C|C|C|C|C|C|C|C|}
    \hline
    \text{\backslashbox{i}{a}} & 2  & 3 & 4 & 5 & 6  & 7  & 8  & 9 & 10 \\ \hline
    0 & 1  & 1 & 1 & 1 & 1  & 1  & 1  & 1 & 1  \\ \hline
    1 & 2  & 3 & 4 & 5 & 6  & 7  & 8  & 9 & 10 \\ \hline
    2 & 4  & 9 & 5 & 3 & 3  & 5  & 9  & 4 & 1  \\ \hline
    3 & 8  & 5 & 9 & 4 & 7  & 2  & 6  & 3 &    \\ \hline
    4 & 5  & 4 & 3 & 9 & 9  & 3  & 4  & 5 &    \\ \hline
    5 & 10 & 1 & 1 & 1 & 10 & 10 & 10 & 1 &    \\ \hline
    6 & 9  &   &   &   & 5  & 4  & 3  &   &    \\ \hline
    7 & 7  &   &   &   & 8  & 6  & 2  &   &    \\ \hline
    8 & 3  &   &   &   & 4  & 9  & 5  &   &    \\ \hline
    9 & 6  &   &   &   & 2  & 8  & 7  &   &    \\ \hline
    10 & 1  &   &   &   & 1  & 1  & 1  &   &    \\ \hline
    \end{tabular}
    \caption{Alle von einem $a \in \field{11}$ erzeugten Untergruppen} \label{table:subgroupsF11}
\end{table}

Ein weiteres Beispiel betrachtet $\field{2}[3]$. Analog zeigt \Cref{table:sol_013F2_3} für jedes $m$ mit $\Tr(m) = 0$ eine Lösung der quadratischen Gleichung $x^2+x=m$. Für $m=y^2$ ergibt sich zum Beispiel $v_1 = y^2 + y$ und $v_2 = y^2 + y + 1$. Wird der Erzeuger $a = y+1$ mit $v_1 = (y+1)^6$ und $v_2 = (y+1)^4$ gewählt, so wird wieder \Cref{equation:013} erfüllt.
\begin{align*}
    {(y+1)^6}^2 + (y+1)^6 &= {(y+1)^4}^2 + (y+1)^4 \\
    (y^2 + y)^2 + (y^2 + y) &= (y^2 + y + 1)^2 + (y^2 + y + 1) \\
    y + (y^2 + y) &= (y+1) + (y^2 + y + 1) \\
    y^2 &= y^2 \\
\end{align*}
Daraus folgt $(\det M_{IJ})(y+1) = 0$ für $I = \{0,1,3\}$ und $J = \{0,4,6\}$.
Da die Gruppenordnung von $\field{2}[3]$ eine Primzahl ist, ist jedes Körperelement $a \notin \{0,1\}$ ein Erzeuger. Somit existiert für jedes dieser Elemente eine Indexmenge $J$, sodass $(\det M_{IJ})(a) = 0$ gilt.

{\renewcommand{\arraystretch}{1.5}
\begin{table}
    \centering
    \begin{tabular}{|C|C|C|C|C|C|C|C|C|}
    \hline
    m               & 0 & 1 & y & y + 1 & y^2 & y^2 + 1 & y^2 + y  & y^2 + y + 1 \\
    \hline
    \mathrm{Tr}(m)  & 0 & 1 & 0 & 1     & 0   & 1       & 0        & 1 \\
    \hline
    v_1         & 0 & - & y^2 & - & y^2 + y & - & y & - \\
    \hline
    \end{tabular}
    \caption{Lösungen $v_1$ zu $x^2 + x = m$ für festes $m \in \field{2}[3]$} \label{table:sol_013F2_3}
\end{table}
}

\begin{table}
    \centering
    \begin{tabular}{|C|C|C|C|C|C|C|C|C|C|C|}
    \hline
    \text{\backslashbox{i}{a}}& y       & y+1     & y^2     & y^2 + 1 & y^2 + y & y^2 + y + 1 \\ \hline
    0 & 1       & 1       & 1       & 1       & 1       & 1           \\ \hline
    1 & y       & y+1     & y^2     & y^2+1   & y^2+y   & y^2+y+1     \\ \hline
    2 & y^2     & y^2+1   & y^2+y   & y^2+y+1 & y       & y+1         \\ \hline
    3 & y+1     & y^2     & y^2+1   & y^2+y   & y^2+y+1 & y           \\ \hline
    4 & y^2+y   & y^2+y+1 & y       & y+1     & y^2     & y^2+1       \\ \hline
    5 & y^2+y+1 & y       & y+1     & y^2     & y^2+1   & y^2+y       \\ \hline
    6 & y^2+1   & y^2+y   & y^2+y+1 & y       & y+1     & y^2         \\ \hline
    7 & 1       & 1       & 1       & 1       & 1       & 1           \\ \hline
    \end{tabular}
    \caption{Alle von einem $a \in \field{2}[3]$ erzeugten Untergruppen} \label{table:subgroupsF2_3}
\end{table}