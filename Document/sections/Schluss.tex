\section{Zusammenfassung}

Diese Bachelorarbeit beschäftigte sich mit der unendlichen Matrix $M = \left( x^{ij} \right)_{i,j \in \mathbb{Z}}$. Untersucht wurde die Invertierbarkeit beliebiger quadratischer Submatrizen nach Substitution der Variable $x$ durch ein invertierbares Element eines endlichen Körpers $\field{p}[n]$. Diese Submatrizen $M_{IJ} \coloneqq \left( x^{ij} \right)_{i \in I,j \in J}$ wurden durch jeweils zwei endliche Indexmengen $I,J \subseteq \mathbb{Z}$ indiziert. 

Zunächst wurden in \Cref{sec:grundlagen} die Grundlagen zu endlichen Körpern erläutert. Dazu zählen neben der Konstruktion dieser algebraischen Strukturen besonders auch die zyklischen Eigenschaften der multiplikativen Gruppe $\field{}^\times \coloneqq \field{} \setminus \{0\}$. So besitzt jedes Element $a \in \field{p}[n]^\times$ eine Ordnung $o \in \mathbb{N}_+$ mit $a^o = 1$.

Die Minoren der Matrix $M$ können als Polynome aus $\field{p}[n]^\times [X]$ aufgefasst werden. Eine Nullstelle $a$ weist demnach auf eine singuläre Submatrix hin. In \Cref{sec:indextransformationen} wurden affine Indextransformationen $f:\mathbb{Z} \rightarrow \mathbb{Z}, i \mapsto a\cdot i + m$ für $a,m \in \mathbb{Z}$ auf die Indexmengen $I,J \subseteq \mathbb{Z}$ angewendet, um ihren Einfluss auf die Minoren herzuleiten. Die Addition von Konstanten aus $\mathbb{Z}$ auf die Indexmengen hat demnach keinen Einfluss auf die Nullstellen der Minoren. Demnach genügt es, normalisierte Indexmengen $I,J$ mit $\min(I) = \min(J) = 0$ zu betrachten. Die Multiplikation mit einer Konstanten $a \in \mathbb{Z}\setminus\{0\}$ entspricht der Substitution der Variable $x$ durch $x^a$. Speziell für die Multiplikation mit $-1$ ergaben sich Symmetrien zum multiplikativ Inversen $x^{-1}$. Mögliche Zusammenhänge der Nullstellen von $\det M_{aI\,bJ}$ für verschiedene $a,b\in \mathbb{Z}$ ist Gegenstand weiterer Forschung.

In \Cref{sec:singulaereSubmatrizen} wurden singuläre Submatrizen anhand einer gewählten Gruppenordnung identifiziert. Dazu wurde ein hinreichendes Kriterium erarbeitet, welches sich auf die Abstände innerhalb der Indexmengen $I$ und $J$ bezieht. Sei $a\in \field{p}[n]^\times$ das in den Minor $\det M_{IJ}$ einzusetzende Element mit $v\cdot w = \ord{a}$. Dann gilt $(\det M_{IJ})(a) = 0$, wenn für ein $k\in \mathbb{Z}$ mindestens $k$ unterschiedliche Indizes aus $I$ in derselben Restklasse modulo $v$ liegen und alle $k$-elementigen Teilmengen von $J$ zwei Indizes aus derselben Restklasse modulo $w$ enthalten. Zukünftig könnten weitere hinreichende Kriterien zur Invertierbarkeit erforscht werden.

Abschließend wurden in \Cref{sec:spezielleIndexmengen} spezielle Indexmengen mit festem $I = \{0,1,3\}$ und beliebigem $J = \{0,\alpha,\beta\}$ untersucht. Die Bedingung an die Invertierbarkeit von $M_{IJ}$ reduzierte sich auf die Lösbarkeit quadratischer Gleichungen. Mithilfe von Erzeugern ließen sich die Lösungen der Gleichungen und demnach die Exponenten $\alpha$ und $\beta$ herleiten.