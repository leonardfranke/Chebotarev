\section{Zusammenfassung}

Diese Bachelorarbeit beschäftigte sich mit den Minoren spezieller Vandermondematrizen der Form 
\begin{equation*}
    \left( x^{ij} \right)_{0\leq i,j \leq k-1} = \begin{pmatrix}
        1     & 1    & 1    & 1    &\cdots& 1 \\
        1     & x^1  & x^2  & x^3  &\cdots& x^{k-1} \\
        1     & x^2  & x^4  & x^6  &\cdots& x^{2k-2} \\
        1     & x^3  & x^6  & x^9  &\cdots& x^{3k-3} \\
        \vdots&\vdots&\vdots&\vdots&\ddots&\vdots \\
        1     &x^{k-1}&x^{2k-2}&x^{3k-3}&\cdots&x^{(k-1)(k-1)}
    \end{pmatrix}.
\end{equation*}
Für die Variable $x$ wurden Elemente endlicher Körper eingesetzt, um daraufhin die Submatrizen auf Invertierbarkeit zu untersuchen. Zunächst wurden in \Cref{sec:grundlagen} die grundlegenden Definitionen und Sätze zu den endlichen Körpern erläutert. Dazu zählt die Konstruktion mittels Restklassen der ganzen Zahlen und Polynome, sowie die zyklischen Eigenschaften der multiplikativen Gruppe.

Die zu untersuchenden Submatrizen wurden allgemein über Indexmengen der ganzen Zahlen indiziert. In \Cref{sec:indextransformationen} wurden affine Indextransformationen auf diese Mengen angewendet, um den Zusammenhang zwischen zueinander verschobenen und skalierten Indexmengen zu untersuchen. Die Verschiebung der Submatrix hat keinen Einfluss auf die Nullstellen der Minoren. Die Skalierung um eine Konstante entspricht der Substitution von $x^a$. Aus den Ergebnissen folgten Symmetrien in der graphischen Repräsentation der Indexmengen zum Inversen.



\begin{itemize}
    \item Grundlagen zuerst geklärt
    \item indextransformationen darauffolgend (Zusammenhang zu Verschobenen und Skalierten Indexmengen untersucht, Ergaben sich symmetrien bzgl negierter Indexmengen und dem Inversen)
    \item Hinreichendes Kriterium auf Basis der Ordnungen und Abständen in den Indexmengen weiter verallgemeinert
    \item Konkrete Indexmengen mit 013 angeguckt
\end{itemize}