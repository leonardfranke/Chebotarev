\section{Einleitung}

Diese Bachelorarbeit beschäftigt sich mit den Minoren verallgemeinerter Vandermondematrizen. Die betrachteten Matrizen besitzen die Form \begin{equation*}
    \left( x^{ij} \right)_{0\leq i,j \leq k-1} = \begin{pmatrix}
        1     & 1    & 1    & 1    &\cdots& 1 \\
        1     & x^1  & x^2  & x^3  &\cdots& x^{k-1} \\
        1     & x^2  & x^4  & x^6  &\cdots& x^{2k-2} \\
        1     & x^3  & x^6  & x^9  &\cdots& x^{3k-3} \\
        \vdots&\vdots&\vdots&\vdots&\ddots&\vdots \\
        1     &x^{k-1}&x^{2k-2}&x^{3k-3}&\cdots&x^{(k-1)(k-1)}
    \end{pmatrix},
\end{equation*}
wobei für $x$ stets Elemente eines endlichen Körpers $\field{p}[n]$ eingesetzt werden. Für gegebenes $k$ sind all die Elemente $a$ gesucht, sodass jede Untermatix von $\left( a^{ij} \right)_{0\leq i,j \leq k-1}$ invertierbar ist. 
Diese Untermatizen ergeben sich dann aus der Wahl zweier gleich mächtiger Indexmengen $I,J \subseteq \{0, \dots, k-1\}$. Da für die relevanten Körperelemente auch negative Exponenten zulässig sind, wird im Hauptteil dieser Arbeit die unendliche Matrix $M = \left( x^{ij} \right)_{i,j \in \mathbb{Z}}$ betrachtet. Entsprechend werden diese Submatrizen mit endlichen Teilmengen $I,J \subseteq \mathbb{Z}$ indiziert.

\begin{comment}
    Ein Eintrag $x^{ij}$ kann als Monom aus dem Polynomring $\field{p}[n][X]$ verstanden werden. Aufgrund der Abgeschlossenheit der Polynomringe bilden die Minoren $\det M_{IJ}$ ebenfalls Polynome.
\end{comment}
    
Für eine ähnliche Konstruktion über den komplexen Zahlen $\mathbb{C}$ gibt es bereits Resultate bezüglich der Minoren. Der Mathematiker Chebotarev zeigte 1926, dass für primitive $p$-te Einheitswurzel (zum Beispiel $a=e^{2\pi i/p}$) alle Untermatizen von $\left( a^{ij} \right)_{0\leq i,j \leq p-1}$ invertierbar sind, falls $p$ Primzahl ist. \cite{CheboProof}
Analog existiert eine Aussage über $p$-te primitive Einheitswurzeln endlicher Körper. Diese gilt jedoch nur für spezielle Primzahlpotenzen $q^{p-1}$, bei denen $q$ sehr groß sein muss. \cite{CheboFiniteFields}
