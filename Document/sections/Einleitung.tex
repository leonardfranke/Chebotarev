\section{Einleitung}

Diese Bachelorarbeit beschäftigt sich mit den Minoren verallgemeinerter Vandermondematrizen. Die betrachteten Matrizen besitzen die Form \begin{equation*}
    \left( a^{ij} \right)_{1\leq i,j \leq n-1} = \begin{pmatrix}
        1     & 1    & 1    & 1    &\cdots& 1 \\
        1     & a^1  & a^2  & a^3  &\cdots& a^{n-1} \\
        1     & a^2  & a^4  & a^6  &\cdots& a^{2n-2} \\
        1     & a^3  & a^6  & a^9  &\cdots& a^{3n-3} \\
        \vdots&\vdots&\vdots&\vdots&\ddots&\vdots \\
        1     &a^{n-1}&a^{2n-2}&a^{3n-3}&\cdots&a^{(n-1)(n-1)}
    \end{pmatrix},
\end{equation*}
wobei $a$ stets Element eines endlichen Körpers ist. Für gegebenes $n$ suchen wir alle $a$, sodass jede Untermatix von $\left( a^{ij} \right)_{1\leq i,j \leq n-1}$ invertierbar ist.
Diese Untermatizen ergeben sich dann aus der Wahl zweier gleich mächtiger Indexmengen $I,J \subseteq \{0, \dots, n-1\}$.

Für eine ähnliche Konstruktion über den komplexen Zahlen $\mathbb{C}$ gibt es bereits Resultate bezüglich der Minoren. Der Mathematiker Chebotarev zeigte 1926, dass für primitive $p$-te Einheitswurzel (zum Beispiel $a=e^{2\pi i/p}$) alle Untermatizen von $\left( a^{ij} \right)_{1\leq i,j \leq p-1}$ invertierbar sind, falls $p$ Primzahl ist. \cite{CheboProof}
Analog existiert eine Aussage über $p$-te primitive Einheitswurzeln endlicher Körper. Diese gilt jedoch nur für spezielle Primzahlpotenzen $q^{p-1}$, bei denen $q$ sehr groß sein muss. \cite{CheboFiniteFields}
