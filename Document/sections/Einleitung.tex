\section{Einleitung}

Diese Bachelorarbeit beschäftigt sich mit den Minoren spezieller Vandermonde-Matrizen. Die betrachteten Matrizen besitzen die Form \begin{equation*}
    \left( x^{ij} \right)_{0\leq i,j \leq k-1} = \begin{pmatrix}
        1     & 1    & 1    & 1    &\cdots& 1 \\
        1     & x^1  & x^2  & x^3  &\cdots& x^{k-1} \\
        1     & x^2  & x^4  & x^6  &\cdots& x^{2k-2} \\
        1     & x^3  & x^6  & x^9  &\cdots& x^{3k-3} \\
        \vdots&\vdots&\vdots&\vdots&\ddots&\vdots \\
        1     &x^{k-1}&x^{2k-2}&x^{3k-3}&\cdots&x^{(k-1)(k-1)}
    \end{pmatrix},
\end{equation*}
wobei für die Variable $x$ stets Elemente eines Körpers $\field{}$ eingesetzt werden. Für gegebenes $k$ sind all die Elemente $a$ des Körpers gesucht, sodass jede quadratische Submatrix von $\left( a^{ij} \right)_{0\leq i,j \leq k-1}$ invertierbar ist. Der Mathematiker Chebotarev beschäftigte sich mit der Substitution von komplexen Zahlen. Konkret zeigt er im Jahr 1926, dass für primitive $p$-te Einheitswurzel (zum Beispiel $a=e^{2\pi i/p}$) genau dann alle quadratischen Submatrizen von $\left( a^{ij} \right)_{0\leq i,j \leq p-1}$ invertierbar sind, wenn $p$ Primzahl ist. Bis auf einen Vorfaktor entspricht dieses Beispiel genau der diskreten Fourier-Matrix \cite{CheboProof}. Analog existiert eine Aussage über $p$-te primitive Einheitswurzeln endlicher Körper $\field{q}[p-1]$. Diese gilt jedoch nur für Primzahlen $q$ und $p$ mit fester Ordnung $\ord{q}[p] = p-1$ und genügend großem $q> \max\left\{\frac{V_{(a_0,\dots,a_{k-1})}}{V_{(0,\dots,k-1)}} : 0\leq a_0 < \cdots < a_{k-1} \leq p-1,\; 2\leq k \leq p-1 \right\}$, wobei $V_{(a_0,\dots,a_{k-1})} \coloneqq ({a_i}^j)_{0\leq i,j \leq k-1}$ die \emph{Standard-Vandermonde-Matrix} der Größe $k$ ist \cite{CheboFiniteFields}. Matrizen über endlichen Körpern mit vielen invertierbaren Submatrizen sind unter anderem in der Codierungstheorie von Bedeutung. Für die Konstruktion linearer Codes mit hohem Minimalabstand $d$ müssen je $d$ Spalten einer gegebenen Matrix linear unabhängig sein \cite{CodingTheory}. 

Diese Arbeit betrachtet ebenfalls endliche Körper $\field{p}[n]$ und untersucht die Invertierbarkeit der Submatrizen für Körperelemente beliebiger Ordnung. Da die Strukturen der Submatrizen beim Substituieren der $0$ trivial sind, werden ausschließlich invertierbare Elemente aus $\field{p}[n]^\times$ eingesetzt. Konkret leben die betrachteten Matrizen im $(\field{p}[n]^\times [X])^{k \times k}$ und die Minoren demnach im Polynomring $\field{p}[n]^\times [X]$. Da für die invertierbaren Körperelemente auch ganzzahlige negative Exponenten definiert sind, wird im Hauptteil dieser Arbeit die unendliche Matrix $M = \left( x^{ij} \right)_{i,j \in \mathbb{Z}}$ analysiert. Die dazugehörigen Submatrizen werden mit jeweils zwei endlichen Teilmengen $I,J$ der ganzen Zahlen indiziert. Sei entsprechend $M_{IJ} \coloneqq \left( x^{ij} \right)_{i \in I,j \in J}$. Für zum Beispiel $I,J = \{-2, \dots, 4\}$ sieht diese folgendermaßen aus.
\begin{equation*}
    M_{IJ} = \begin{pmatrix}
        x^4     & x^{2} & 1    & x^{-2} & x^{-4}& x^{-6}& x^{-8}\\
        x^{2}   & x^{1} & 1    & x^{-1} & x^{-2}& x^{-3}& x^{-4}\\
        1       & 1     & 1    & 1      & 1     & 1     & 1     \\
        x^{-2}  & x^{-1}& 1    & x^{1}  & x^{2} & x^{3} & x^{4} \\
        x^{-4}  & x^{-2}& 1    & x^{2}  & x^{4} & x^{6} & x^{8} \\
        x^{-6}  & x^{-3}& 1    & x^{3}  & x^{6} & x^{9} & x^{12} \\
        x^{-8}  & x^{-4}& 1    & x^{4}  & x^{8} & x^{12} & x^{16} \\
    \end{pmatrix}
\end{equation*}

Die Sortierung beim Angeben der Indexmengen ist irrelevant, da das Austauschen mehrerer Zeilen oder Spalten bei den Determinanten nur zum Vorzeichenwechsel führt. Aufgrund von zyklischen Eigenschaften der multiplikativen Gruppe $\field{p}[n]^\times$ nimmt die unendliche Matrix $M$ stets die Form einer Blockmatrix an. Für $\field{3}$, den Körper mit 3 Elementen, vereinfacht sich die bereits gezeigte Submatrix wie folgt.

\begin{equation*}
    M_{IJ} = \left( \begin{array}{cc|ccc|cc}
        x^1     & x^{2} & 1     & x^{1} & x^{2} & 1 & x^{1} \\
        x^{2}   & x^{1} & 1     & x^{2} & x^{1} & 1 & x^{2} \\
        \hline
        1       & 1     & 1     & 1     & 1     & 1 & 1     \\
        x^{1}   & x^{2} & 1     & x^{1} & x^{2} & 1 & x^{1} \\
        x^{2}   & x^{1} & 1     & x^{2} & x^{1} & 1 & x^{2} \\
        \hline
        1       & 1     & 1     & 1     & 1     & 1 & 1     \\
        x^1     & x^2   & 1     & x^1   & x^2   & 1 & x^1   \\
    \end{array} \right)
\end{equation*}

Diese Arbeit ist wie folgt gegliedert: In \Cref{sec:grundlagen} werden die benötigten Grundlagen zu endlichen Körpern und zyklischen Gruppen geklärt. Der Einfluss von affinen Abbildungen auf die Indexmengen und die dazugehörigen Minoren wird in \Cref{sec:indextransformationen} untersucht. In \Cref{sec:singulaereSubmatrizen} wird ein hinreichendes Kriterium erarbeitet, das angibt, welche Indexmengen für eine feste Gruppenordnung zu singulären Submatrizen führen. Weitere spezielle Indexmengen, deren Invertierbarkeit sich nicht über die vorher gezeigten Sätze herleiten lässt, werden abschließend in \Cref{sec:spezielleIndexmengen} untersucht.
