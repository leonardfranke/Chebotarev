\section{Singuläre Submatrizen} \label{sec:singulaereSubmatrizen}

Sei weiterhin $M = \left( x^{ij} \right)_{i,j \in \mathbb{Z}}$ die unendliche Matrix und $M_{IJ} \coloneqq \left( x^{ij} \right)_{i \in I,j \in J}$ die dazugehörige Submatrix zu den Indexmengen $I,J \subseteq \mathbb{Z}$. Dieses Kapitel befasst sich mit der Invertierbarkeit der $M_{IJ}$ für gegebenes $a \in \field{p}[n]^\times$. Dazu wird ein hinreichendes Kriterium erarbeitet, das angibt, für welche Indexmengen die Submatrix $M_{IJ}$ nicht invertierbar ist. Wieder können Eigenschaften der zyklischen Gruppe $\field{p}[n]^\times$ ausgenutzt werden. Die in diesem Kapitel gezeigten Sätze basieren vorrangig auf der Gleichung

\begin{equation*}
    a^i = a^j \iff i = j \bmod \ord{a},
\end{equation*}

welche sich direkt aus \Cref{satz:cyclicity} ergibt. Zwei Potenzen sind somit genau dann gleich, wenn sich die Exponenten um ein Vielfaches der Ordnung unterscheiden. Diese Bedingung gilt ebenso für die Gleichheit zweier Zeilen/Spalten der Matrizen $M_{IJ}$. 

Sei für die folgenden Beispiele stets $a = 2 \in \field{11}$ mit der Ordnung $\ord{a} = 10$. Seien zunächst $I=\{0,1,6,11\}$ und $J=\{0,1,2,3\}$. Dann ist 
\begin{equation*}
    M_{IJ}(2) = \begin{pmatrix}
        2^{0} & 2^{0} & 2^{0} & 2^{0} \\
        2^{0} & 2^{1} & 2^{2} & 2^{3} \\
        2^{0} & 2^{6} & 2^{12} & 2^{18} \\
        2^{0} & 2^{11} & 2^{22} & 2^{33} 
    \end{pmatrix} \underbrace{=}_{2^{10}=1} \begin{pmatrix}
        2^{0} & 2^{0} & 2^{0} & 2^{0} \\
        2^{0} & 2^{1} & 2^{2} & 2^{3} \\
        2^{0} & 2^{6} & 2^{2} & 2^{8} \\
        2^{0} & 2^{1} & 2^{2} & 2^{3} 
    \end{pmatrix}.
\end{equation*}

Da die Differenz $i_4 - i_2 = 11 - 1 = 10$ der Ordnung von $a$ entspricht, unterscheiden sich die Exponenten in den Zeilen 2 und 4 um ein Vielfaches dieser Ordnung. Dadurch sind diese Zeilen äquivalent und $(\det M_{IJ})(a) = 0$.
Seien nun $I=\{0,2,6,7\}$ und $J=\{0,2,4,6\}$. In diesen Indexmengen existieren keine zwei Elemente mit dem passenden Abstand von $10$. Die zweite und vierte Zeile der Submatrix sind dennoch wieder äquivalent.

\begin{equation*}
    M_{IJ}(2) = \begin{pmatrix}
        2^{0} & 2^{0} & 2^{0} & 2^{0} \\
        2^{0} & 2^{4} & 2^{8} & 2^{12} \\
        2^{0} & 2^{12} & 2^{24} & 2^{36} \\
        2^{0} & 2^{14} & 2^{28} & 2^{42} 
    \end{pmatrix} \underbrace{=}_{2^{10}=1} \begin{pmatrix}
        2^{0} & 2^{0} & 2^{0} & 2^{0} \\
        2^{0} & 2^{4} & 2^{8} & 2^{2} \\
        2^{0} & 2^{2} & 2^{4} & 2^{6} \\
        2^{0} & 2^{4} & 2^{8} & 2^{2} 
    \end{pmatrix}
\end{equation*}

Dies ergibt sich aus der Faktorisierung von $10 = 5 \cdot 2$. Die Differenz ${i_4 - i_2 = 7 - 2 = 5}$ entspricht dem ersten Faktor. Da alle Differenzen der Indizes in $J$ ein Vielfaches der $2$ sind, unterscheiden sich die Exponenten in den Zeilen 2 und 4 wieder um ein Vielfaches der Ordnung, wodurch die lineare Abhängigkeit entsteht. Das folgende Lemma formalisiert dieses Resultat.

\begin{lemma} \label{lemma:equal-columns}
    Sei $M = \left( x^{ij} \right)_{i,j \in \mathbb{Z}}$. Seien weiter $I,J \subseteq \mathbb{Z}$ zwei Indexmengen und $a \in \field{p}[n]^\times$ mit der Ordnung $o = \ord{a} = v\cdot w$. Falls
    \begin{equation} \label{equation:all-equal}
        \forall \; i \neq i' \in I: \quad i = i' \bmod v
    \end{equation}
    und
    \begin{equation} \label{equation:two-equal}
        \exists \; j \neq j' \in J: \quad j = j' \bmod w
    \end{equation}
    gelten, folgt $(\det M_{IJ})(a) = 0$.
\end{lemma}

{
\Crefname{equation}{Bedingung}{Bedingungen}

\begin{proof}
    Seien $j,j' \in J$ die Indizes, welche die obige \Cref{equation:two-equal} erfüllen. Im Folgenden wird gezeigt, dass die dazugehörigen Spalten in $M_{IJ}(a)$ linear abhängig sind. Seien $j = (c + bw)$, $j' = (c + b'w)$ und $I = \{(d+l_1v),\dots,(d+l_kv)\}$ mit $c,b,b',d,l_1,\dots, l_k \in \mathbb{Z}$. Die Spalten zu $j$ und $j'$ haben die folgende Form:
    \begin{align*}
        \begin{pmatrix}
            a^{i_1j} & a^{i_1j'} \\
            \vdots & \vdots \\
            a^{i_kj} & a^{i_kj'}
        \end{pmatrix} &=
        \begin{pmatrix}
            a^{(d+l_1v)(c + bw)} & a^{(d+l_1v)(c + b'w)} \\
            \vdots & \vdots \\
            a^{(d+l_kv)(c + bw)} & a^{(d+l_kv)(c + b'w)}
        \end{pmatrix} \\
        &= \begin{pmatrix}
            a^{dc + dbw +l_1vc + l_1bvw} & a^{dc + db'w +l_1vc + l_1b'vw} \\
            \vdots & \vdots \\
            a^{dc + dbw +l_kvc + l_kbvw} & a^{dc + db'w +l_kvc + l_kb'vw}
        \end{pmatrix}
    \end{align*}
    Die konstanten Terme $a^{dc + dbw}$ und $a^{dc + db'w}$ können aus den Spalten geteilt werden. Da außerdem die Gleichung $a^{vw} = a^{o} = 1$ angewendet werden kann, verbleiben folgende äquivalente Spalten:
    \begin{equation*}
        \begin{pmatrix}
            a^{l_1vc} {a^{vw}}^{l_1b} & a^{l_1vc} {a^{vw}}^{l_1b'} \\
            \vdots & \vdots \\
            a^{l_kvc} {a^{vw}}^{l_kb} & a^{l_kvc} {a^{vw}}^{l_kb'}
        \end{pmatrix} =
        \begin{pmatrix}
            a^{l_1vc} & a^{l_1vc} \\
            \vdots & \vdots \\
            a^{l_kvc} & a^{l_kvc}
        \end{pmatrix}
    \end{equation*}
    Aus der linearen Abhängigkeit der beiden Spalten folgt $(\det M_{IJ})(a) = 0$.
\end{proof}

Für eine Indexmenge $I$ mit $|I| > \ord{a}$ existieren stets zwei Elemente $i$ und $i'$ mit $i = i' \bmod \ord{a}$. Da dadurch alle Bedingungen des gezeigten Lemmas erfüllt sind, folgt direkt $(\det M_{IJ})(a) = 0$. Es genügt somit, Indexmengen zu untersuchen, deren Mächtigkeit kleiner als eine gewählte (Gruppen-)Ordnung ist.

Die \Cref{equation:all-equal}, dass alle Elemente einer Indexmenge in derselben Restklasse liegen müssen, ist eine starke Einschränkung in der Wahl der Indizes. Diese Anforderung kann abgeschwächt werden, wenn zusätzliche Bedingungen an die jeweils andere Indexmenge gestellt werden.

\begin{satz} \label{satz:equal-columns-subs}
    Sei $M = \left( x^{ij} \right)_{i,j \in \mathbb{Z}}$. Seien weiter $I,J \subseteq \mathbb{Z}$ zwei Indexmengen und $a \in \field{p}[n]^\times$ mit der Ordnung $o = \ord{a} = v\cdot w$.

    Falls eine Teilmenge $I' \subseteq I$ existiert, die \Cref{equation:all-equal} erfüllt und alle $J' \subseteq J$ mit $|J'| = |I'|$ \Cref{equation:two-equal} erfüllen, folgt $(\det M_{IJ})(a) = 0$.
\end{satz}

\begin{proof}
    Zur Bestimmung der Determinante wird eine Zeilenentwicklung über die Indizes $I'$ durchgeführt. 
\begin{equation*}
    (\det M_{IJ})(a) = \left( \sum_{\substack{J'\subseteq J:\\|J'| = |I'|}} (-1)^{\epsilon_{I'J'}} \left( \det M_{I'J'} \cdot \det M_{I\backslash I'\,J\backslash J'} \right) \right)(a)
\end{equation*}
Konkret hängen die Vorfaktoren $(-1)^{\epsilon_{I'J'}}$ von den Teilmengen $I'$ und $J'$ ab. Da $I'$ und alle $J'$ bereits die \Cref{equation:all-equal,equation:two-equal} erfüllen, kann \Cref{lemma:equal-columns} auf die Terme $\det M_{I'J'}$ angewendet werden. Somit ist jeder Summand gleich $0$ und die Aussage folgt.
\end{proof}

Sei weiterhin $a = 2 \in \field{11}$ mit der Ordnung $\ord{a} = 10$. Für ein Beispiel werden die Indexmengen $I = \{0,2,3,4\}$ und $J = \{0,2,5,7\}$ betrachtet, welche \Cref{lemma:equal-columns} nicht erfüllen. Da die Teilmenge $I\backslash\{3\}$ die \Cref{equation:all-equal} für $v=2$ erfüllt, kann entlang der Zeile zum Exponenten $3$ entwickelt werden. Die für die Berechnung relevanten Submatrizen ergeben sich aus diesen Paarungen.

\begin{align*}
    I_1=\{0,2,4\} \text{ und } J_1=\{2,5,7\}, \\
    I_2=\{0,2,4\} \text{ und } J_2=\{0,5,7\}, \\
    I_3=\{0,2,4\} \text{ und } J_3=\{0,2,7\}, \\
    I_4=\{0,2,4\} \text{ und } J_4=\{0,2,5\}.
\end{align*}

Die jeweiligen Teilmengen von $J$ erfüllen \Cref{equation:two-equal} für $w=5$, sodass \Cref{satz:equal-columns-subs} angewendet werden kann und $(\det M_{IJ})(a) = 0$ folgt.
}

\begin{comment}
    Anstatt alle Teilmengen von $J$ einer bestimmten Kardinalität auf die Bedingung zu prüfen, kann auch das minimale $\gamma$ gesucht werden    
\end{comment}