\section{Singuläre Submatrizen}

\begin{lemma} \label{lemma:equal-columns}
    Seien $a \in \field{p}{n}, o = ord(a) = v\cdot w$ und $I,J \subseteq \mathbb{Z}$.
    Falls
    \begin{equation} \label{equation:all-equal}
        \forall \; i, i' \in I, i \neq i': i \equiv i' \pmod v
    \end{equation}
    und
    \begin{equation} \label{equation:two-equal}
        \exists \; j, j' \in J, j \neq j':  j \equiv j' \pmod w
    \end{equation}
    gelten, folgt
    \begin{equation*}
        (\det M_{IJ})(a) = 0
    \end{equation*}
\end{lemma}

\begin{proof}
    Wir betrachten $j,j' \in J$, welche die obige Bedingung erfüllen und zeigen, dass die dazugehörigen Spalten in $M_{IJ}$ linear abhängig sind. Seien $j = (c + bw)$, $j' = (c + b'w)$ und $I = \{(d+l_1v),\dots,(d+l_kv)\}$. Die Spalten zu $j$ und $j'$ haben die Form
    \begin{align*}
        \begin{pmatrix}
            a^{i_1j} & a^{i_1j'} \\
            \vdots & \vdots \\
            a^{i_kj} & a^{i_kj'}
        \end{pmatrix} &=
        \begin{pmatrix}
            a^{(d+l_1v)(c + bw)} & a^{(d+l_1v)(c + b'w)} \\
            \vdots & \vdots \\
            a^{(d+l_kv)(c + bw)} & a^{(d+l_kv)(c + b'w)}
        \end{pmatrix} \\
        &= \begin{pmatrix}
            a^{dc + dbw +l_1vc + l_1bvw} & a^{dc + db'w +l_1vc + l_1b'vw} \\
            \vdots & \vdots \\
            a^{dc + dbw +l_kvc + l_kbvw} & a^{dc + db'w +l_kvc + l_kb'vw}
        \end{pmatrix} \\
        &= \begin{pmatrix}
            a^{dc + dbw +l_1vc + l_1bvw} & a^{dc + db'w +l_1vc + l_1b'vw} \\
            \vdots & \vdots \\
            a^{dc + dbw +l_kvc + l_kbvw} & a^{dc + db'w +l_kvc + l_kb'vw}
        \end{pmatrix} \\
    \end{align*}
    Die konstanten Terme $a^{dc + dbw}$ und $a^{dc + db'w}$ können aus den Spalten faktorisiert werden, sodass verbleiben
    \begin{equation*}
        \begin{pmatrix}
            a^{l_1vc} {a^{vw}}^{l_1b} & a^{l_1vc} {a^{vw}}^{l_1b'} \\
            \vdots & \vdots \\
            a^{l_kvc} {a^{vw}}^{l_kb} & a^{l_kvc} {a^{vw}}^{l_kb'}
        \end{pmatrix}
    \end{equation*}
    Da $a^{vw} = a^{o} = 1$ gilt, verbleiben die gleichen Spalten.
    \begin{equation*}
        \begin{pmatrix}
            a^{l_1vc} & a^{l_1vc} \\
            \vdots & \vdots \\
            a^{l_kvc} & a^{l_kvc}
        \end{pmatrix}
    \end{equation*}
\end{proof}
{
\Crefname{equation}{Bedingung}{Bedingungen}
\begin{satz}
    Seien $a \in \field{p}{n}, o = ord(a) = v\cdot w$ und $I,J \subseteq \mathbb{Z}$.
    Falls ein $I' \subseteq I$ mit $|I'| = \gamma \geq 1$ existiert, dass \Cref{equation:all-equal} erfüllt und alle $J' \subseteq J$ mit $|J'| = \gamma$ \Cref{equation:two-equal} erfüllen, folgt
    \begin{equation*}
        (\det M_{IJ})(a) = 0
    \end{equation*}
\end{satz}

\begin{proof}
    Zu Bestimmung der Determinante führen wir eine Zeilenentwicklung über die Indizes $I'$ durch. 
\begin{equation*}
    (\det M_{IJ})(a) = \left( \sum_{J':|J'| = \gamma} (-1)^{\sum I' + \sum J'} \left( \det M_{I'J'} \cdot \det M_{I\backslash I',J\backslash J'} \right) \right)(a)
\end{equation*}
Da $I'$ und alle $J'$ die \Cref{equation:all-equal,equation:two-equal} erfüllen, kann \Cref{lemma:equal-columns} auf die Terme $\det M_{I'J'}$ angewendet werden. Somit ist jeder Summand gleich $0$ und die Aussage folgt.
\end{proof}

}