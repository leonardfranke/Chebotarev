\documentclass{article}

\usepackage{comment}
\usepackage{todonotes}
\usepackage{xparse}
\usepackage{slashbox}
\usepackage[ngerman]{babel}
\usepackage{amsthm}
\usepackage{amsfonts}
\usepackage{amsmath}
\usepackage{array}
\usepackage{mathtools}
\usepackage[unicode=true,
 bookmarks=true,bookmarksnumbered=false,bookmarksopen=false,
 breaklinks=false,pdfborder={0 0 1},backref=false,colorlinks=false]
 {hyperref}
\usepackage[noabbrev,nameinlink, ngerman]{cleveref}

\newtheorem{satz}{Satz}
\newtheorem{lemma}[satz]{Lemma}
\newtheorem{korollar}[satz]{Korollar}
\theoremstyle{definition}
\newtheorem{definition}{Definition}

\newcolumntype{C}{>{$}c<{$}}

\renewcommand{\det}{\mathrm{det}\;}
\NewDocumentCommand{\ord}{m o}{%
 \mathrm{ord}_{\IfValueT{#2}{{#2}}}(#1)%
}
\newcommand{\charac}{\mathrm{char}}
\newcommand{\Tr}{\mathrm{Tr}}
\newcommand{\Modulo}{\mathbin\mathrm{mod}}
\NewDocumentCommand{\field}{m o}{%
  \mathbb{F}_{#1\IfValueT{#2}{^{#2}}}%
}

\begin{document}

\section{Einleitung}

Diese Bachelorarbeit beschäftigt sich mit den Minoren verallgemeinerter Vandermondematrizen. Die betrachteten Matrizen besitzen die Form \begin{equation*}
    \left( x^{ij} \right)_{0\leq i,j \leq k-1} = \begin{pmatrix}
        1     & 1    & 1    & 1    &\cdots& 1 \\
        1     & x^1  & x^2  & x^3  &\cdots& x^{k-1} \\
        1     & x^2  & x^4  & x^6  &\cdots& x^{2k-2} \\
        1     & x^3  & x^6  & x^9  &\cdots& x^{3k-3} \\
        \vdots&\vdots&\vdots&\vdots&\ddots&\vdots \\
        1     &x^{k-1}&x^{2k-2}&x^{3k-3}&\cdots&x^{(k-1)(k-1)}
    \end{pmatrix},
\end{equation*}
wobei für $x$ stets Elemente eines endlichen Körpers $\field{p}[n]$ eingesetzt werden. Für gegebenes $k$ sind all die Elemente $a$ gesucht, sodass jede Untermatix von $\left( a^{ij} \right)_{0\leq i,j \leq k-1}$ invertierbar ist. 
Diese Untermatizen ergeben sich dann aus der Wahl zweier gleich mächtiger Indexmengen $I,J \subseteq \{0, \dots, k-1\}$. Da für die relevanten Körperelemente auch negative Exponenten zulässig sind, wird im Hauptteil dieser Arbeit die unendliche Matrix $M = \left( x^{ij} \right)_{i,j \in \mathbb{Z}}$ betrachtet. Entsprechend werden diese Submatrizen mit endlichen Teilmengen $I,J \subseteq \mathbb{Z}$ indiziert.

\begin{comment}
    Ein Eintrag $x^{ij}$ kann als Monom aus dem Polynomring $\field{p}[n][X]$ verstanden werden. Aufgrund der Abgeschlossenheit der Polynomringe bilden die Minoren $\det M_{IJ}$ ebenfalls Polynome.
\end{comment}
    
Für eine ähnliche Konstruktion über den komplexen Zahlen $\mathbb{C}$ gibt es bereits Resultate bezüglich der Minoren. Der Mathematiker Chebotarev zeigte 1926, dass für primitive $p$-te Einheitswurzel (zum Beispiel $a=e^{2\pi i/p}$) alle Untermatizen von $\left( a^{ij} \right)_{0\leq i,j \leq p-1}$ invertierbar sind, falls $p$ Primzahl ist. \cite{CheboProof}
Analog existiert eine Aussage über $p$-te primitive Einheitswurzeln endlicher Körper. Diese gilt jedoch nur für spezielle Primzahlpotenzen $q^{p-1}$, bei denen $q$ sehr groß sein muss. \cite{CheboFiniteFields}


\section{Grundlagen} \label{sec:grundlagen}

In diesem Kapitel werden die für den Hauptteil benötigten Definitionen und Sätze über endliche Körper geklärt. Die Konstruktion dieser Strukturen geschieht über Restklassenringe der ganzen Zahlen und über spezielle Polynomringe. Abschließend werden Untergruppen dieser Körper und ihre zyklischen Gruppen, Erzeuger und Ordnungen untersucht.

Die ersten grundlegenden endlichen Körper lassen sich über Restklassenringe der ganzen Zahlen konstruieren. Diese werden formal über Restabbildungen definiert.

\begin{satz}[{\cite[S. 6]{Kurzweil}}]
    Sei $n \in \mathbb{N}$ und $\mathbb{Z}_n \coloneqq \{0,\dots,n-1\}$. Dann existieren für jedes $a \in \mathbb{Z}$ eindeutig bestimmte Zahlen $f \in \mathbb{Z}$ und $r \in \mathbb{Z}_n$ mit $a = n \cdot f + r$.
\end{satz}

Sei zum Beispiel $n = 4$. Dann gelten folgende Darstellungen:

\begin{align*}
    -2 &= 4 \cdot -1 + 2 \\
    -1 &= 4 \cdot -1 + 3 \\
    0 &= 4 \cdot 0 + 0 \\
    1 &= 4 \cdot 0 + 1 \\
    2 &= 4 \cdot 0 + 2 \\
    3 &= 4 \cdot 0 + 3 \\
    4 &= 4 \cdot 1 + 0 \\
    5 &= 4 \cdot 1 + 1
\end{align*}

Die Zahl $r$ wird als der \emph{Rest} von $a$ modulo $n$ bezeichnet. Im Folgenden wird dies mit $r = a \bmod n$ abgekürzt. Die Abbildung $\varrho_n : \mathbb{Z} \rightarrow \mathbb{Z}_n$, welche jeder ganzen Zahl ihren Rest modulo $n$ zuordnet, ist die \emph{Restabbildung}. Mit ihr kann aus der Menge $\mathbb{Z}_n$ ein Ring konstruiert werden.

\begin{satz}[{\cite[S. 9]{Kurzweil}}]
    Sei $n \in \mathbb{N}$. Die Menge $\mathbb{Z}_n$ zusammen mit der Multiplikation $a \otimes b = \varrho_n(a \cdot b)$ und der Addition $a \oplus b = \varrho_n(a + b)$ bilden einen Ring. Dieser wird als Restklassenring modulo $n$ bezeichnet.
\end{satz}

Die Operationen im Restklassenring nutzen die von den ganzen Zahlen bekannte Addition und Multiplikation mit anschließender Modulodivision. \Cref{table:tableZ4,table:tableZ5} zeigen die Additions- und Multiplikationstafeln für $n=4$ und $n=5$. In diesen können einige der Ringaxiome verifiziert werden. Zum Beispiel existieren jeweils die neutralen Elemente $0$ und $1$. Außerdem besitzt jedes Element ein additiv Inverses. Bezüglich der Multiplikation existieren jedoch nicht immer inverse Elemente. Dies zeigt, dass $\mathbb{Z}_n$ im Allgemeinen keinen Körper bildet. Der folgende Satz gibt an, für welche natürlichen Zahlen das dennoch der Fall ist.

\begin{table}[]
    \centering
    \begin{tabular}{|C|C|C|C|C|}
    \hline
    + & 0  & 1 & 2 & 3 \\ \hline
    0 & 0  & 1 & 2 & 3 \\ \hline
    1 & 1  & 2 & 3 & 0 \\ \hline
    2 & 2  & 3 & 0 & 1 \\ \hline
    3 & 3  & 0 & 1 & 2 \\ \hline
    \end{tabular}
    \quad
    \begin{tabular}{|C|C|C|C|C|}
        \hline
    \cdot & 0  & 1 & 2 & 3 \\ \hline
        0 & 0  & 0 & 0 & 0 \\ \hline
        1 & 0  & 1 & 2 & 3 \\ \hline
        2 & 0  & 2 & 0 & 2 \\ \hline
        3 & 0  & 3 & 2 & 1 \\ \hline
        \end{tabular}
    \caption{Additions- und Multiplikationstafel für den Restklassenring $\mathbb{Z}_4$ \cite[S. 10]{Kurzweil}} \label{table:tableZ4}
\end{table}

\begin{table}[]
    \centering
    \begin{tabular}{|C|C|C|C|C|C|}
    \hline
    + & 0  & 1 & 2 & 3 & 4 \\ \hline
    0 & 0  & 1 & 2 & 3 & 4 \\ \hline
    1 & 1  & 2 & 3 & 4 & 0 \\ \hline
    2 & 2  & 3 & 4 & 0 & 1 \\ \hline
    3 & 3  & 4 & 0 & 1 & 2 \\ \hline
    4 & 4  & 0 & 1 & 2 & 3 \\ \hline
    \end{tabular}
    \quad
    \begin{tabular}{|C|C|C|C|C|C|}
        \hline
    \cdot & 0  & 1 & 2 & 3 & 4 \\ \hline
        0 & 0  & 0 & 0 & 0 & 0 \\ \hline
        1 & 0  & 1 & 2 & 3 & 4 \\ \hline
        2 & 0  & 2 & 4 & 1 & 3 \\ \hline
        3 & 0  & 3 & 1 & 4 & 2 \\ \hline
        4 & 0  & 4 & 2 & 2 & 1 \\ \hline
        \end{tabular}
    \caption{Additions- und Multiplikationstafel für den Restklassenring $\mathbb{Z}_5$ \cite[S. 10]{Kurzweil}} \label{table:tableZ5}
\end{table}

\begin{satz}[{\cite[S. 12]{Kurzweil}}]
    $\mathbb{Z}_n$ ist genau dann ein Körper, wenn $n$ Primzahl ist.
\end{satz}

In diesem Fall wird der Körper als $\field{p}$ bezeichnet. Von den beiden Beispielen $\mathbb{Z}_4$ und $\mathbb{Z}_5$ ist somit nur der Zweite ein Körper, da $5$ Primzahl ist.
Die wie oben konstruierten Körper besitzen alle die Mächtigkeit einer Primzahl. Darauf aufbauend können mithilfe von Polynomen weitere endliche Körper unterschiedlicher Mächtigkeit gebildet werden. 
Analog zu den ganzen Zahlen wird dafür zunächst ein Ring erzeugt.

\begin{satz}[{\cite[S. 22]{Kurzweil}}]
    Sei $\mathbb{F}$ Körper. Die Menge $\mathbb{F} {[X]}$ der Polynome in der Variablen $X$ mit Koeffizienten in $\mathbb{F}$ zusammen mit der von Polynomen bekannten Addition und Multiplikation bilden einen Ring.
\end{satz}

Auch für diese Menge lässt sich eine Modulodivision mit Restabbildung definieren. 

\begin{satz}[{\cite[S. 26]{Kurzweil}}]
    Sei $n\in \mathbb{N}$ und $\field{} {[X]}_n \coloneqq \{A \in \mathbb{F} {[X]} \mid \mathrm{grad}(A) < n\}$ die Menge der Polynome mit Koeffizienten in $\field{}$ vom Grad kleiner als $n$. Sei weiter $N \in \mathbb{F} {[X]}$ ein Polynom mit $n = \mathrm{grad}(N)$. Dann existieren für jedes $A \in \mathbb{F} {[X]}$ eindeutig bestimmte Polynome $F \in \mathbb{F} {[X]}$ und $R \in \field{} {[X]}_n$ mit $A = N \cdot F + R$.
\end{satz}

%Vllt Beispiel für Polynomdivision

Das Polynom $R$ wird ebenfalls als der \emph{Rest} von $A$ modulo $N$ bezeichnet, was wiederum mit $R = A \bmod N$ abgekürzt wird. Mithilfe der \emph{Restabbildung} $\varrho_N : \mathbb{F} {[X]} \rightarrow \field{} {[X]}_n$, welche jedem Polynom seinen Rest modulo $N$ zuordnet, lässt sich wieder ein Ring erzeugen.

\begin{satz}[{\cite[S. 28]{Kurzweil}}]
    Sei $N \in \mathbb{F} {[X]}$ mit $n = \mathrm{grad}(N)$. Die Menge $\field{} {[X]}_n$ zusammen mit der Multiplikation $A \otimes B = \varrho_N(A \cdot B)$ und der Addition $A \oplus B = \varrho_N(A + B)$ bilden einen Ring. Dieser wird als Polynomring modulo $N$ bezeichnet und mit $\field{N} {[X]}$ abgekürzt.
\end{satz}

Falls $N \in \mathbb{F} {[X]}$ Grad $n$ hat und der Körper $\mathbb{F}$ endlich mit $p$ Elementen ist, besitzt $\field{N} {[X]}$ die Mächtigkeit $p^n$.
Sei zum Beispiel $\mathbb{F} = \field{2}$ und $N$ ein Polynom vom Grad 2. Dann umfasst $\field{N} {[X]}$ alle 4 Polynome vom Grad kleiner als 2. \Cref{table:tableF41,table:tableF42} zeigen die Additions- und Multiplikationstafeln für $N=x^2 + x$ und $N= x^2 + x + 1$.

\begin{table}[]
    \centering
    \begin{tabular}{|C|C|C|C|C|}
    \hline
    +   & 0     & 1     & x     & x + 1 \\ \hline
    0   & 0     & 1     & x     & x+1 \\ \hline
    1   & 1     & 0     & x+1   & x \\ \hline
    x   & x     & x+1   & 0     & 1 \\ \hline
    x+1 & x+1   & x     & 1     & 0 \\ \hline
    \end{tabular}
    \quad
    \begin{tabular}{|C|C|C|C|C|}
        \hline
    \cdot   & 0 & 1     & x     & x + 1 \\ \hline
        0   & 0 & 0     & 0     & 0 \\ \hline
        1   & 0 & 1     & x     & x+1 \\ \hline
        x   & 0 & x     & x     & 0 \\ \hline
        x+1 & 0 & x+1   & 0     & x+1 \\ \hline
        \end{tabular}
    \caption{Additions- und Multiplikationstafel für den Polynomring $\field{N} {[X]}$ für ${N=x^2 + x}$ und $\field{} = \field{2}$} \label{table:tableF41}
\end{table}

\begin{table}[]
    \centering
    \begin{tabular}{|C|C|C|C|C|}
    \hline
    +   & 0     & 1     & x     & x + 1 \\ \hline
    0   & 0     & 1     & x     & x+1 \\ \hline
    1   & 1     & 0     & x+1   & x \\ \hline
    x   & x     & x+1   & 0     & 1 \\ \hline
    x+1 & x+1   & x     & 1     & 0 \\ \hline
    \end{tabular}
    \begin{tabular}{|C|C|C|C|C|}
        \hline
    \cdot   & 0 & 1     & x     & x + 1 \\ \hline
        0   & 0 & 0     & 0     & 0 \\ \hline
        1   & 0 & 1     & x     & x+1 \\ \hline
        x   & 0 & x     & x + 1& 1 \\ \hline
        x+1 & 0 & x+1   & 1     & x \\ \hline
        \end{tabular}
    \caption{Additions- und Multiplikationstafel für den Polynomring $\field{N} {[X]}$ für ${N=x^2 + x + 1}$ und $\field{} = \field{2}$} \label{table:tableF42}
\end{table}

Auch für diese Ringe können einige Ringaxiome verifiziert werden. Die konstanten Polynome $0$ und $1$ sind die neutralen Elemente. Offensichtlich existieren nicht immer multiplikativ inverse Elemente, sodass auch $\field{N} {[X]}$ im Allgemeinen kein Körper ist. Die Eigenschaft, welche das Polynom $N$ erfüllen muss, damit $\field{N} {[X]}$ ein Körper ist, wird im folgenden Abschnitt betrachtet.

Analog zur Teilbarkeit von Primzahlen wird dazu die Teilbarkeit von Polynomen untersucht. Seien $A,B \in \mathbb{F} {[X]}$. $A$ heißt \emph{Teiler} von $B$, falls ein Polynom $F \in \mathbb{F} {[X]}$ mit $B = A \cdot F$ existiert. A heißt \emph{trivialer Teiler}, wenn $\grad{A} = 0$ oder $\grad{F} = 0$ gilt. Eine Zerlegung mit einem trivialen Teiler existiert für jedes Polynom. Sei zum Beispiel $\lambda \in \field{}$  der Leitkoeffizient von $A$. Dann ist $\lambda^{-1}A = F$ das zu $A$ normierte Polynom und es gilt $A = \lambda\cdot F$.

Sei im Folgenden $\field{} = \field{3}$ und $A=2x^2 + 1 \in \mathbb{F} {[X]}$. Dann lässt sich das Polynom $A$ darstellen als $2(x+2)(x+1) = 2x^2 + 6x + 4 = 2x^2 + 1$. Für ein anderes Polynom $B=x^2 + 1$ existieren jedoch nur triviale Zerlegungen. Die Teilbarkeit ist direkt vom gewählten Körper abhängig. Über $\field{2}$ lässt sich dasselbe Polynom $B=x^2 + 1$ wiederum als $(x+1)^2$ darstellen.

Ein Polynom aus $\field{} {[X]}$ heißt \emph{irreduzibel} über $\field{}$, falls es in $\field{} {[X]}$ nur triviale Teiler besitzt. Die Irreduzibilität ist genau die Eigenschaft, welche ein Polynom $N$ erfüllen muss, damit $\field{N} {[X]}$ ein Körper ist. 

\begin{satz}[{\cite[S. 46]{Kurzweil}}]
    $\field{N} {[X]}$ ist genau dann ein Körper, wenn $N \in \field{}{[X]}$ irreduzibel über $\field{}$ ist.
\end{satz}

Somit ist geklärt, weshalb die Beispiele von \Cref{table:tableF41,table:tableF42} einen Körper oder nur einen Ring bilden. Für $N=x^2 + x$ und $\field{} = \field{2}$ existiert die Zerlegung $x(x+1)$, aus der sich zusätzlich die Polynome $x$ und $x+1$ als Nullteiler ergeben. Für $N=x^2 + x + 1$ existiert keine solche Zerlegung, sodass für dieses Polynom ein Körper entsteht.

Mithilfe von irreduziblen Polynomen lassen sich nun endliche Körper der Mächtigkeit $p^n$ für eine Primzahl $p$ konstruieren. Zu klären bleibt die Existenz beliebiger endlicher Körper mit genannter Mächtigkeit und die Eindeutigkeit dieser für verschiedene Polynome gleichen Grades. Dazu geben die folgenden beiden Sätze Aufschluss.

\begin{satz}[Existenzsatz {\cite[S. 141]{Kurzweil}}]
    Sei $p^n$ eine Primzahlpotenz. Dann existiert ein endlicher Körper mit $p^n$ Elementen.
\end{satz}

\begin{satz}[Eindeutigkeitssatz {\cite[S. 142]{Kurzweil}}]
    Sei $p^n$ eine Primzahlpotenz. Bis auf Isomorphie existiert nur ein endlicher Körper mit $p^n$ Elementen.
\end{satz}

Aus dem Eindeutigkeitssatz folgt, dass sich zwei endliche Körper gleicher Mächtigkeit strukturell äquivalent verhalten. Demnach ist das Angeben des irreduziblen Polynoms nur bei konkreten Rechnungen relevant. Deshalb wird im Folgenden $\field{p}[n]$ für den Körper mit $p^n$ Elementen geschrieben, ohne das irreduzible Polynom zu nennen. Der Existenzsatz gibt zudem an, dass auch jeder $\field{p}[n]$ für eine Primzahl $p$ und beliebiges $n \in \mathbb{N}$ existiert. Im letzten Abschnitt dieses Kapitels wird auf für diese Arbeit notwendige Eigenschaften endlicher Körper eingegangen.

Da im Hauptteil dieser Arbeit Matrizen der Form $\left( x^{ij} \right)_{i \in I,j \in J}$ für zwei endliche Teilmengen $I,J \subseteq \mathbb{Z}$ betrachtet werden, sind Potenzen von Körperelementen interessant. Die Menge $\field{}^\times \coloneqq \field{} \setminus \{0\}$ wird als \emph{multiplikative Gruppe} von $\field{}$ bezeichnet und erfüllt dahingehend nützliche Eigenschaften. Sie bildet eine Gruppe bezüglich der Multiplikation, da sie nullteilerfrei ist, sowie Inverse und ein neutrales Element enthält.

\begin{definition}[{\cite[S. 83]{Kurzweil}}]
    Sei $G$ Gruppe. $G$ heißt zyklische Gruppe, falls ein $a \in G$ existiert, sodass $G$ nur aus Potenzen von $a$ besteht, also $G = \langle a \rangle \coloneqq \{ a^i \mid i \in \mathbb{Z}\}$ gilt. Das Element $a$ heißt dann \emph{Erzeuger} von $G$.
\end{definition}

Im Folgenden wird die Gruppe $\field{7}^\times$ betrachtet. Die Mengen $\langle a \rangle$ sehen wie folgt aus.

\begin{align*}
    \langle 1 \rangle &= \{ 1\} \\
    \langle 2 \rangle &= \{ 1,2,4\} \\
    \langle 3 \rangle &= \{ 1,2,3,4,5,6\} \\
    \langle 4 \rangle &= \{ 1,2,4\} \\
    \langle 5 \rangle &= \{ 1,2,3,4,5,6\} \\
    \langle 6 \rangle &= \{ 1,6\} \\
\end{align*}

Da $\field{7}^\times = \langle 3 \rangle = \langle 5 \rangle$ gilt, ist $\field{7}^\times$ eine zyklische Gruppe mit Erzeugern $3$ und $5$. Der folgende Satz klärt den Zusammenhang zwischen multiplikativen Gruppen endlicher Körper und zyklischen Gruppen.

\begin{satz}[{\cite[S. 83]{Kurzweil}}]
    Sei $\field{}$ Körper. Dann ist $\field{}^\times$ und jede weitere endliche Untergruppe von $\field{}^\times$ eine zyklische Gruppe.
\end{satz}

Demnach besitzt die multiplikative Gruppe $\field{}^\times$ mindestens einen Erzeuger. Die anderen Mengen $\langle a \rangle$ im obigen Beispiel bilden zwar nicht die gesamte multiplikative Gruppe, jedoch Untergruppen, die per Konstruktion zyklisch mit Erzeuger $a$ sind. Da das neutrale Element in jeder Untergruppe enthalten ist, existiert zu jedem Erzeuger $a$ ein Exponent $i \in \mathbb{Z}$ mit $a^i = 1$. Das kleinste $i \in \mathbb{N}_+$ mit $a^i = 1$ wird als die \emph{Ordnung} von $a$ bezeichnet und mit $\ord{a}$ abgekürzt. Für die Elemente aus $\field{7}^\times$ sind dies konkret:

\begin{align*}
    \ord{1} &= 1 \\
    \ord{2} &= 3 \\
    \ord{3} &= 6 \\
    \ord{4} &= 3 \\
    \ord{5} &= 6 \\
    \ord{6} &= 2
\end{align*}

Aus der Existenz der Ordnungen ergibt sich direkt folgender nützlicher Satz.

\begin{satz}[{\cite[S. 86]{Kurzweil}}] \label{satz:cyclicity}
    Sei $G$ zyklische Gruppe und $a \in G$ ein Element mit Ordnung $o = \ord{a}$. Dann gilt für $i \in \mathbb{Z}$:
    \begin{equation*}
        a^i = a^{i+ko}, \quad k \in \mathbb{Z}
    \end{equation*}
\end{satz}

Die Potenzen der Gruppenelemente wiederholen sich zyklisch nach $\ord{a}$ Iterationen. Folglich stimmt die Mächtigkeit der Untergruppe $\langle a \rangle$ mit der Ordnung von $a$ überein. Ist $G$ eine zyklische Gruppe, so wird die Mächtigkeit $|G|$ auch als die Ordnung von $G$ bezeichnet.
Der folgende Satz zeigt, wie die Ordnungen von Unter- und Obergruppe zusammenhängen.

\begin{satz}[{\cite[S. 87]{Kurzweil}}]
    Sei $G$ zyklische Gruppe und $a \in G$ ein Element mit Ordnung $o = \ord{a}$. Dann ist $o$ Teiler von $|G|$.
\end{satz}

Zusätzlich folgt, dass auch die Ordnungen aller Untergruppen Teiler von $|G|$ sind. Somit ist klar, dass die Ordnungen aller Elemente vom $\field{7}^\times$ in der Menge $\{1,2,3,6\}$ liegen müssen. Für beliebige endliche Körper $\field{p}[n]$ sind die Ordnungen aller Elemente der multiplikativen Gruppe Teiler von $p^n-1$.

Damit sind die notwendigen Grundlagen über endliche Körper geklärt. 

\section{Definitionen}

\begin{definition}
    Seien $I = \{i_1,\dots,i_k\},J = \{i_1,\dots,i_k\} \subseteq \mathbb{Z}$ Dann ist $M_{IJ} := (x^{ij})_{i \in I, j \in J}$. 
\end{definition}

\section{Lineare Indexabbildungen}

Die Submatrizen einer Matrix $M$ werden über zwei gleich mächtige Indexmengen $I,J$ indiziert. Diese Submatrix wird als $M_{IJ}$ bezeichnet. 
In diesem Kapitel werden lineare Abbildungen auf die Indexmengen $I,J$ angewendet und der Zusammengang zum Minor $\det M_{IJ}$ untersucht.

\todo{Sortierung von I irrelevant, da Determinante alternierend ist}

Die Matrix $M$ ist weiterhin von der Form $\left( x^{ij} \right)_{0\leq i,j \leq k-1}$, wobei für $x$ ausschließlich Elemente eines endlichen Körpers $\field{p}[n]$ substituiert werden. Ein Eintrag $x^{ij}$ kann somit als Monom aus dem Polynomring $\field{p}[n][X]$ verstanden werden. Aufgrund der Abgeschlossenheit des Polynomrings ist der Minor $\det M_{IJ}$ ebenfalls ein Polynom. Da $M$ und ihre Submatrizen wiederum als Submatrizen der unendlich größen Matrix $\left( x^{ij} \right)_{i,j \in \mathbb{Z}}$ aufgefasst werden können, sei im Folgenden $M$ stets eben diese Matrix. Dies vereinfacht einige Aussagen und Erläuterungen in diesem Kapitel und ermöglicht allgemeinere Indexmengen $I,J \subseteq \mathbb{Z}$ zu betrachten.

Für zum Beispiel $I,J = \{-2, \dots, 4\}$ sieht $M_{IJ}$ folgendermaßen aus.
\begin{equation*}
    M_{IJ} = \begin{pmatrix}
        x^4     & x^{2} & 1    & x^{-2} & x^{-4}& x^{-6}& x^{-8}\\
        x^{2}   & x^{1} & 1    & x^{-1} & x^{-2}& x^{-3}& x^{-4}\\
        1       & 1     & 1    & 1      & 1     & 1     & 1     \\
        x^{-2}  & x^{-1}& 1    & x^{1}  & x^{2} & x^{3} & x^{4} \\
        x^{-4}  & x^{-2}& 1    & x^{2}  & x^{4} & x^{6} & x^{8} \\
        x^{-6}  & x^{-3}& 1    & x^{3}  & x^{6} & x^{9} & x^{12} \\
        x^{-8}  & x^{-4}& 1    & x^{4}  & x^{8} & x^{12} & x^{16} \\
    \end{pmatrix}
\end{equation*}

Falls der $\field{3}$ als endlicher Körper gewählt wird, besitzt $M_{IJ}$ aufgrund der zyklischen Eigenschaften des Körpers wiederum folgende Form.

\begin{equation*}
    M_{IJ} = \begin{pmatrix}
        x^1     & x^{2} & 1     & x^{1} & x^{2} & 1 & x^{1} \\
        x^{2}   & x^{1} & 1     & x^{2} & x^{1} & 1 & x^{2} \\
        1       & 1     & 1     & 1     & 1     & 1 & 1     \\
        x^{1}   & x^{2} & 1     & x^{1} & x^{2} & 1 & x^{1} \\
        x^{2}   & x^{1} & 1     & x^{2} & x^{1} & 1 & x^{2} \\
        1       & 1     & 1     & 1     & 1     & 1 & 1     \\
        x^1     & x^2   & 1     & x^1   & x^2   & 1 & x^1   \\
    \end{pmatrix}
\end{equation*}

Durch das Reduzieren der Exponenten um die entsprechende Gruppenordnung, erhält die unendlich große Matrix $M$ die Form einer Blockmatrix, die nur aus den Blöcken

\begin{equation*}
    \begin{pmatrix}
        1     & 1     & 1     \\
        1     & x^{1} & x^{2} \\
        1     & x^{2} & x^{1} \\
    \end{pmatrix}
\end{equation*}

besteht. Für allgemeine $\field{p}[n]$ haben die Blöcke diese Form.

\begin{equation*}
    \begin{pmatrix}
        1     & 1       & 1         &\cdots & 1 \\
        1     & a^1     & a^2       &\cdots & a^{p^n-1} \\
        1     & a^2     & a^4       &\cdots & a^{2p^n-2} \\
        \vdots&\vdots   &\vdots     &\ddots &\vdots \\
        1     &a^{p^n-1}&a^{2p^n-2} &\cdots &a^{(p^n-1)(p^n-1)}
    \end{pmatrix}
\end{equation*}

Da bei symmetrische Matrizen die $i$-te Zeile der $i$-ten Spalte entspricht, gilt $M_{IJ} = M_{JI}^T$ für beliebige Indexmengen $I,J$. Für gleich mächtige Indexmengen folgt direkt die erste nützliche Eigenschaft $\det M_{IJ} = \det M_{JI}$, welche in diesem Kapitel häufig genutzt wird.

Zuerst wird das Addieren einer Konstanten auf die Indexmengen untersucht, das dem Verschieben der Submatrix entspricht. Für eine Menge $I \subseteq \mathbb{Z}$ und ein $m \in \mathbb{Z}$ sei $I+m := \{i+m:i\in I\}$. Die Minoren von $M_{IJ}$ und $M_{I+m,J+n}$ hängen wie folgt zusammen.

\begin{satz} \label{satz:translation}
    Sei $M = \left( x^{ij} \right)_{i,j \in \mathbb{Z}}$. Seien weiter $I,J \subseteq \mathbb{Z}$ mit $|I|=|J|=k$ und $m,n \in \mathbb{Z}$. Dann gilt
    \begin{equation*}
        \det{} M_{I+m,J+n} = x^{mnk} x^{m(j_1 +\cdots + j_k)} x^{n(i_1+\cdots +i_k)} \det{} M_{IJ}
    \end{equation*}
\end{satz}

\begin{proof}
    Die Einträge der Matrix $M_{I+m,J+n}$ haben die Form 
    \begin{equation*}
        x^{(i+m)(j+n)} = x^{ij} \cdot x^{mj} \cdot x^{ni} \cdot x^{mn} 
    \end{equation*} für $i\in I, j\in J$. Mithilfe der Multilinearität der Determinante können für feste Zeilen- und Spaltenindizes konstante Terme herausgezogen werden.
    \begin{align*}
        \det M_{I+m,J+n}    &= \det \left(x^{ij} \cdot x^{mj} \cdot x^{ni} \cdot x^{mn} \right)_{i\in I,j\in J} \\ 
                            &= \left( \prod_{i \in I} x^{ni} \cdot x^{mn}\right) \det \left(x^{ij} \cdot x^{mj} \right)_{i\in I,j\in J} \\ 
                            &= \left( \prod_{i \in I} x^{ni} \cdot x^{mn}\right) \left( \prod_{j \in J} x^{mj}\right) \det \left(x^{ij}\right)_{i\in I,j\in J} \\ 
                            &= x^{kmn} \left( \prod_{i \in I} x^{ni}\right) \left( \prod_{j \in J} x^{mj}\right) \det M_{IJ} \\ 
                            &= x^{mnk} x^{n(i_1 +\cdots +i_k)} x^{m(j_1+\cdots +j_k)} \det M_{IJ}.
    \end{align*}
\end{proof}

Die beiden Minoren besitzen somit die gleichen Nullstellen. Ist andersherum betrachtet eine Nullstelle $a$ von $\det M_{I,J}$ bekannt, ist $a$ ebenfalls Nullstelle von $\det M_{I+m,J+n}$ für alle $m,n \in \mathbb{Z}$. Der Satz zeigt zudem, dass die Nullstellen ausschließlich von den Abständen der Indizes in $I$ und $J$ abhängen und nicht von einer Verschiebung innerhalb von $M$. Somit kann es sich anbieten nur normalisierte Indexmengen zu betrachten, für die $\min(I) = \min(J) = 0$ gelten.

Als nächstes wird die Multiplikation der Indexmengen mit einer Konstanten untersucht. Für eine Menge $I \subseteq \mathbb{Z}$ und ein $a \in \mathbb{Z}$ sei $aI := \{ai:i\in I\}$. Die Minoren von $M_{IJ}$ und $M_{aI,bJ}$ hängen wie folgt zusammen.

\begin{satz} \label{satz:skalierung}
    Sei $M = \left( x^{ij} \right)_{i,j \in \mathbb{Z}}$. Seien weiter $I,J \subseteq \mathbb{Z}$ mit $|I|=|J|=k$ und $a,b \in \mathbb{Z}\setminus\{0\}$. Dann gilt
    \begin{equation*}
        \left( \det{} M_{aI,bJ} \right) (x) = \left( \det{} M_{IJ} \right) (x^{ab})
    \end{equation*}
\end{satz}

\begin{proof}    
    \begin{align*}
        M_{aI,bJ}   &= \left(x^{ai\cdot bj}\right)_{i\in I,j\in J} \\
                    &= \left((x^{ab})^{ij}\right)_{i\in I,j\in J}
    \end{align*}

    Die Matrix $M_{aI,bJ}$ weißt somit die selbe Struktur auf wie $M_{I,J}$. Anstelle der Variable $x$ besitzt $M_{aI,bJ}$ Variablen der Form $x^{ab}$ in ihren Einträgen. Durch Substitution von $x$ durch $x^{ab}$ lässt sich der Minor $\det M_{aI,bJ}$ aus dem Minor $\det M_{I,J}$ gewinnen.
\end{proof}

Aus dem gezeigten Satz folgt direkt, dass $\det{} M_{I,J}$ und $\det{} M_{-I,-J}$ die selben Nullstellen besitzen. Ist andersherum betrachtet eine Nullstelle $a$ von $\det M_{I,J}$ bekannt, ist $a$ ebenfalls Nullstelle von $\det M_{-I,-J}$ und $a^{-1}$ Nullstelle von $\det M_{-I,J}$ und $\det M_{I,-J}$.

Die \Cref{satz:translation,satz:skalierung} lassen sich im folgenden Korollar zusammenfügen.

\begin{korollar}
    Sei $M = \left( x^{ij} \right)_{i,j \in \mathbb{Z}}$. Seien weiter $I,J \subseteq \mathbb{Z}$ mit $|I|=|J|=k$ und $m,n,a,b \in \mathbb{Z}$. Dann gilt
    \begin{equation*}
        \left( \det{} M_{aI+m,bJ+n} \right) (x) = x^{mnk} x^{m(j_1 +\cdots + j_k)} x^{n(i_1+\cdots +i_k)} \left( \det{} M_{IJ} \right) (x^{ab})
    \end{equation*}
\end{korollar}

\begin{proof}
    \begin{align*}
        \left( \det{} M_{aI+m,bJ+n} \right) (x)     &= x^{mnk} x^{m(bj_1 +\cdots + bj_k)} x^{n(ai_1+\cdots +ai_k)} \left( \det{} M_{aI,bJ} \right) (x) \\
                                                    &= x^{mnk} x^{mb(j_1 +\cdots + j_k)} x^{na(i_1+\cdots +i_k)} \left( \det{} M_{aI,bJ} \right) (x) \\
                                                    &= x^{mnk} x^{mb(j_1 +\cdots + j_k)} x^{na(i_1+\cdots +i_k)} \left( \det{} M_{IJ} \right) (x^{ab}) \\
    \end{align*}
\end{proof}

Die Zusammenhänge zwischen transformierten Indexmengen können über Graphen dargestellt werden. Für festes $a \in \field{p}[n]$ sei $G_a$ der Graph mit Knotenmenge $V=\{I \in\mathcal{P}(\mathbb{Z}) \mid |I| \in \mathbb{N}_+\}$ und Kantenmenge $E = \{(I,J)\mid (\det M_{IJ})(a) = 0\}$. Der Graph $G_a$ enthält somit genau dann die Kante $(I,J)$, falls die Submatrix $M_{IJ}(a)$ nicht invertierbar ist. 

Aus \Cref{satz:translation} folgt, dass eine Kante $(I,J)$ alle weiteren Kanten $(I+m,J)$ für $m\in \mathbb{Z}$ induziert. Alle Knoten der Form $I+m$ besitzen somit in $G_a$ die selben adjazenten Knoten. Dieses Verhalten ist in \Cref{fig:subgraphs} dargestellt. Falls die Indexmengen $I$ und $J$ unterschiedlich sind, ergibt sich der vollständig bipartite Teilgraph aus \Cref{fig:biclique-graph}. Falls jedoch $I$ und $J$ gleich sind, entsteht der in \Cref{fig:complete-graph} dargestellte vollständige Teilgraph, in dem alle Knoten der Form $I+m$ miteinander verbunden sind. 

\begin{figure}[]
    \centering
    \begin{subfigure}[b]{0.47\textwidth}
        \centering
        \begin{tikzpicture}[node distance={5mm}, main/.style = {draw, circle, minimum size=1.2cm}]

            \node (D0) at (-0.5,-1.5) {\large $\cdots$};
            \node (D1) at (5,-1.5) {\large $\cdots$};

            % Nodes
            \node[main] (I0) at (0,0) {\small $I-1$};
            \node[main] (I1) at (1.5,0) {\small $I$};
            \node[main] (I2) at (3,0) {\small $I+1$};
            \node[main] (I3) at (4.5,0) {\small $I+2$};
            \node[main] (J0) at (0,-3) {\small $J-1$};
            \node[main] (J1) at (1.5,-3) {\small $J$};
            \node[main] (J2) at (3,-3) {\small $J+1$};
            \node[main] (J3) at (4.5,-3) {\small $J+2$};
    
            % Edges
            \draw (I0) to (J0);
            \draw (I0) to (J1);
            \draw (I0) to (J2);
            \draw (I0) to (J3);
            \draw (I1) to (J0);
            \draw (I1) to (J1);
            \draw (I1) to (J2);
            \draw (I1) to (J3);
            \draw (I2) to (J0);
            \draw (I2) to (J1);
            \draw (I2) to (J2);
            \draw (I2) to (J3);
            \draw (I3) to (J0);
            \draw (I3) to (J1);
            \draw (I3) to (J2);
            \draw (I3) to (J3);
        \end{tikzpicture}
        \caption{Vollständig bipartiter Teilgraph zu Knoten $I \neq J$ mit nicht invertierbarem $M_{IJ}$}
        \label{fig:biclique-graph}
    \end{subfigure}
    \hfill
    \begin{subfigure}[b]{0.47\textwidth}
        \centering
        \begin{tikzpicture}[node distance={5mm}, main/.style = {draw, circle, minimum size=1.2cm}]
            % Nodes
            \node[main] (I0) at ({360/6 * 0}: 2cm) {\small $I+2$};
            \node[main] (I1) at ({360/6 * 1}: 2cm) {\small $I+1$};
            \node[main] (I2) at ({360/6 * 2}: 2cm) {\small $I$};
            \node[main] (I3) at ({360/6 * 3}: 2cm) {\small $I-1$};
            \node (D1) at ({360/6 * 4}: 2cm) {\large $\cdots$};
            \node[main] (I4) at ({360/6 * 5}: 2cm) {\small $I+3$};

            \draw (I0) to[loop, out=22.5, in=-22.5, distance=1cm] (I0);         
            \draw (I0) to (I1);
            \draw (I0) to (I2);
            \draw (I0) to (I3);
            \draw (I0) to (I4);            
            \draw (I1) to[loop, out=82.5, in=37.5, distance=1cm] (I1);            
            \draw (I1) to (I2);
            \draw (I1) to (I3);
            \draw (I1) to (I4);
            \draw (I2) to[loop, out=142.5, in=97.5, distance=1cm] (I2);    
            \draw (I2) to (I3);
            \draw (I2) to (I4);
            \draw (I3) to[loop, out=202.5, in=157.5, distance=1cm] (I3);    
            \draw (I3) to (I4);
            \draw (I4) to[loop, out=322.5, in=277.5, distance=1cm] (I4);    
        \end{tikzpicture}
        \caption{Vollständiger Teilgraph zum Knoten $I$ mit nicht invertierbarem $M_{II}$}
        \label{fig:complete-graph}
    \end{subfigure}
    \caption{Bipartite Untergraphen in $G_a$ und $G_{a^{-1}}$}
    \label{fig:subgraphs}
\end{figure}

Aus \Cref{satz:skalierung} ergeben sich weitere Symmetrien in $G_a$. Demnach sind folgende Aussagen äquivalent:

\begin{align*}
    (\det{} M_{I,J})(a) &= 0 \\
    (\det{} M_{-I,-J})(a) &= 0 \\
    (\det{} M_{-I,J})(a^{-1}) &= 0 \\
    (\det{} M_{I,-J})(a^{-1}) &= 0
\end{align*}

Zu den in \Cref{fig:subgraphs} beschriebenen Teilgraphen aus $G_a$ sind die folgende Teilgraphen isomorph:

\begin{align*}
    &\text{Zu } -I \text{ und } -J \text{ in } G_a \\
    &\text{Zu } -I \text{ und } J \text{ in } G_{a^{-1}} \\
    &\text{Zu } I \text{ und } -J \text{ in } G_{a^{-1}} \\
\end{align*}

Somit induziert die bijektive Abbildung $f:V \to V, I \mapsto -I$ einen nicht trivialen Graphautomorphismus und einen Graphisomorphismus zwischen $G_a$ und $G_{a^{-1}}$.

\todo{Graph mit vollständig Bipartiten Untergraphen konstruieren und erläutern, Isomorphismus über das inverse Element erläutern}

\section{Singuläre Submatrizen}

Diese Kapitel befasst sich mit der Invertierbarkeit von Submatrizen für gegebene $a \in \field{p}[n]$. Dafür wollen wir im Folgenden hinreichende Bedingungen an die Indexmengen $I$ und $J$ stellen.

Die erste Aussage bezieht sich auf die lineare Abhängigkeit zweier Spalten/Zeilen. Für die vorrangehenden Bespiele sei stets $a = 2 \in \field{11}$ mit der Ordnung $\ord{a} = 10$. Seien zunächst $I=\{0,1,6,11\}$ und $J=\{0,1,2,3\}$. Dann ist 
\begin{equation*}
    M_{IJ}(2) = \begin{pmatrix}
        2^{0} & 2^{0} & 2^{0} & 2^{0} \\
        2^{0} & 2^{1} & 2^{2} & 2^{3} \\
        2^{0} & 2^{6} & 2^{12} & 2^{18} \\
        2^{0} & 2^{11} & 2^{22} & 2^{33} 
    \end{pmatrix} \underbrace{=}_{2^{10}=1} \begin{pmatrix}
        2^{0} & 2^{0} & 2^{0} & 2^{0} \\
        2^{0} & 2^{1} & 2^{2} & 2^{3} \\
        2^{0} & 2^{6} & 2^{2} & 2^{8} \\
        2^{0} & 2^{1} & 2^{2} & 2^{3} 
    \end{pmatrix}
\end{equation*}

Da die Differenz $i_4 - i_2 = 11 - 1 = 10$ der Ordnung von $a$ entspricht, unterscheiden sich die Exponenten in diesen Zeilen um ein Vielfaches dieser Ordnung. Dadurch sind die zweite und vierte Zeile äquivalent und $(\det M_{IJ})(a) = 0$. 
Seien nun $I=\{0,2,6,7\}$ und $J=\{0,2,4,6\}$. In diesen Indexmengen existieren keine zwei Elemente mit dem passenden Abstand von $10$. Die zweite und vierte Zeile der Submatrix

\begin{equation*}
    M_{IJ}(2) = \begin{pmatrix}
        2^{0} & 2^{0} & 2^{0} & 2^{0} \\
        2^{0} & 2^{4} & 2^{8} & 2^{12} \\
        2^{0} & 2^{12} & 2^{24} & 2^{36} \\
        2^{0} & 2^{14} & 2^{28} & 2^{42} 
    \end{pmatrix} \underbrace{=}_{2^{10}=1} \begin{pmatrix}
        2^{0} & 2^{0} & 2^{0} & 2^{0} \\
        2^{0} & 2^{4} & 2^{8} & 2^{2} \\
        2^{0} & 2^{2} & 2^{4} & 2^{6} \\
        2^{0} & 2^{4} & 2^{8} & 2^{2} 
    \end{pmatrix}
\end{equation*}

sind dennoch wieder äquivalent. Dies ergibt sich aus der Faktorisierung von $10 = 5 \cdot 2$. Die Differenz $i_4 - i_2 = 7 - 2 = 5$ entspricht dem ersten Faktor. Da alle Differenzen in $I$ ein Vielfaches der $2$ sein, unterscheiden sich die Exponenten in Zeile 2 und 4 wieder um ein Vielfaches der Ordnung, wodurch die lineare Abhängigkeit entsteht. Das folgende Lemma formalisiert dieses Resultat.



\begin{lemma} \label{lemma:equal-columns}
    Seien $a \in \field{p}[n], o = \ord{a} = v\cdot w$ und $I,J \subseteq \mathbb{Z}$.
    Falls
    \begin{equation} \label{equation:all-equal}
        \forall \; i, i' \in I, i \neq i':\quad i \equiv i' \pmod v
    \end{equation}
    und
    \begin{equation} \label{equation:two-equal}
        \exists \; j, j' \in J, j \neq j':\quad  j \equiv j' \pmod w
    \end{equation}
    gelten, folgt
    \begin{equation*}
        (\det M_{IJ})(a) = 0
    \end{equation*}
\end{lemma}

{
\Crefname{equation}{Bedingung}{Bedingungen}

\begin{proof}
    Wir betrachten $j,j' \in J$, welche die obige \Cref{equation:two-equal} erfüllen und zeigen, dass die dazugehörigen Spalten in $M_{IJ}$ linear abhängig sind. Seien $j = (c + bw)$, $j' = (c + b'w)$ und $I = \{(d+l_1v),\dots,(d+l_kv)\}$. Die Spalten zu $j$ und $j'$ haben die Form
    \begin{align*}
        \begin{pmatrix}
            a^{i_1j} & a^{i_1j'} \\
            \vdots & \vdots \\
            a^{i_kj} & a^{i_kj'}
        \end{pmatrix} &=
        \begin{pmatrix}
            a^{(d+l_1v)(c + bw)} & a^{(d+l_1v)(c + b'w)} \\
            \vdots & \vdots \\
            a^{(d+l_kv)(c + bw)} & a^{(d+l_kv)(c + b'w)}
        \end{pmatrix} \\
        &= \begin{pmatrix}
            a^{dc + dbw +l_1vc + l_1bvw} & a^{dc + db'w +l_1vc + l_1b'vw} \\
            \vdots & \vdots \\
            a^{dc + dbw +l_kvc + l_kbvw} & a^{dc + db'w +l_kvc + l_kb'vw}
        \end{pmatrix} \\
        &= \begin{pmatrix}
            a^{dc + dbw +l_1vc + l_1bvw} & a^{dc + db'w +l_1vc + l_1b'vw} \\
            \vdots & \vdots \\
            a^{dc + dbw +l_kvc + l_kbvw} & a^{dc + db'w +l_kvc + l_kb'vw}
        \end{pmatrix} \\
    \end{align*}
    Die konstanten Terme $a^{dc + dbw}$ und $a^{dc + db'w}$ können aus den Spalten faktorisiert werden, sodass verbleiben
    \begin{equation*}
        \begin{pmatrix}
            a^{l_1vc} {a^{vw}}^{l_1b} & a^{l_1vc} {a^{vw}}^{l_1b'} \\
            \vdots & \vdots \\
            a^{l_kvc} {a^{vw}}^{l_kb} & a^{l_kvc} {a^{vw}}^{l_kb'}
        \end{pmatrix}
    \end{equation*}
    Da $a^{vw} = a^{o} = 1$ gilt, verbleiben die gleichen Spalten.
    \begin{equation*}
        \begin{pmatrix}
            a^{l_1vc} & a^{l_1vc} \\
            \vdots & \vdots \\
            a^{l_kvc} & a^{l_kvc}
        \end{pmatrix}
    \end{equation*}

    Aus der linearen Abhängigkeit der beiden Spalten folgt $(\det M_{IJ})(a) = 0$.
\end{proof}

Die \Cref{equation:all-equal}, dass alle Elemente einer Indexmenge in der selben Restklasse liegen müssen, ist eine starke Einschränkung in der Wahl der Indizes. Diese Anforderung kann abgeschwächt werden, wenn zusätzliche Bedingungen an die jeweils andere Indexmenge gestellt werden. 



\begin{satz} \label{satz:equal-columns-subs}
    Seien $a \in \field{p}[n], o = \ord{a} = v\cdot w$ und $I,J \subseteq \mathbb{Z}$.
    Falls ein $I' \subseteq I$ mit $|I'| = \gamma \geq 1$ existiert, dass \Cref{equation:all-equal} erfüllt und alle $J' \subseteq J$ mit $|J'| = \gamma$ \Cref{equation:two-equal} erfüllen, folgt
    \begin{equation*}
        (\det M_{IJ})(a) = 0
    \end{equation*}
\end{satz}

\begin{proof}
    Zu Bestimmung der Determinante führen wir eine Zeilenentwicklung über die Indizes $I'$ durch. 
\begin{equation*}
    (\det M_{IJ})(a) = \left( \sum_{J':|J'| = \gamma} (-1)^{\sum I' + \sum J'} \left( \det M_{I'J'} \cdot \det M_{I\backslash I',J\backslash J'} \right) \right)(a)
\end{equation*}
Da $I'$ und alle $J'$ die \Cref{equation:all-equal,equation:two-equal} erfüllen, kann \Cref{lemma:equal-columns} auf die Terme $\det M_{I'J'}$ angewendet werden. Somit ist jeder Summand gleich $0$ und die Aussage folgt.
\end{proof}


Wir betrachten zum Beispiel die Indexmengen $I = \{0,2,3,4\}$ und $J = \{0,2,5,7\}$. Da die Teilmenge $I\backslash\{3\}$ die \Cref{equation:all-equal} erfüllt, wird entlang der Zeile zum Exponenten $3$ entwickelt. Die für die Berechnung relevanten Submatrizen ergeben sich aus den Paarungen

\begin{align*}
    I_1=\{0,2,4\} \text{ und } J_1=\{2,5,7\}, \\
    I_2=\{0,2,4\} \text{ und } J_2=\{0,5,7\}, \\
    I_3=\{0,2,4\} \text{ und } J_3=\{0,2,7\}, \\
    I_4=\{0,2,4\} \text{ und } J_4=\{0,2,5\}.
\end{align*}

Diese erfüllen jeweils \Cref{equation:all-equal,equation:two-equal}, sodass \Cref{satz:equal-columns-subs} angewendet werden kann und $(\det M_{IJ})(a) = 0$ folgt.
}

Anstatt alle Teilmengen von $J$ einer bestimmten Kardinalität auf die Bedingung zu prüfen, kann auch das minimale $\gamma$ gesucht werden

\todo{Maximales Gamma bestimmen, Abschätzung Mächtigkeit zu Partition}

\section{Spezielle Indexmengen}

\newcolumntype{C}{>{$}c<{$}}

In diesem Kapitel wollen wir spezielle Indexmengen betrachten und Bedingungen aufstellen, wann die Submatrix $M_{IJ}$ für ein $a \in \field{p}[n]$ invertierbar ist. Wir betrachten zunächst den allgemeinen Fall für $|I| = |J| = 3$. Seien $I = \{0, \alpha, \beta\}$ und $J = \{0, \gamma, \delta\}$. Dann besitzt $M_{IJ}$ die Form 
\begin{equation*}
    \begin{pmatrix}
        1 & 1 & 1 \\
        1 & x^{\alpha\gamma} & x^{\beta\gamma} \\
        1 & x^{\alpha\delta} & x^{\beta\delta}
    \end{pmatrix}.
\end{equation*}

Mithilfe von Zeilenumformungen und anschließender Spaltenentwicklung ergibt sich die Determinante wie folgt:

\begin{align*}
    \det \begin{pmatrix}
        1 & 1 & 1 \\
        1 & x^{\alpha\gamma} & x^{\beta\gamma} \\
        1 & x^{\alpha\delta} & x^{\beta\delta}
    \end{pmatrix} 
    &= \det \begin{pmatrix}
        1 & 1 & 1 \\
        0 & x^{\alpha\gamma} -1 & x^{\beta\gamma} -1 \\
        0 & x^{\alpha\delta} -1 & x^{\beta\delta} -1
    \end{pmatrix} \\
    &= \det \begin{pmatrix}
        x^{\alpha\gamma} -1 & x^{\beta\gamma} -1 \\
        x^{\alpha\delta} -1 & x^{\beta\delta} -1
    \end{pmatrix} \\
    &= (x^{\alpha\gamma} -1)(x^{\beta\delta} -1) - (x^{\alpha\delta} -1)(x^{\beta\gamma} -1)
\end{align*}

Für ein $a \in \field{p}[n]$ ist die Submatrix $M_{IJ}$ somit genau dann nicht invertierbar, falls $a$ die Gleichung $(x^{\alpha\gamma} -1)(x^{\beta\delta} -1) = (x^{\alpha\delta} -1)(x^{\beta\gamma} -1)$ löst. Da diese Gleichung noch sehr allgemein ist, setzen wir noch konkreter $I = \{0,1,3\}$. Dadurch reduziert sich die Bedingung zu

\begin{align}
        & (x^{\gamma} -1)(x^{3\delta} -1) = (x^{\delta} -1)(x^{3\gamma} -1) \nonumber \\
    \iff & \frac{(x^{3\delta} -1)}{(x^{\delta} -1)} = \frac{(x^{3\gamma} -1)}{(x^{\gamma} -1)} \nonumber \\
    \iff & {x^\delta}^2 + x^{\delta} + 1 = {x^{\gamma}}^2 + x^{\gamma} + 1 \nonumber \\
    \iff & {x^\delta}^2 + x^{\delta} = {x^{\gamma}}^2 + x^{\gamma}. \label{equation:013}
\end{align}

Anders formuliert sind nun zwei Elemente $v_1 = a^\delta$ und $v_2 = a^\gamma$ gesucht, die $x^2 + x = m$ für ein festes $m \in \field{p}[n]$ lösen. Abhängig von der Charakteristik des Körpers existieren zwei unterschiedliche Lösungsformeln.

\begin{equation*}
    v_{1,2} = \begin{cases}
        -2^{-1} \pm \sqrt{2^{-2} + m}                                   & \text{falls } \mathrm{char}(\field{p}[n]) \neq 2 \\  
        \sum_{k=1}^{n-1} m^{2^j}(\sum_{l=0}^{k-1} u^{2^k}),\quad v_1 + 1 & \text{sonst}
    \end{cases} 
\end{equation*}

Sofern $v_1,v_2 \neq 0$ sind, existieren für jeden Erzeuger $a \in \field{p}[n]$ Exponenten $\gamma$ und $\delta$ mit $v_1 = a^\gamma$ und $v_2 = a^\delta$. Für diese Erzeuger ist dann die \Cref{equation:013} erfüllt und $M_{IJ}$ nicht invertierbar. 

Im Folgenden betrachten wir ein Beispiel für den Körper $\field{11}$, wobei $-2^{-1} = -6 = 5$ und $2^{-2} = 3$ gilt. In \Cref{table:sol_013F11} sind für jedes 


{\renewcommand{\arraystretch}{1.5}
\begin{table}
    \centering
    \begin{tabular}{|C|C|C|C|C|C|C|C|C|C|C|C|C}
    \hline
    m          & 0    & 1   & 2   & 3 & 4 & 5 & 6   & 7  & 8   & 9   & 10 \\
    \hline
    \sqrt{3+m} & 5    & 2   & 4   & - & - & - & 3   & -  & 0   & 1   & -  \\
    \hline
    v_{1,2} = 5 \pm \sqrt{3+m}   & 10,0 & 3,7 & 9,1 & - & - & - & 8,2 & -  & 5,5 & 6,4 & -  \\
    \hline
    \end{tabular}
    \caption{Lösungen $v_{1,2}$ zu $x^2 + x = m$ für festes $m \in \field{11}$.} \label{table:sol_013F11}
\end{table}
}

\begin{table}
    \centering
    \begin{tabular}{|C|C|C|C|C|C|C|C|C|C|C|C|}
    \hline
    \text{\backslashbox{a}{i}} & 0 & 1  & 2 & 3 & 4 & 5  & 6 & 7 & 8 & 9 & 10 \\ \hline
    1  & 1 & 1  &   &   &   &    &   &   &   &   &   \\ \hline
    2  & 1 & 2  & 4 & 8 & 5 & 10 & 9 & 7 & 3 & 6 & 1 \\ \hline
    3  & 1 & 3  & 9 & 5 & 4 & 1  &   &   &   &   &   \\ \hline
    4  & 1 & 4  & 5 & 9 & 3 & 1  &   &   &   &   &   \\ \hline
    5  & 1 & 5  & 3 & 4 & 9 & 1  &   &   &   &   &   \\ \hline
    6  & 1 & 6  & 3 & 7 & 9 & 10 & 5 & 8 & 4 & 2 & 1 \\ \hline
    7  & 1 & 7  & 5 & 2 & 3 & 10 & 4 & 6 & 9 & 8 & 1 \\ \hline
    8  & 1 & 8  & 9 & 6 & 4 & 10 & 3 & 2 & 5 & 7 & 1 \\ \hline
    9  & 1 & 9  & 4 & 3 & 5 & 1  &   &   &   &   &   \\ \hline
    10 & 1 & 10 & 1 &   &   &    &   &   &   &   &   \\ \hline
    \end{tabular}
    \caption{Alle von einem $a \in \field{11}$ erzeugten Untergruppen.} \label{table:subgroupsF11}
\end{table}

{\renewcommand{\arraystretch}{1.5}
\begin{table}
    \centering
    \begin{tabular}{|C|C|C|C|C|C|C|C|C|}
    \hline
    m               & 0 & 1 & o & o + 1 & o^2 & o^2 + 1 & o^2 + o  & o^2 + o + 1 \\
    \hline
    \mathrm{Tr}(m)  & 0 & 1 & 0 & 1     & 0   & 1       & 0        & 1 \\
    \hline
    v_1         & 0 & - & o^2 & - & o^2 + o & - & o & - \\
    \hline
    \end{tabular}
    \caption{Lösungen $v_1$ zu $x^2 + x = m$ für festes $m \in \field{2}[3]$.} \label{table:sol_013F2_3}
\end{table}
}

\begin{table}
    \centering
    \begin{tabular}{|C|C|C|C|C|C|C|C|C|C|C|C|}
    \hline
    \text{\backslashbox{i}{a}}& 1 & o       & o+1     & o^2     & o^2 + 1 & o^2 + o & o^2 + o + 1 \\ \hline
    0 & 1 & 1       & 1       & 1       & 1       & 1       & 1           \\ \hline
    1 & 1 & o       & o+1     & o^2     & o^2+1   & o^2+o   & o^2+o+1     \\ \hline
    2 &   & o^2     & o^2+1   & o^2+o   & o^2+o+1 & o       & o+1         \\ \hline
    3 &   & o+1     & o^2     & o^2+1   & o^2+o   & o^2+o+1 & o           \\ \hline
    4 &   & o^2+o   & o^2+o+1 & o       & o+1     & o^2     & o^2+1       \\ \hline
    5 &   & o^2+o+1 & o       & o+1     & o^2     & o^2+1   & o^2+o       \\ \hline
    6 &   & o^2+1   & o^2+o   & o^2+o+1 & o       & o+1     & o^2         \\ \hline
    7 &   & 1       & 1       & 1       & 1       & 1       & 1           \\ \hline
    \end{tabular}
    \caption{Alle von einem $a \in \field{2}[3]$ erzeugten Untergruppen.} \label{table:subgroupsF2_3}
\end{table}


\bibliographystyle{plain}
\bibliography{ref}

\listoftodos

\end{document}