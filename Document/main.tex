\documentclass{article}

\usepackage[ngerman]{babel}
\usepackage{amsthm}
\usepackage{amsfonts}
\usepackage{amsmath}
\usepackage[unicode=true,
 bookmarks=true,bookmarksnumbered=false,bookmarksopen=false,
 breaklinks=false,pdfborder={0 0 1},backref=false,colorlinks=false]
 {hyperref}
\usepackage[noabbrev,nameinlink, ngerman]{cleveref}

\newtheorem{satz}{Satz}
\newtheorem{lemma}[satz]{Lemma}
\newtheorem{korollar}[satz]{Korollar}
\theoremstyle{definition}
\newtheorem{definition}{Definition}

\renewcommand{\det}{det\;}
\newcommand{\field}[2]{\mathbb{F}_{#1^#2}}


\begin{document}

\begin{definition}
    Seien $I := \{i_1,\dots, i_k\}$ und $J := \{j_1,\dots, j_k\}$ zwei Teilmengen von $\mathbb{Z}$ für $k \in \mathbb{N}^+$. Dann ist $M_{IJ} := (x^{ij})_{i \in I, j \in J}$. 
\end{definition}

\include{sections/LineareIndextransformationen}

\end{document}