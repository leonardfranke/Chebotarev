\documentclass{article}

\usepackage[ngerman]{babel}
\usepackage{amsthm}
\usepackage{amsfonts}
\usepackage{amsmath}
\usepackage[unicode=true,
 bookmarks=true,bookmarksnumbered=false,bookmarksopen=false,
 breaklinks=false,pdfborder={0 0 1},backref=false,colorlinks=false]
 {hyperref}
\usepackage[noabbrev,nameinlink, ngerman]{cleveref}

\newtheorem{satz}{Satz}
\newtheorem{lemma}[satz]{Lemma}
\newtheorem{korollar}[satz]{Korollar}
\theoremstyle{definition}
\newtheorem{definition}{Definition}

\renewcommand{\det}{det\;}



\begin{document}

\begin{definition}
    Seien $I := \{i_1,\dots, i_k\}$ und $J := \{j_1,\dots, j_k\}$ zwei Teilmengen von $\mathbb{Z}$ für $k \in \mathbb{N}^+$. Dann ist $M_{IJ} := (x^{ij})_{i \in I, j \in J}$. 
\end{definition}

\begin{lemma} \label{lemma:transpose}
    \begin{equation*}
        \det M_{IJ} = \det M_{JI}
    \end{equation*}
\end{lemma}

\begin{proof}
    \begin{align*}
        \det M_{IJ} &= \det (x^{ij}) \\
                    &= \det (x^{ji}) \\
                    &= \det M_{JI}
    \end{align*}
\end{proof}

\begin{lemma} \label{lemma:translation}
    Sei $I + m := \{ i + m, i \in I\}$ für $m \in Z$. Dann gilt
    \begin{equation*}
        \det{} M_{I+m,J} = x^{m(j_1+\cdots +j_k)} \det M_{I,J}
    \end{equation*}
\end{lemma}

\begin{proof}
    Die Einträge der Matrix $M_{I+m,J}$ haben die Form 
    \begin{equation*}
        x^{(i+m)j} = x^{ij} \cdot x^{mj}
    \end{equation*} für $i\in I, j\in J$.

    Jeder Eintrag einer Spalte $j \in J$ enthält den Faktor $x^{mj}$. Aus der Multilinearität der Determinante folgt
    \begin{align*}
        \det M_{I+m,J}  &= \det (x^{ij} \cdot x^{mj}) \\ 
                        &= \prod_{j \in J} x^{mj} \det (x^{ij}) \\ 
                        &= \prod_{j \in J} x^{mj}  \det M_{I,J} \\
                        &= x^{m(j_1+\cdots +j_k)} \det M_{I,J}
    \end{align*}
\end{proof}

\begin{satz}
    Seien $I,J \subseteq \mathbb{Z}$ und $m,n \in \mathbb{Z}$. Dann gilt
    \begin{equation*}
        \det{} M_{I+m,J+n} = x^{mnk} x^{m(j_1 +\cdots + j_k)} x^{n(i_1+\cdots +i_k)} \det{} M_{I,J}
    \end{equation*}
\end{satz}

\begin{proof}
    Aus \cref{lemma:transpose,lemma:translation} folgt direkt
    \begin{align*}
        \det{} M_{I+m,J+n}  &= x^{m(j_1 + n +\cdots + j_k + n)} \det{} M_{I,J+n} \\
                            &= x^{m(j_1 +\cdots + j_k + kn)} \det{} M_{I,J+n} \\
                            &= x^{mnk} x^{m(j_1 +\cdots + j_k)} \det{} M_{I,J+n} \\
                            &= x^{mnk} x^{m(j_1 +\cdots + j_k)} \det{} M_{J+n,I} \\
                            &= x^{mnk} x^{m(j_1 +\cdots + j_k)} x^{n(i_1+\cdots +i_k)} \det{} M_{J,I} \\
                            &= x^{mnk} x^{m(j_1 +\cdots + j_k)} x^{n(i_1+\cdots +i_k)} \det{} M_{I,J} \\
    \end{align*}
\end{proof}

\begin{satz}
    Sei $aI := \{ a \cdot i, i \in I\}$ für $a \in Z$. Der Minor zu $I,J$ besitzt die Form  
    \begin{equation*}
        \det{} M_{I,J} = \beta_l x^{\alpha_l} + \cdots + \beta_0 x^{\alpha_0}.
    \end{equation*} 
    Der Minor zu $aI,bJ$ hat dann die Form
    \begin{equation*}
        \det{} M_{aI,bJ} =  \beta_l x^{ab \alpha_l} + \cdots + \beta_0 x^{ab \alpha_0}
    \end{equation*}
\end{satz}

\begin{proof}
    Die Exponenten $\alpha_k$ für $I,J$ haben die Form
    \begin{equation*}
        \alpha_k = \sum_{v,w \in V_k,W_k \subseteq I,J} vw.
    \end{equation*}  
    Die Exponenten $\alpha_k'$ für $aI,bJ$ entsprechend 
    \begin{align*}
        \alpha_k'   &= \sum_{v',w' \in V_k',W_k' \subseteq aI,bJ} v'w' \\
                    &= \sum_{v,w \in V_k,W_k \subseteq I,J} avbw \\
                    &= ab \sum_{v,w \in V_k,W_k \subseteq I,J} vw \\
                    &= ab \alpha_k
    \end{align*} 
\end{proof}

\begin{korollar}
    Der Minor zu $I,J$ besitzt die Form  
    \begin{equation*}
        \det{} M_{I,J} = \beta_l x^{\alpha_l} + \cdots + \beta_0 x^{\alpha_0}.
    \end{equation*} 
    Der Minor zu $aI + m,bJ + n$ hat dann die Form
    \begin{equation*}
        \det{} M_{aI + m,bJ + n} = x^{mb(j_1+\cdots +j_k)} x^{na(i_1+\cdots +i_k)} \left( \beta_l x^{ab \alpha_l} + \cdots + \beta_0 x^{ab \alpha_0} \right)
    \end{equation*}
\end{korollar}

\end{document}