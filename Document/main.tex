\documentclass[parskip=half]{scrartcl}

\usepackage{tikz}
\usepackage{comment}
\usepackage{subcaption}
\usepackage{todonotes}
\usepackage{xparse}
\usepackage{slashbox}
\usepackage[ngerman]{babel}
\usepackage{amsthm}
\usepackage{amsfonts}
\usepackage{amsmath}
\usepackage{array}
\usepackage{mathtools}
\usepackage[unicode=true,
 bookmarks=true,bookmarksnumbered=false,bookmarksopen=false,
 breaklinks=false,pdfborder={0 0 1},backref=false,colorlinks=false]
 {hyperref}
\usepackage[noabbrev,nameinlink, ngerman]{cleveref}

\newtheorem{satz}{Satz}
\newtheorem{lemma}[satz]{Lemma}
\newtheorem{korollar}[satz]{Korollar}
\theoremstyle{definition}
\newtheorem{definition}{Definition}
\Crefname{satz}{Satz}{Sätze}

\newcolumntype{C}{>{$}c<{$}}

\newcommand{\grad}[1]{\mathrm{grad}(#1)}
\renewcommand{\det}{\mathrm{det}\;}
\NewDocumentCommand{\ord}{m o}{%
 \mathrm{ord}_{\IfValueT{#2}{{#2}}}(#1)%
}
\newcommand{\charac}{\mathrm{char}}
\newcommand{\Tr}{\mathrm{Tr}}
\newcommand{\Modulo}{\mathbin\mathrm{mod}}
\NewDocumentCommand{\field}{m o}{%
  \mathbb{F}_{#1\IfValueT{#2}{^{#2}}}%
}

\begin{document}

\section{Einleitung}

\todo{Thema kontrollieren}

Diese Bachelorarbeit beschäftigt sich mit den Minoren verallgemeinerter Vandermondematrizen. Die betrachteten Matrizen besitzen die Form \begin{equation*}
    \left( x^{ij} \right)_{0\leq i,j \leq k-1} = \begin{pmatrix}
        1     & 1    & 1    & 1    &\cdots& 1 \\
        1     & x^1  & x^2  & x^3  &\cdots& x^{k-1} \\
        1     & x^2  & x^4  & x^6  &\cdots& x^{2k-2} \\
        1     & x^3  & x^6  & x^9  &\cdots& x^{3k-3} \\
        \vdots&\vdots&\vdots&\vdots&\ddots&\vdots \\
        1     &x^{k-1}&x^{2k-2}&x^{3k-3}&\cdots&x^{(k-1)(k-1)}
    \end{pmatrix},
\end{equation*}
wobei für die Variable $x$ stets Elemente eines Körpers $\field{}$ eingesetzt werden. Für gegebenes $k$ sind all die Elemente $a$ des Körpers gesucht, sodass jede quadratische Submatrix von $\left( a^{ij} \right)_{0\leq i,j \leq k-1}$ invertierbar ist. Der Mathematiker Chebotarev beschäftige sich mit der Substitution von komplexen Zahlen. Konkret zeigt er im Jahr 1926, dass für primitive $p$-te Einheitswurzel (zum Beispiel $a=e^{2\pi i/p}$) genau dann alle quadratischen Submatrizen von $\left( a^{ij} \right)_{0\leq i,j \leq p-1}$ invertierbar sind, wenn $p$ Primzahl ist. Bis auf einen Vorfaktor entspricht dieses Beispiel genau der diskreten Fouriermatrix \cite{CheboProof}. Analog existiert eine Aussage über $p$-te primitive Einheitswurzeln endlicher Körper. Diese gilt jedoch nur für Primzahlpotenzen $q^{p-1}$ mit fester Ordnung $\ord{q}[p] = p-1$ und genügend großem $q$ \cite{CheboFiniteFields}.
\todo{Schranke für q verstehen und einbauen}

Diese Arbeit betrachtet ebenfalls endliche Körper $\field{p}[n]$ und untersucht die Invertierbarkeit der Submatrizen für Körperelemente beliebiger Ordnung. Da die Strukturen der Submatrizen beim Substituieren der $0$ trivial sind, werden ausschließlich invertierbare Elemente aus $\field{p}[n]^\times$ eingesetzt. Konkret leben die betrachteten Matrizen im $(\field{p}[n]^\times [X])^{k \times k}$ und die Minoren demnach im Polynomring $\field{p}[n]^\times [X]$. Da für die invertierbaren Körperelemente auch ganzzahlige negative Exponenten definiert sind, wird im Hauptteil dieser Arbeit die unendliche Matrix $M = \left( x^{ij} \right)_{i,j \in \mathbb{Z}}$ analysiert. Die dazugehörigen Submatrizen werden mit jeweils zwei endlichen Teilmengen $I,J$ der ganzen Zahlen indiziert. Sei entsprechend $M_{IJ} \coloneqq \left( x^{ij} \right)_{i \in I,j \in J}$. Für zum Beispiel $I,J = \{-2, \dots, 4\}$ sieht diese folgendermaßen aus.
\begin{equation*}
    M_{IJ} = \begin{pmatrix}
        x^4     & x^{2} & 1    & x^{-2} & x^{-4}& x^{-6}& x^{-8}\\
        x^{2}   & x^{1} & 1    & x^{-1} & x^{-2}& x^{-3}& x^{-4}\\
        1       & 1     & 1    & 1      & 1     & 1     & 1     \\
        x^{-2}  & x^{-1}& 1    & x^{1}  & x^{2} & x^{3} & x^{4} \\
        x^{-4}  & x^{-2}& 1    & x^{2}  & x^{4} & x^{6} & x^{8} \\
        x^{-6}  & x^{-3}& 1    & x^{3}  & x^{6} & x^{9} & x^{12} \\
        x^{-8}  & x^{-4}& 1    & x^{4}  & x^{8} & x^{12} & x^{16} \\
    \end{pmatrix}
\end{equation*}

Die Sortierung beim Angeben der Indexmengen ist irrelevant, da das Austauschen mehrerer Zeilen oder Spalten bei den Determinanten nur zum Vorzeichenwechsel führt. Aufgrund von zyklischen Eigenschaften der multiplikativen Gruppe $\field{p}[n]^\times$ nimmt die unendliche Matrix $M$ stehts die Form einer Blockmatrix an. Für $\field{3}$, den Körper mit 3 Elementen, vereinfacht sich die bereits gezeigte Submatrix wie folgt.

\todo{Blockmatrix vllt verschieben}

\begin{equation*}
    M_{IJ} = \left( \begin{array}{cc|ccc|cc}
        x^1     & x^{2} & 1     & x^{1} & x^{2} & 1 & x^{1} \\
        x^{2}   & x^{1} & 1     & x^{2} & x^{1} & 1 & x^{2} \\
        \hline
        1       & 1     & 1     & 1     & 1     & 1 & 1     \\
        x^{1}   & x^{2} & 1     & x^{1} & x^{2} & 1 & x^{1} \\
        x^{2}   & x^{1} & 1     & x^{2} & x^{1} & 1 & x^{2} \\
        \hline
        1       & 1     & 1     & 1     & 1     & 1 & 1     \\
        x^1     & x^2   & 1     & x^1   & x^2   & 1 & x^1   \\
    \end{array} \right)
\end{equation*}

Diese Arbeit ist wie folgt gegliedert. In \Cref{sec:grundlagen} werden die benötigten Grundlagen zu endlichen Körpern und zyklischen Gruppen geklärt. Der Einfluss von affinen Abbildungen auf die Indexmengen und den dazugehörigen Minoren wird in \Cref{sec:indextransformationen} untersucht. In \Cref{sec:singulaereSubmatrizen} wird ein hinreichendes Kriterium erarbeitet, das angibt, welche Indexmengen für eine feste Gruppenordnung zu singulären Submatrizen führen. Weitere spezielle Indexmengen, deren Invertierbarkeit sich nicht über die vorher gezeigten Sätze herleiten lässt, werden abschließend in \Cref{sec:spezielleIndexmengen} untersucht.


\section{Grundlagen}

In diesem Kapitel klären wir die für den Hauptteil benötigten Definitionen und Sätze über endliche Körper. Zunächst konstruieren wir diese über Restklassenringe der ganzen Zahlen und über spezielle Polynomringe. Anschließend betrachten wir Untergruppen dieser Körper und untersuchen ihre zyklische Gruppen, Erzeuger und Ordnungen.

\subsection{Restklassenringe}

Die ersten grundlegenden endlichen Körper lassen sich über Restklassenringe der ganzen Zahlen konstruieren. Dazu betrachten wir im Folgenden Restabbildungen.

\begin{satz}
    Sei $\mathbb{Z}_n := \{0,\dots,n-1\}$. Dann existieren für jedes $a \in \mathbb{Z}$ eindeutig bestimmte Zahlen $f \in \mathbb{Z}$ und $r \in \mathbb{Z}_n$ mit $a = n \cdot f + r$.
\end{satz}

Sei zum Beispiel $n = 4$. Dann gelten folgende Darstellung.

\begin{align*}
    -2 &= 4 \cdot -1 + 2 \\
    -1 &= 4 \cdot -1 + 3 \\
    0 &= 4 \cdot 0 + 0 \\
    1 &= 4 \cdot 0 + 1 \\
    2 &= 4 \cdot 0 + 2 \\
    3 &= 4 \cdot 0 + 3 \\
    4 &= 4 \cdot 1 + 0 \\
    5 &= 4 \cdot 1 + 1
\end{align*}

Die Zahl $r$ wird als der Rest von $a$ modulo $n$ bezeichnet. Wir schreiben auch kurz $r = a \bmod n$. Die Abbildung $\varrho_n : \mathbb{Z} \rightarrow \mathbb{Z}_n$, welche jeder ganzen Zahl ihren Rest zuordnet, ist die Restabbildung. Mit ihr kann aus der Restklasse $\mathbb{Z}_n$ einen Ring konstruiert werden.

\begin{satz}
    Sei $n \in \mathbb{N}$. Die Menge $\mathbb{Z}_n$ zusammen mit der Multiplikation $a \otimes b \mapsto \varrho_n(a \cdot b)$ und der Addition $a \oplus b \mapsto \varrho_n(a + b)$ bildet einen Ring. Dieser wird als Restklassenring modulo $n$ bezeichnet.
\end{satz}

Die Operationen im Restklassenring nutzen die von den ganzen Zahlen bekannte Addition und Multiplikation mit anschließenden Modulodivision. \Cref{table:tableZ4,table:tableZ5} zeigen die Addition- und Multiplikationstafel für $n=4$ und $n=5$. In diesen können einige der Ringaxiome verifiziert werden. Zum Beispiel existieren jeweils die neutralen Elemente $0$ und $1$. Außerdem besitzt jedes Element eine additiv Inverses. Bezüglich der Multiplikation existieren jedoch nicht immer inverse Elemente. Dies zeigt, dass der $\mathbb{Z}_n$ im Allgemeinen keinen Körper bildet. 

\begin{table}[]
    \centering
    \begin{tabular}{|C|C|C|C|C|}
    \hline
    + & 0  & 1 & 2 & 3 \\ \hline
    0 & 0  & 1 & 2 & 3 \\ \hline
    1 & 1  & 2 & 3 & 0 \\ \hline
    2 & 2  & 3 & 0 & 1 \\ \hline
    3 & 3  & 0 & 1 & 2 \\ \hline
    \end{tabular}
    \quad
    \begin{tabular}{|C|C|C|C|C|}
        \hline
    \cdot & 0  & 1 & 2 & 3 \\ \hline
        0 & 0  & 0 & 0 & 0 \\ \hline
        1 & 0  & 1 & 2 & 3 \\ \hline
        2 & 0  & 2 & 0 & 2 \\ \hline
        3 & 0  & 3 & 2 & 1 \\ \hline
        \end{tabular}
    \caption{Addition- und Multiplikationstafel für den Restklassenring $\mathbb{Z}_4$} \label{table:tableZ4}
\end{table}

\begin{table}[]
    \centering
    \begin{tabular}{|C|C|C|C|C|C|}
    \hline
    + & 0  & 1 & 2 & 3 & 4 \\ \hline
    0 & 0  & 1 & 2 & 3 & 4 \\ \hline
    1 & 1  & 2 & 3 & 4 & 0 \\ \hline
    2 & 2  & 3 & 4 & 0 & 1 \\ \hline
    3 & 3  & 4 & 0 & 1 & 2 \\ \hline
    4 & 4  & 0 & 1 & 2 & 3 \\ \hline
    \end{tabular}
    \quad
    \begin{tabular}{|C|C|C|C|C|C|}
        \hline
    \cdot & 0  & 1 & 2 & 3 & 4 \\ \hline
        0 & 0  & 0 & 0 & 0 & 0 \\ \hline
        1 & 0  & 1 & 2 & 3 & 4 \\ \hline
        2 & 0  & 2 & 4 & 1 & 3 \\ \hline
        3 & 0  & 3 & 1 & 4 & 2 \\ \hline
        4 & 0  & 4 & 2 & 2 & 1 \\ \hline
        \end{tabular}
    \caption{Addition- und Multiplikationstafel für den Restklassenring $\mathbb{Z}_5$} \label{table:tableZ5}
\end{table}

Wie der folgende Satz zeigen wird, bildet der Restklassenring $\mathbb{Z}_n$ unter einer bestimmten Vorraussetzung einen Körper.

\begin{satz}
    $\mathbb{Z}_n$ ist genau dann ein Körper, wenn $n$ eine Primzahl ist. 
\end{satz}

In diesem Fall schreiben wir $\field{p} := \mathbb{Z}_p$ für eine Primzahl $p$. Von den beiden Beispielen $\mathbb{Z}_4$ und $\mathbb{Z}_5$ ist somit nur der zweite ein Körper, da die $5$ Primzahl ist.

\subsection{Polynomringe}

Die über Restklassenringen konstruierten Körper besitzen alle die Mächtigkeit eine Primzahl. Darauf aufbauend können mithilfe von Polynomen weitere endliche Körper gebildet werden. 
Analog zu den ganzen Zahlen wird dafür zunächst ein Ring erzeugt.

\begin{satz}
    Sei $\mathbb{F}$ Körper. Die Menge $\mathbb{F} {[X]}$ der Polynome mit Koeffizienten in $\mathbb{F}$ zusammen mit der von Polynomen bekannten Addition und Multiplikation ist ein Ring. 
\end{satz}

Auch für diese Menge lässt sich eine Modulodivison mit Restabbildung definieren. 

\begin{satz}
    Sei $n \in \mathbb{F} {[X]}$ mit $N = \mathrm{grad}(n)$. Dann existieren für jedes $a \in \mathbb{F} {[X]}$ eindeutig bestimmte Polynome $f ,r \in \mathbb{F} {[X]}$ mit $a = n \cdot f + r$ und $\mathrm{grad}(r) < N$.
\end{satz}

%Vllt Beispiel für Polynomdivision

Das Polynom $r$ wird ebenfalls als der Rest von $a$ modulo $n$ bezeichnet. Mithilfe der Restabbildung $\varrho_n : \mathbb{F} {[X]} \rightarrow \field{n} {[X]}$, welche jedem Polynom ihren Rest zuordnet, lässt sich wieder ein Ring erzeugen.

\begin{satz}
    Sei $n \in \mathbb{F} {[X]}$. Die Menge $\field{n} {[X]}$ zusammen mit der Multiplikation $a \otimes b \mapsto \varrho_n(a \cdot b)$ und der Addition $a \oplus b \mapsto \varrho_n(a + b)$ bildet einen Ring. Dieser wird als Polynomring modulo $n$ bezeichnet.
\end{satz}

Sei zum Beispiel $\mathbb{F} = \field{2}$ und $n$ ein Polynom vom Grad 2. Dann umfasst $\field{n} {[X]}$ alle Polynome vom Grad kleiner als 2. \Cref{table:tableF41,table:tableF42} zeigen die Addition- und Multiplikationstafeln für $n=x^2 + x$ und $n= x^2 + x + 1$.

\begin{table}[]
    \centering
    \begin{tabular}{|C|C|C|C|C|}
    \hline
    +   & 0     & 1     & x     & x + 1 \\ \hline
    0   & 0     & 1     & x     & x+1 \\ \hline
    1   & 1     & 0     & x+1   & x \\ \hline
    x   & x     & x+1   & 0     & 1 \\ \hline
    x+1 & x+1   & x     & 1     & 0 \\ \hline
    \end{tabular}
    \quad
    \begin{tabular}{|C|C|C|C|C|}
        \hline
    \cdot   & 0 & 1     & x     & x + 1 \\ \hline
        0   & 0 & 0     & 0     & 0 \\ \hline
        1   & 0 & 1     & x     & x+1 \\ \hline
        x   & 0 & x     & x     & 0 \\ \hline
        x+1 & 0 & x+1   & 0     & x+1 \\ \hline
        \end{tabular}
    \caption{Addition- und Multiplikationstafel für den Polynomring $\field{n} {[X]}$ für $n=x^2 + x$} \label{table:tableF41}
\end{table}

\begin{table}[]
    \centering
    \begin{tabular}{|C|C|C|C|C|}
    \hline
    +   & 0     & 1     & x     & x + 1 \\ \hline
    0   & 0     & 1     & x     & x+1 \\ \hline
    1   & 1     & 0     & x+1   & x \\ \hline
    x   & x     & x+1   & 0     & 1 \\ \hline
    x+1 & x+1   & x     & 1     & 0 \\ \hline
    \end{tabular}
    \begin{tabular}{|C|C|C|C|C|}
        \hline
    \cdot   & 0 & 1     & x     & x + 1 \\ \hline
        0   & 0 & 0     & 0     & 0 \\ \hline
        1   & 0 & 1     & x     & x+1 \\ \hline
        x   & 0 & x     & x + 1& 1 \\ \hline
        x+1 & 0 & x+1   & 1     & x \\ \hline
        \end{tabular}
    \caption{Addition- und Multiplikationstafel für den Polynomring $\field{n} {[X]}$ für $n=x^2 + x + 1$} \label{table:tableF42}
\end{table}

Analog können wieder einige Ringaxiome verifiziert werden. Die konstanten Polynome $0$ und $1$ sind die neutralen Elemente. Offensicht existieren nicht immer multiplikativ inverse Element, sodass auch der $\field{n} {[X]}$ im Allgemeinen kein Körper ist. Die Eigenschaft, welche das Polynom $n$ erfüllen muss, damit der $\field{n} {[X]}$ ein Körper ist, wird im folgenden Abschnitt betrachtet.

\subsection{Endliche Körper}

\begin{comment}
\begin{enumerate}
    \item Abschnitt Einfache endliche Körper:
    \item Definition: Restklassenring $\mathbb{Z}_n$
    \item Satz: $\mathbb{Z}_n$ ist Ring
    \item Satz: $\mathbb{Z}_p$ ist Körper für $p$ Primzahl
    \item (Satz von Fermat)    
    
    \item Abschnitt Polynomringe:
    \item Definition: Polynom
    \item Satz: Division mit Rest existiert (2.6)
    \item Satz: Polynomringe (${\field{p}}[X]$ und ${\field{p}}_{N} [X]$) sind Ringe
\end{enumerate}
\end{comment}    
\begin{enumerate}    
    \item Abschnitt Endliche Körper:
    \item Teiler und Irreduzible Polynome
    \item Satz: ${\field{p}}_{N} [X]$ ist Körper für N irreduzibel (3.7)
    \item Satz: Existenzsatz (10.4)
    \item Satz: Eindeutigkeitssatz (10.9)
    
    \item Abschnitt zyklische Gruppe:
    \item Definition: zyklische Gruppe (6.4)
    \item Satz: Zyklische Gruppe besitzt primitives Element
    \item Potenzgesetze
    \item Satz: Multiplikative Gruppe ${\field{p}}^{\ast}$ und ihre Untergruppen sind zyklisch
    \item (Potenzen bilden Untergruppe (6.4))
    \item Definition: Ordnung eines Elements (/ einer Gruppe)
    \item Satz: Potenzen sind zyklisch (6.8)
    \item Satz: Elemente bilden eigene Untergruppe mit Ordnung als Teiler (6.9)
\end{enumerate}

\todo{Ungenauigkeiten zu F* anpassen}

\section{Indextransformationen}

\begin{lemma} \label{lemma:det-polynom}
    $\det M_{IJ} \in \field{p}[n][X]$.
\end{lemma}

\begin{proof}
    Wir beweisen diesen Satz mittels Induktion über die Größe der Submatrizen. 

    Induktionsanfang: $|I| = |J| = 1$
    \begin{equation*}
        \det M_{IJ} = x^{i_1j_1} \in \field{p}[n][X]
    \end{equation*}

    Induktionsschritt: $|I| = |J| = n+1$
    \begin{equation*}
        \det M_{IJ} = \sum_{k=1}^{n+1} \underbrace{(-1)^{k+l} x^{i_kj_l}}_{\in \field{p}[n][X]} \underbrace{\det M_{I\backslash i_k J \backslash j_l}}_{\mathrm{IV:}\; \in \field{p}[n][X]}
    \end{equation*}
    Aufgrund der Abgeschlossenheit von Polynomen bezüglich Addition und Multiplikation ist $\det M_{IJ}$ ebenfalls in $\field{p}[n][X]$.
\end{proof}

\begin{lemma} \label{lemma:transpose}
    \begin{equation*}
        \det M_{IJ} = \det M_{JI}
    \end{equation*}
\end{lemma}

\begin{proof}
    \begin{align*}
        \det M_{IJ} &= \det (x^{ij})_{\substack{i\in I \\ j\in J}} \\
                    &= \det \left( (x^{ji})_{\substack{j\in J \\ i\in I}} \right)^T \\
                    &= \det \left( M_{JI} \right)^T \\
                    &= \det M_{JI}
    \end{align*}
\end{proof}

\subsection{Lineare Indextransformationen}

\begin{lemma} \label{lemma:translation}
    \begin{equation*}
        \det{} M_{I+m,J} = x^{m(j_1+\cdots +j_k)} \det M_{I,J}
    \end{equation*}
\end{lemma}

\begin{proof}
    Die Einträge der Matrix $M_{I+m,J}$ haben die Form 
    \begin{equation*}
        x^{(i+m)j} = x^{ij} \cdot x^{mj}
    \end{equation*} für $i\in I, j\in J$.

    Jeder Eintrag einer Spalte $j \in J$ enthält den Faktor $x^{mj}$. Aus der Multilinearität der Determinante folgt
    \begin{align*}
        \det M_{I+m,J}  &= \det (x^{ij} \cdot x^{mj}) \\ 
                        &= \prod_{j \in J} x^{mj} \det (x^{ij}) \\ 
                        &= \prod_{j \in J} x^{mj}  \det M_{I,J} \\
                        &= x^{m(j_1+\cdots +j_k)} \det M_{I,J}
    \end{align*}
\end{proof}

\begin{satz}
    \begin{equation*}
        \det{} M_{I+m,J+n} = x^{mnk} x^{m(j_1 +\cdots + j_k)} x^{n(i_1+\cdots +i_k)} \det{} M_{I,J}
    \end{equation*}
\end{satz}

\begin{proof}
    Aus \cref{lemma:transpose,lemma:translation} folgt direkt
    \begin{align*}
        \det{} M_{I+m,J+n}  &= x^{m(j_1 + n +\cdots + j_k + n)} \det{} M_{I,J+n} \\
                            &= x^{m(j_1 +\cdots + j_k + kn)} \det{} M_{I,J+n} \\
                            &= x^{mnk} x^{m(j_1 +\cdots + j_k)} \det{} M_{I,J+n} \\
                            &= x^{mnk} x^{m(j_1 +\cdots + j_k)} \det{} M_{J+n,I} \\
                            &= x^{mnk} x^{m(j_1 +\cdots + j_k)} x^{n(i_1+\cdots +i_k)} \det{} M_{J,I} \\
                            &= x^{mnk} x^{m(j_1 +\cdots + j_k)} x^{n(i_1+\cdots +i_k)} \det{} M_{I,J} \\
    \end{align*}
\end{proof}

\begin{satz}
    \begin{equation*}
        \left( \det{} M_{aI,bJ} \right) (x) = \left( \det{} M_{I,J} \right) (x^{ab})
    \end{equation*}
\end{satz}

\begin{proof}
    Wir beweisen diesen Satz mittels Induktion über die Größe der Submatrizen. 

    Induktionsanfang: $|I| = |J| = 1$
    \begin{equation*}
        \left( \det{} M_{aI,bJ} \right) (x) = x^{ai_1bj_1} = {x^{ab}}^{i_1j_1} = \left( \det{} M_{I,J} \right) (x^{ab})
    \end{equation*}

    Induktionsschritt: $|I| = |J| = n+1$
    \begin{align*}
        \left( \det{} M_{aI,bJ} \right) (x) &= \sum_{k=1}^{n+1} (-1)^{k+l} x^{ai_kbj_l} \left( \det M_{aI\backslash ai_k, bJ \backslash bj_l} \right) (x) \\
                                            &= \sum_{k=1}^{n+1} (-1)^{k+l} {x^{ab}}^{i_kj_l} \left( \det M_{I\backslash i_k, J \backslash j_l} \right) (x^{ab}) \\
                                            &= \left( \det{} M_{I,J} \right) (x^{ab})
    \end{align*}
\end{proof}

\todo{Nullstellen fürs Inverse Element automatisch bekannt}

\begin{korollar}
    \begin{equation*}
        \left( \det{} M_{aI + m,bJ + n} \right) (x) = x^{mnk} x^{ma(j_1 +\cdots + j_k)} x^{nb(i_1+\cdots +i_k)} \left( \det{} M_{I,J} \right) (x^{ab})
    \end{equation*}
\end{korollar}

\begin{proof}
    \begin{align*}
        \left( \det{} M_{aI + m,bJ + n} \right) (x) &= x^{mnk} x^{m(bj_1 +\cdots + bj_k)} x^{n(ai_1+\cdots +ai_k)} \left( \det{} M_{aI,bJ} \right) (x) \\
                                                    &= x^{mnk} x^{mb(j_1 +\cdots + j_k)} x^{na(i_1+\cdots +i_k)} \left( \det{} M_{aI,bJ} \right) (x) \\
                                                    &= x^{mnk} x^{mb(j_1 +\cdots + j_k)} x^{na(i_1+\cdots +i_k)} \left( \det{} M_{I,J} \right) (x^{ab}) \\
    \end{align*}
\end{proof}

\include{sections/SinguläreSubmatrizen}

\section{Spezielle Indexmengen} \label{sec:spezielleIndexmengen}

Im vorherigen Abschnitten wurden hinreichende Bedingungen zur Invertierbarkeit vorgestellt, die sich grundlegend auf die Struktur der Indexmengen und auf die Ordnungen der Körperelemente bezogen. In diesem Kapitel werden weitere quadratische Submatrizen untersucht, deren Invertierbarkeit sich nicht über die gezeigten Sätze herleiten lässt.

Zuerst werden Indexmengen mit zusammenhängenden Indizes betrachtet. Sei ohne Beschränkung der Allgemeinheit $I = \{0,\dots,k-1\}$ und $J$ beliebig. Die Matrix $M_{IJ}$ entspricht dann der Vandermondematrix $V_{(x^{j_1},\dots,x^{j_k})}$, deren Determinante bekannt ist. \cite{VandermondeDet}
\begin{equation*}
    \det V_{(x^{j_1},\dots,x^{j_k})} = \prod_{1\leq v<w\leq k} (x^{j_v} - x^{j_w})
\end{equation*}


Damit $(\det M_{IJ})(a) = 0$ gilt, müssen wieder zwei Indizes existieren, deren Differenz ein Vielfaches der Ordnung von $a$ entspricht. Dies ist unter anderem der Fall, wenn die Mächtigkeit der Indexmengen größer als die Ordnung ist.

\begin{sloppypar}
    Als Nächstes werden spezielle Indexmengen mit $|I| = |J| = 3$ untersucht. Seien ${I = \{0, 1, 3\}}$ und $J = \{0, \alpha, \beta\}$ für beliebige $\alpha,\beta \in \mathbb{Z}$. Dann besitzt $M_{IJ}$ diese Form.
\end{sloppypar}
\begin{equation*}
    \begin{pmatrix}
        1 & 1 & 1 \\
        1 & x^{\alpha} & x^{\beta} \\
        1 & x^{3\alpha} & x^{3\beta}
    \end{pmatrix}
\end{equation*}

Mithilfe von Zeilenumformungen und anschließender Spaltenentwicklung ergibt sich die Determinante wie folgt.

\begin{align*}
    \det \begin{pmatrix}
        1 & 1 & 1 \\
        1 & x^{\alpha} & x^{\beta} \\
        1 & x^{3\alpha} & x^{3\beta}
    \end{pmatrix}
    &= \det \begin{pmatrix}
        1 & 1 & 1 \\
        0 & x^{\alpha} -1 & x^{\beta} -1 \\
        0 & x^{3\alpha} -1 & x^{3\beta} -1
    \end{pmatrix} \\
    &= \det \begin{pmatrix}
        x^{\alpha} -1 & x^{\beta} -1 \\
        x^{3\alpha} -1 & x^{3\beta} -1
    \end{pmatrix} \\
    &= (x^{\alpha} -1)(x^{3\beta} -1) - (x^{3\alpha} -1)(x^{\beta} -1)
\end{align*}

Für ein $a \in \field{p}[n]^\times$ ist die Submatrix $M_{IJ}$ somit genau dann nicht invertierbar, falls $a$ die Gleichung $(x^{\alpha} -1)(x^{3\beta} -1) = (x^{3\alpha} -1)(x^{\beta} -1)$ löst. Diese Bedingung reduziert sich weiter zu:

\begin{align}
        & (x^{\alpha} -1)(x^{3\beta} -1) = (x^{3\alpha} -1)(x^{\beta} -1) \nonumber \\
    \iff & \frac{(x^{3\beta} -1)}{(x^{\beta} -1)} = \frac{(x^{3\alpha} -1)}{(x^{\alpha} -1)} \nonumber \\
    \iff & {x^\beta}^2 + x^{\beta} + 1 = {x^{\alpha}}^2 + x^{\alpha} + 1 \nonumber \\
    \iff & {x^\beta}^2 + x^{\beta} = {x^{\alpha}}^2 + x^{\alpha} \label{equation:013}
\end{align}

Anders formuliert sind nun zwei Elemente $v_1 = a^\alpha$ und $v_2 = a^\beta$ gesucht, die $x^2 + x = m$ für ein festes $m \in \field{p}[n]$ lösen. Abhängig von der Charakteristik des Körpers existieren zwei unterschiedliche Lösungsformeln.

\begin{lemma}[{\cite[S. 4]{QuadEquCharNot2}}]
    Sei $\charac(\field{p}[n]) \neq 2$. Die quadratische Gleichung $x^2 + x = m$ ist genau dann lösbar, wenn die Wurzel $\sqrt{2^{-2} + m}$ existiert. Die Lösungen lauten dann
    \begin{equation*}
        v_{1,2} = -2^{-1} \pm \sqrt{2^{-2} + m}.
    \end{equation*}
\end{lemma}

\begin{lemma}[{\cite[S. 140]{QuadEquChar2}}]
    Sei $\charac(\field{p}[n]) = 2$. Die quadratische Gleichung $x^2 + x = m$ ist genau dann lösbar, wenn $\Tr(m) = 0$ gilt. Die erste Lösung lautet dann
    \begin{equation*}
        v_1 = 1 + \sum_{k=1}^{n-1} m^{2^k}(\sum_{l=0}^{k-1} u^{2^l})
    \end{equation*}
     für ein beliebiges $u \in \field{p}[n]$ mit $\Tr(u) = 1$. Die zweite Lösung ist $v_2 = v_1 + 1$.
\end{lemma}

Sofern die Lösungen $v_1$ und $v_2$ existieren, liefert ein gemeinsamer Erzeuger $a$ die gesuchten Exponenten $\alpha$ und $\beta$. Da die Potenzen $v_1 = a^\alpha$ und $v_2 = a^\beta$ \Cref{equation:013} lösen, ist $M_{IJ}$ mit $J = \{0,1,3\}$ und $J = \{0,\alpha,\beta\}$ für dieses $a$ nicht invertierbar.

\sloppy Im Folgenden wird ein Beispiel für $\field{11}$ betrachtet. In diesem Körper gelten ${-2^{-1} = -6 = 5}$ und $2^{-2} = 3$. \Cref{table:sol_013F11} gibt für jedes $m$ die Wurzel $\sqrt{3 + m}$ an. Im Allgemeinen existieren diese, falls $p \neq 2$ gilt, für $\frac{p^n-1}{2} + 1$ Körperelemente \cite{RootsFiniteFields}. In diesem Beispiel somit für sechs der elf Elemente. Die dritte Zeile enthält die sich daraus ergebenden Lösungen $v_{1,2} = 5 \pm \sqrt{3+m}$.
In \Cref{table:subgroupsF11} sind die Untergruppen von $\field{11}$ aufgelistet. Für die Wahl von $m=1$ ergeben sich zum Beispiel $v_1 = 3$ und $v_2 = 7$. Ein möglicher Erzeuger ist $a = 8$ mit $v_1 = 8^6$ und $v_2 = 8^9$. Somit ist die \Cref{equation:013} erfüllt und $(\det M_{IJ})(8) = 0$ für $I = \{0,1,3\}$ und $J = \{0,6,9\}$.

\begin{align*}
    {8^6}^2 + 8^6 &= {8^{9}}^2 + 8^9 \\
    3^2 + 3 &= 7^2 + 7 \\
    9 + 3 &= 5 + 7
\end{align*}

Die Fälle, bei denen $v_1$ oder $v_2$ gleich 0 sind, müssen hier nicht weiter betrachtet werden, da diese Werte nicht in der selben Untergruppe auftauchen können. Falls $v_1$ oder $v_2$ gleich $1$ sind, muss einer der Exponenten $\alpha$ oder $\beta$ gleich $0$ sein. Weil dieser Exponent nicht doppelt in der Indexmenge $J$ vorkommen soll, ist dieser Fall ebenfalls nicht relevant. Gleiches gilt für $v_1 = v_2$. Alle zulässigen Lösungen $v_1$ und $v_2$ führen zu den folgenden Indexmengen.

\begin{align*}
    &a = 6:\quad J=\{0,2,3\}, J=\{0,7,9\} \text{ und } J=\{0,1,8\} \\
    &a = 7:\quad J=\{0,1,4\}, J=\{0,3,9\} \text{ und } J=\{0,6,7\} \\
    &a = 8:\quad J=\{0,6,9\}, J=\{0,1,7\} \text{ und } J=\{0,3,4\}     
\end{align*}

{\renewcommand{\arraystretch}{1.5}
\begin{table}
    \centering
    \begin{tabular}{|C|C|C|C|C|C|C|C|C|C|C|C|C}
    \hline
    m          & 0    & 1   & 2   & 3 & 4 & 5 & 6   & 7  & 8   & 9   & 10 \\
    \hline
    \sqrt{3+m} & 5    & 2   & 4   & - & - & - & 3   & -  & 0   & 1   & -  \\
    \hline
    v_{1,2} = 5 \pm \sqrt{3+m}   & 10,0 & 3,7 & 9,1 & - & - & - & 8,2 & -  & 5,5 & 6,4 & -  \\
    \hline
    \end{tabular}
    \caption{Lösungen $v_{1,2}$ zu $x^2 + x = m$ für festes $m \in \field{11}$.} \label{table:sol_013F11}
\end{table}
}

\begin{table}[]
    \centering
    \begin{tabular}{|C|C|C|C|C|C|C|C|C|C|}
    \hline
    \text{\backslashbox{i}{a}} & 2  & 3 & 4 & 5 & 6  & 7  & 8  & 9 & 10 \\ \hline
    0 & 1  & 1 & 1 & 1 & 1  & 1  & 1  & 1 & 1  \\ \hline
    1 & 2  & 3 & 4 & 5 & 6  & 7  & 8  & 9 & 10 \\ \hline
    2 & 4  & 9 & 5 & 3 & 3  & 5  & 9  & 4 & 1  \\ \hline
    3 & 8  & 5 & 9 & 4 & 7  & 2  & 6  & 3 &    \\ \hline
    4 & 5  & 4 & 3 & 9 & 9  & 3  & 4  & 5 &    \\ \hline
    5 & 10 & 1 & 1 & 1 & 10 & 10 & 10 & 1 &    \\ \hline
    6 & 9  &   &   &   & 5  & 4  & 3  &   &    \\ \hline
    7 & 7  &   &   &   & 8  & 6  & 2  &   &    \\ \hline
    8 & 3  &   &   &   & 4  & 9  & 5  &   &    \\ \hline
    9 & 6  &   &   &   & 2  & 8  & 7  &   &    \\ \hline
    10 & 1  &   &   &   & 1  & 1  & 1  &   &    \\ \hline
    \end{tabular}
    \caption{Alle von einem $a \in \field{11}$ erzeugten Untergruppen.} \label{table:subgroupsF11}
\end{table}

Ein weiteres Beispiel betrachtet $\field{2}[3]$. Analog zeigt \Cref{table:sol_013F2_3} für jedes $m$ mit $\Tr(m) = 0$ eine Lösung der quadratischen Gleichung $x^2+x=m$. Für $m=y^2$ ergibt sich zum Beispiel $v_1 = y^2 + y$ und $v_2 = y^2 + y + 1$. Wird der Erzeuger $a = y+1$ mit $v_1 = (y+1)^6$ und $v_2 = (y+1)^4$ gewählt, so wird wieder \Cref{equation:013} erfüllt.
\begin{align*}
    {(y+1)^6}^2 + (y+1)^6 &= {(y+1)^4}^2 + (y+1)^4 \\
    (y^2 + y)^2 + (y^2 + y) &= (y^2 + y + 1)^2 + (y^2 + y + 1) \\
    y + (y^2 + y) &= (y+1) + (y^2 + y + 1) \\
    y^2 &= y^2 \\
\end{align*}
Daraus folgt $(\det M_{IJ})(y+1) = 0$ für $I = \{0,1,3\}$ und $J = \{0,4,6\}$.
Da die Gruppenordnung von $\field{2}[3]$ eine Primzahl ist, ist jedes Körperelement $a \notin \{0,1\}$ ein Erzeuger. Somit existiert für jedes dieser Elemente eine Indexmenge $J$, sodass $(\det M_{IJ})(a) = 0$ gilt.

{\renewcommand{\arraystretch}{1.5}
\begin{table}
    \centering
    \begin{tabular}{|C|C|C|C|C|C|C|C|C|}
    \hline
    m               & 0 & 1 & y & y + 1 & y^2 & y^2 + 1 & y^2 + y  & y^2 + y + 1 \\
    \hline
    \mathrm{Tr}(m)  & 0 & 1 & 0 & 1     & 0   & 1       & 0        & 1 \\
    \hline
    v_1         & 0 & - & y^2 & - & y^2 + y & - & y & - \\
    \hline
    \end{tabular}
    \caption{Lösungen $v_1$ zu $x^2 + x = m$ für festes $m \in \field{2}[3]$.} \label{table:sol_013F2_3}
\end{table}
}

\begin{table}
    \centering
    \begin{tabular}{|C|C|C|C|C|C|C|C|C|C|C|}
    \hline
    \text{\backslashbox{i}{a}}& y       & y+1     & y^2     & y^2 + 1 & y^2 + y & y^2 + y + 1 \\ \hline
    0 & 1       & 1       & 1       & 1       & 1       & 1           \\ \hline
    1 & y       & y+1     & y^2     & y^2+1   & y^2+y   & y^2+y+1     \\ \hline
    2 & y^2     & y^2+1   & y^2+y   & y^2+y+1 & y       & y+1         \\ \hline
    3 & y+1     & y^2     & y^2+1   & y^2+y   & y^2+y+1 & y           \\ \hline
    4 & y^2+y   & y^2+y+1 & y       & y+1     & y^2     & y^2+1       \\ \hline
    5 & y^2+y+1 & y       & y+1     & y^2     & y^2+1   & y^2+y       \\ \hline
    6 & y^2+1   & y^2+y   & y^2+y+1 & y       & y+1     & y^2         \\ \hline
    7 & 1       & 1       & 1       & 1       & 1       & 1           \\ \hline
    \end{tabular}
    \caption{Alle von einem $a \in \field{2}[3]$ erzeugten Untergruppen.} \label{table:subgroupsF2_3}
\end{table}

\bibliographystyle{plain}
\bibliography{ref}

\listoftodos

\end{document}