\documentclass{article}

\usepackage{xparse}
\usepackage{slashbox}
\usepackage[ngerman]{babel}
\usepackage{amsthm}
\usepackage{amsfonts}
\usepackage{amsmath}
\usepackage{array}
\usepackage{mathtools}
\usepackage[unicode=true,
 bookmarks=true,bookmarksnumbered=false,bookmarksopen=false,
 breaklinks=false,pdfborder={0 0 1},backref=false,colorlinks=false]
 {hyperref}
\usepackage[noabbrev,nameinlink, ngerman]{cleveref}

\newtheorem{satz}{Satz}
\newtheorem{lemma}[satz]{Lemma}
\newtheorem{korollar}[satz]{Korollar}
\theoremstyle{definition}
\newtheorem{definition}{Definition}

\renewcommand{\det}{\mathrm{det}\;}
\newcommand{\Modulo}{\mathbin\mathrm{mod}}
\NewDocumentCommand{\field}{m o}{%
  \mathbb{F}_{#1\IfValueT{#2}{^{#2}}}%
}

\begin{document}

\section{Definitionen}

\begin{definition}
    Seien $I = \{i_1,\dots,i_k\},J = \{i_1,\dots,i_k\} \subseteq \mathbb{Z}$ Dann ist $M_{IJ} := (x^{ij})_{i \in I, j \in J}$. 
\end{definition}

\section{Lineare Indexabbildungen}

Die Submatrizen einer Matrix $M$ werden über zwei gleich mächtige Indexmengen $I,J$ indiziert. Diese Submatrix wird als $M_{IJ}$ bezeichnet. 
In diesem Kapitel werden lineare Abbildungen auf die Indexmengen $I,J$ angewendet und der Zusammengang zum Minor $\det M_{IJ}$ untersucht.

\todo{Sortierung von I irrelevant, da Determinante alternierend ist}

Die Matrix $M$ ist weiterhin von der Form $\left( x^{ij} \right)_{0\leq i,j \leq k-1}$, wobei für $x$ ausschließlich Elemente eines endlichen Körpers $\field{p}[n]$ substituiert werden. Ein Eintrag $x^{ij}$ kann somit als Monom aus dem Polynomring $\field{p}[n][X]$ verstanden werden. Aufgrund der Abgeschlossenheit des Polynomrings ist der Minor $\det M_{IJ}$ ebenfalls ein Polynom. Da $M$ und ihre Submatrizen wiederum als Submatrizen der unendlich größen Matrix $\left( x^{ij} \right)_{i,j \in \mathbb{Z}}$ aufgefasst werden können, sei im Folgenden $M$ stets eben diese Matrix. Dies vereinfacht einige Aussagen und Erläuterungen in diesem Kapitel und ermöglicht allgemeinere Indexmengen $I,J \subseteq \mathbb{Z}$ zu betrachten.

Für zum Beispiel $I,J = \{-2, \dots, 4\}$ sieht $M_{IJ}$ folgendermaßen aus.
\begin{equation*}
    M_{IJ} = \begin{pmatrix}
        x^4     & x^{2} & 1    & x^{-2} & x^{-4}& x^{-6}& x^{-8}\\
        x^{2}   & x^{1} & 1    & x^{-1} & x^{-2}& x^{-3}& x^{-4}\\
        1       & 1     & 1    & 1      & 1     & 1     & 1     \\
        x^{-2}  & x^{-1}& 1    & x^{1}  & x^{2} & x^{3} & x^{4} \\
        x^{-4}  & x^{-2}& 1    & x^{2}  & x^{4} & x^{6} & x^{8} \\
        x^{-6}  & x^{-3}& 1    & x^{3}  & x^{6} & x^{9} & x^{12} \\
        x^{-8}  & x^{-4}& 1    & x^{4}  & x^{8} & x^{12} & x^{16} \\
    \end{pmatrix}
\end{equation*}

Falls der $\field{3}$ als endlicher Körper gewählt wird, besitzt $M_{IJ}$ aufgrund der zyklischen Eigenschaften des Körpers wiederum folgende Form.

\begin{equation*}
    M_{IJ} = \begin{pmatrix}
        x^1     & x^{2} & 1     & x^{1} & x^{2} & 1 & x^{1} \\
        x^{2}   & x^{1} & 1     & x^{2} & x^{1} & 1 & x^{2} \\
        1       & 1     & 1     & 1     & 1     & 1 & 1     \\
        x^{1}   & x^{2} & 1     & x^{1} & x^{2} & 1 & x^{1} \\
        x^{2}   & x^{1} & 1     & x^{2} & x^{1} & 1 & x^{2} \\
        1       & 1     & 1     & 1     & 1     & 1 & 1     \\
        x^1     & x^2   & 1     & x^1   & x^2   & 1 & x^1   \\
    \end{pmatrix}
\end{equation*}

Durch das Reduzieren der Exponenten um die entsprechende Gruppenordnung, erhält die unendlich große Matrix $M$ die Form einer Blockmatrix, die nur aus den Blöcken

\begin{equation*}
    \begin{pmatrix}
        1     & 1     & 1     \\
        1     & x^{1} & x^{2} \\
        1     & x^{2} & x^{1} \\
    \end{pmatrix}
\end{equation*}

besteht. Für allgemeine $\field{p}[n]$ haben die Blöcke diese Form.

\begin{equation*}
    \begin{pmatrix}
        1     & 1       & 1         &\cdots & 1 \\
        1     & a^1     & a^2       &\cdots & a^{p^n-1} \\
        1     & a^2     & a^4       &\cdots & a^{2p^n-2} \\
        \vdots&\vdots   &\vdots     &\ddots &\vdots \\
        1     &a^{p^n-1}&a^{2p^n-2} &\cdots &a^{(p^n-1)(p^n-1)}
    \end{pmatrix}
\end{equation*}

Da bei symmetrische Matrizen die $i$-te Zeile der $i$-ten Spalte entspricht, gilt $M_{IJ} = M_{JI}^T$ für beliebige Indexmengen $I,J$. Für gleich mächtige Indexmengen folgt direkt die erste nützliche Eigenschaft $\det M_{IJ} = \det M_{JI}$, welche in diesem Kapitel häufig genutzt wird.

Zuerst wird das Addieren einer Konstanten auf die Indexmengen untersucht, das dem Verschieben der Submatrix entspricht. Für eine Menge $I \subseteq \mathbb{Z}$ und ein $m \in \mathbb{Z}$ sei $I+m := \{i+m:i\in I\}$. Die Minoren von $M_{IJ}$ und $M_{I+m,J+n}$ hängen wie folgt zusammen.

\begin{satz} \label{satz:translation}
    Sei $M = \left( x^{ij} \right)_{i,j \in \mathbb{Z}}$. Seien weiter $I,J \subseteq \mathbb{Z}$ mit $|I|=|J|=k$ und $m,n \in \mathbb{Z}$. Dann gilt
    \begin{equation*}
        \det{} M_{I+m,J+n} = x^{mnk} x^{m(j_1 +\cdots + j_k)} x^{n(i_1+\cdots +i_k)} \det{} M_{IJ}
    \end{equation*}
\end{satz}

\begin{proof}
    Die Einträge der Matrix $M_{I+m,J+n}$ haben die Form 
    \begin{equation*}
        x^{(i+m)(j+n)} = x^{ij} \cdot x^{mj} \cdot x^{ni} \cdot x^{mn} 
    \end{equation*} für $i\in I, j\in J$. Mithilfe der Multilinearität der Determinante können für feste Zeilen- und Spaltenindizes konstante Terme herausgezogen werden.
    \begin{align*}
        \det M_{I+m,J+n}    &= \det \left(x^{ij} \cdot x^{mj} \cdot x^{ni} \cdot x^{mn} \right)_{i\in I,j\in J} \\ 
                            &= \left( \prod_{i \in I} x^{ni} \cdot x^{mn}\right) \det \left(x^{ij} \cdot x^{mj} \right)_{i\in I,j\in J} \\ 
                            &= \left( \prod_{i \in I} x^{ni} \cdot x^{mn}\right) \left( \prod_{j \in J} x^{mj}\right) \det \left(x^{ij}\right)_{i\in I,j\in J} \\ 
                            &= x^{kmn} \left( \prod_{i \in I} x^{ni}\right) \left( \prod_{j \in J} x^{mj}\right) \det M_{IJ} \\ 
                            &= x^{mnk} x^{n(i_1 +\cdots +i_k)} x^{m(j_1+\cdots +j_k)} \det M_{IJ}.
    \end{align*}
\end{proof}

Die beiden Minoren besitzen somit die gleichen Nullstellen. Ist andersherum betrachtet eine Nullstelle $a$ von $\det M_{I,J}$ bekannt, ist $a$ ebenfalls Nullstelle von $\det M_{I+m,J+n}$ für alle $m,n \in \mathbb{Z}$. Der Satz zeigt zudem, dass die Nullstellen ausschließlich von den Abständen der Indizes in $I$ und $J$ abhängen und nicht von einer Verschiebung innerhalb von $M$. Somit kann es sich anbieten nur normalisierte Indexmengen zu betrachten, für die $\min(I) = \min(J) = 0$ gelten.

Als nächstes wird die Multiplikation der Indexmengen mit einer Konstanten untersucht. Für eine Menge $I \subseteq \mathbb{Z}$ und ein $a \in \mathbb{Z}$ sei $aI := \{ai:i\in I\}$. Die Minoren von $M_{IJ}$ und $M_{aI,bJ}$ hängen wie folgt zusammen.

\begin{satz} \label{satz:skalierung}
    Sei $M = \left( x^{ij} \right)_{i,j \in \mathbb{Z}}$. Seien weiter $I,J \subseteq \mathbb{Z}$ mit $|I|=|J|=k$ und $a,b \in \mathbb{Z}\setminus\{0\}$. Dann gilt
    \begin{equation*}
        \left( \det{} M_{aI,bJ} \right) (x) = \left( \det{} M_{IJ} \right) (x^{ab})
    \end{equation*}
\end{satz}

\begin{proof}    
    \begin{align*}
        M_{aI,bJ}   &= \left(x^{ai\cdot bj}\right)_{i\in I,j\in J} \\
                    &= \left((x^{ab})^{ij}\right)_{i\in I,j\in J}
    \end{align*}

    Die Matrix $M_{aI,bJ}$ weißt somit die selbe Struktur auf wie $M_{I,J}$. Anstelle der Variable $x$ besitzt $M_{aI,bJ}$ Variablen der Form $x^{ab}$ in ihren Einträgen. Durch Substitution von $x$ durch $x^{ab}$ lässt sich der Minor $\det M_{aI,bJ}$ aus dem Minor $\det M_{I,J}$ gewinnen.
\end{proof}

Aus dem gezeigten Satz folgt direkt, dass $\det{} M_{I,J}$ und $\det{} M_{-I,-J}$ die selben Nullstellen besitzen. Ist andersherum betrachtet eine Nullstelle $a$ von $\det M_{I,J}$ bekannt, ist $a$ ebenfalls Nullstelle von $\det M_{-I,-J}$ und $a^{-1}$ Nullstelle von $\det M_{-I,J}$ und $\det M_{I,-J}$.

Die \Cref{satz:translation,satz:skalierung} lassen sich im folgenden Korollar zusammenfügen.

\begin{korollar}
    Sei $M = \left( x^{ij} \right)_{i,j \in \mathbb{Z}}$. Seien weiter $I,J \subseteq \mathbb{Z}$ mit $|I|=|J|=k$ und $m,n,a,b \in \mathbb{Z}$. Dann gilt
    \begin{equation*}
        \left( \det{} M_{aI+m,bJ+n} \right) (x) = x^{mnk} x^{m(j_1 +\cdots + j_k)} x^{n(i_1+\cdots +i_k)} \left( \det{} M_{IJ} \right) (x^{ab})
    \end{equation*}
\end{korollar}

\begin{proof}
    \begin{align*}
        \left( \det{} M_{aI+m,bJ+n} \right) (x)     &= x^{mnk} x^{m(bj_1 +\cdots + bj_k)} x^{n(ai_1+\cdots +ai_k)} \left( \det{} M_{aI,bJ} \right) (x) \\
                                                    &= x^{mnk} x^{mb(j_1 +\cdots + j_k)} x^{na(i_1+\cdots +i_k)} \left( \det{} M_{aI,bJ} \right) (x) \\
                                                    &= x^{mnk} x^{mb(j_1 +\cdots + j_k)} x^{na(i_1+\cdots +i_k)} \left( \det{} M_{IJ} \right) (x^{ab}) \\
    \end{align*}
\end{proof}

Die Zusammenhänge zwischen transformierten Indexmengen können über Graphen dargestellt werden. Für festes $a \in \field{p}[n]$ sei $G_a$ der Graph mit Knotenmenge $V=\{I \in\mathcal{P}(\mathbb{Z}) \mid |I| \in \mathbb{N}_+\}$ und Kantenmenge $E = \{(I,J)\mid (\det M_{IJ})(a) = 0\}$. Der Graph $G_a$ enthält somit genau dann die Kante $(I,J)$, falls die Submatrix $M_{IJ}(a)$ nicht invertierbar ist. 

Aus \Cref{satz:translation} folgt, dass eine Kante $(I,J)$ alle weiteren Kanten $(I+m,J)$ für $m\in \mathbb{Z}$ induziert. Alle Knoten der Form $I+m$ besitzen somit in $G_a$ die selben adjazenten Knoten. Dieses Verhalten ist in \Cref{fig:subgraphs} dargestellt. Falls die Indexmengen $I$ und $J$ unterschiedlich sind, ergibt sich der vollständig bipartite Teilgraph aus \Cref{fig:biclique-graph}. Falls jedoch $I$ und $J$ gleich sind, entsteht der in \Cref{fig:complete-graph} dargestellte vollständige Teilgraph, in dem alle Knoten der Form $I+m$ miteinander verbunden sind. 

\begin{figure}[]
    \centering
    \begin{subfigure}[b]{0.47\textwidth}
        \centering
        \begin{tikzpicture}[node distance={5mm}, main/.style = {draw, circle, minimum size=1.2cm}]

            \node (D0) at (-0.5,-1.5) {\large $\cdots$};
            \node (D1) at (5,-1.5) {\large $\cdots$};

            % Nodes
            \node[main] (I0) at (0,0) {\small $I-1$};
            \node[main] (I1) at (1.5,0) {\small $I$};
            \node[main] (I2) at (3,0) {\small $I+1$};
            \node[main] (I3) at (4.5,0) {\small $I+2$};
            \node[main] (J0) at (0,-3) {\small $J-1$};
            \node[main] (J1) at (1.5,-3) {\small $J$};
            \node[main] (J2) at (3,-3) {\small $J+1$};
            \node[main] (J3) at (4.5,-3) {\small $J+2$};
    
            % Edges
            \draw (I0) to (J0);
            \draw (I0) to (J1);
            \draw (I0) to (J2);
            \draw (I0) to (J3);
            \draw (I1) to (J0);
            \draw (I1) to (J1);
            \draw (I1) to (J2);
            \draw (I1) to (J3);
            \draw (I2) to (J0);
            \draw (I2) to (J1);
            \draw (I2) to (J2);
            \draw (I2) to (J3);
            \draw (I3) to (J0);
            \draw (I3) to (J1);
            \draw (I3) to (J2);
            \draw (I3) to (J3);
        \end{tikzpicture}
        \caption{Vollständig bipartiter Teilgraph zu Knoten $I \neq J$ mit nicht invertierbarem $M_{IJ}$}
        \label{fig:biclique-graph}
    \end{subfigure}
    \hfill
    \begin{subfigure}[b]{0.47\textwidth}
        \centering
        \begin{tikzpicture}[node distance={5mm}, main/.style = {draw, circle, minimum size=1.2cm}]
            % Nodes
            \node[main] (I0) at ({360/6 * 0}: 2cm) {\small $I+2$};
            \node[main] (I1) at ({360/6 * 1}: 2cm) {\small $I+1$};
            \node[main] (I2) at ({360/6 * 2}: 2cm) {\small $I$};
            \node[main] (I3) at ({360/6 * 3}: 2cm) {\small $I-1$};
            \node (D1) at ({360/6 * 4}: 2cm) {\large $\cdots$};
            \node[main] (I4) at ({360/6 * 5}: 2cm) {\small $I+3$};

            \draw (I0) to[loop, out=22.5, in=-22.5, distance=1cm] (I0);         
            \draw (I0) to (I1);
            \draw (I0) to (I2);
            \draw (I0) to (I3);
            \draw (I0) to (I4);            
            \draw (I1) to[loop, out=82.5, in=37.5, distance=1cm] (I1);            
            \draw (I1) to (I2);
            \draw (I1) to (I3);
            \draw (I1) to (I4);
            \draw (I2) to[loop, out=142.5, in=97.5, distance=1cm] (I2);    
            \draw (I2) to (I3);
            \draw (I2) to (I4);
            \draw (I3) to[loop, out=202.5, in=157.5, distance=1cm] (I3);    
            \draw (I3) to (I4);
            \draw (I4) to[loop, out=322.5, in=277.5, distance=1cm] (I4);    
        \end{tikzpicture}
        \caption{Vollständiger Teilgraph zum Knoten $I$ mit nicht invertierbarem $M_{II}$}
        \label{fig:complete-graph}
    \end{subfigure}
    \caption{Bipartite Untergraphen in $G_a$ und $G_{a^{-1}}$}
    \label{fig:subgraphs}
\end{figure}

Aus \Cref{satz:skalierung} ergeben sich weitere Symmetrien in $G_a$. Demnach sind folgende Aussagen äquivalent:

\begin{align*}
    (\det{} M_{I,J})(a) &= 0 \\
    (\det{} M_{-I,-J})(a) &= 0 \\
    (\det{} M_{-I,J})(a^{-1}) &= 0 \\
    (\det{} M_{I,-J})(a^{-1}) &= 0
\end{align*}

Zu den in \Cref{fig:subgraphs} beschriebenen Teilgraphen aus $G_a$ sind die folgende Teilgraphen isomorph:

\begin{align*}
    &\text{Zu } -I \text{ und } -J \text{ in } G_a \\
    &\text{Zu } -I \text{ und } J \text{ in } G_{a^{-1}} \\
    &\text{Zu } I \text{ und } -J \text{ in } G_{a^{-1}} \\
\end{align*}

Somit induziert die bijektive Abbildung $f:V \to V, I \mapsto -I$ einen nicht trivialen Graphautomorphismus und einen Graphisomorphismus zwischen $G_a$ und $G_{a^{-1}}$.

\section{Singuläre Submatrizen}

Diese Kapitel befasst sich mit der Invertierbarkeit von Submatrizen für gegebene $a \in \field{p}[n]$. Dafür wollen wir im Folgenden hinreichende Bedingungen an die Indexmengen $I$ und $J$ stellen.

Die erste Aussage bezieht sich auf die lineare Abhängigkeit zweier Spalten/Zeilen. Für die vorrangehenden Bespiele sei stets $a = 2 \in \field{11}$ mit der Ordnung $\ord{a} = 10$. Seien zunächst $I=\{0,1,6,11\}$ und $J=\{0,1,2,3\}$. Dann ist 
\begin{equation*}
    M_{IJ}(2) = \begin{pmatrix}
        2^{0} & 2^{0} & 2^{0} & 2^{0} \\
        2^{0} & 2^{1} & 2^{2} & 2^{3} \\
        2^{0} & 2^{6} & 2^{12} & 2^{18} \\
        2^{0} & 2^{11} & 2^{22} & 2^{33} 
    \end{pmatrix} \underbrace{=}_{2^{10}=1} \begin{pmatrix}
        2^{0} & 2^{0} & 2^{0} & 2^{0} \\
        2^{0} & 2^{1} & 2^{2} & 2^{3} \\
        2^{0} & 2^{6} & 2^{2} & 2^{8} \\
        2^{0} & 2^{1} & 2^{2} & 2^{3} 
    \end{pmatrix}
\end{equation*}

Da die Differenz $i_4 - i_2 = 11 - 1 = 10$ der Ordnung von $a$ entspricht, unterscheiden sich die Exponenten in diesen Zeilen um ein Vielfaches dieser Ordnung. Dadurch sind die zweite und vierte Zeile äquivalent und $(\det M_{IJ})(a) = 0$. 
Seien nun $I=\{0,2,6,7\}$ und $J=\{0,2,4,6\}$. In diesen Indexmengen existieren keine zwei Elemente mit dem passenden Abstand von $10$. Die zweite und vierte Zeile der Submatrix

\begin{equation*}
    M_{IJ}(2) = \begin{pmatrix}
        2^{0} & 2^{0} & 2^{0} & 2^{0} \\
        2^{0} & 2^{4} & 2^{8} & 2^{12} \\
        2^{0} & 2^{12} & 2^{24} & 2^{36} \\
        2^{0} & 2^{14} & 2^{28} & 2^{42} 
    \end{pmatrix} \underbrace{=}_{2^{10}=1} \begin{pmatrix}
        2^{0} & 2^{0} & 2^{0} & 2^{0} \\
        2^{0} & 2^{4} & 2^{8} & 2^{2} \\
        2^{0} & 2^{2} & 2^{4} & 2^{6} \\
        2^{0} & 2^{4} & 2^{8} & 2^{2} 
    \end{pmatrix}
\end{equation*}

sind dennoch wieder äquivalent. Dies ergibt sich aus der Faktorisierung von $10 = 5 \cdot 2$. Die Differenz $i_4 - i_2 = 7 - 2 = 5$ entspricht dem ersten Faktor. Da alle Differenzen in $I$ ein Vielfaches der $2$ sein, unterscheiden sich die Exponenten in Zeile 2 und 4 wieder um ein Vielfaches der Ordnung, wodurch die lineare Abhängigkeit entsteht. Das folgende Lemma formalisiert dieses Resultat.



\begin{lemma} \label{lemma:equal-columns}
    Seien $a \in \field{p}[n], o = \ord{a} = v\cdot w$ und $I,J \subseteq \mathbb{Z}$.
    Falls
    \begin{equation} \label{equation:all-equal}
        \forall \; i, i' \in I, i \neq i':\quad i \equiv i' \pmod v
    \end{equation}
    und
    \begin{equation} \label{equation:two-equal}
        \exists \; j, j' \in J, j \neq j':\quad  j \equiv j' \pmod w
    \end{equation}
    gelten, folgt
    \begin{equation*}
        (\det M_{IJ})(a) = 0
    \end{equation*}
\end{lemma}

{
\Crefname{equation}{Bedingung}{Bedingungen}

\begin{proof}
    Wir betrachten $j,j' \in J$, welche die obige \Cref{equation:two-equal} erfüllen und zeigen, dass die dazugehörigen Spalten in $M_{IJ}$ linear abhängig sind. Seien $j = (c + bw)$, $j' = (c + b'w)$ und $I = \{(d+l_1v),\dots,(d+l_kv)\}$. Die Spalten zu $j$ und $j'$ haben die Form
    \begin{align*}
        \begin{pmatrix}
            a^{i_1j} & a^{i_1j'} \\
            \vdots & \vdots \\
            a^{i_kj} & a^{i_kj'}
        \end{pmatrix} &=
        \begin{pmatrix}
            a^{(d+l_1v)(c + bw)} & a^{(d+l_1v)(c + b'w)} \\
            \vdots & \vdots \\
            a^{(d+l_kv)(c + bw)} & a^{(d+l_kv)(c + b'w)}
        \end{pmatrix} \\
        &= \begin{pmatrix}
            a^{dc + dbw +l_1vc + l_1bvw} & a^{dc + db'w +l_1vc + l_1b'vw} \\
            \vdots & \vdots \\
            a^{dc + dbw +l_kvc + l_kbvw} & a^{dc + db'w +l_kvc + l_kb'vw}
        \end{pmatrix} \\
        &= \begin{pmatrix}
            a^{dc + dbw +l_1vc + l_1bvw} & a^{dc + db'w +l_1vc + l_1b'vw} \\
            \vdots & \vdots \\
            a^{dc + dbw +l_kvc + l_kbvw} & a^{dc + db'w +l_kvc + l_kb'vw}
        \end{pmatrix} \\
    \end{align*}
    Die konstanten Terme $a^{dc + dbw}$ und $a^{dc + db'w}$ können aus den Spalten faktorisiert werden, sodass verbleiben
    \begin{equation*}
        \begin{pmatrix}
            a^{l_1vc} {a^{vw}}^{l_1b} & a^{l_1vc} {a^{vw}}^{l_1b'} \\
            \vdots & \vdots \\
            a^{l_kvc} {a^{vw}}^{l_kb} & a^{l_kvc} {a^{vw}}^{l_kb'}
        \end{pmatrix}
    \end{equation*}
    Da $a^{vw} = a^{o} = 1$ gilt, verbleiben die gleichen Spalten.
    \begin{equation*}
        \begin{pmatrix}
            a^{l_1vc} & a^{l_1vc} \\
            \vdots & \vdots \\
            a^{l_kvc} & a^{l_kvc}
        \end{pmatrix}
    \end{equation*}

    Aus der linearen Abhängigkeit der beiden Spalten folgt $(\det M_{IJ})(a) = 0$.
\end{proof}

Die \Cref{equation:all-equal}, dass alle Elemente einer Indexmenge in der selben Restklasse liegen müssen, ist eine starke Einschränkung in der Wahl der Indizes. Diese Anforderung kann abgeschwächt werden, wenn zusätzliche Bedingungen an die jeweils andere Indexmenge gestellt werden. 



\begin{satz} \label{satz:equal-columns-subs}
    Seien $a \in \field{p}[n], o = \ord{a} = v\cdot w$ und $I,J \subseteq \mathbb{Z}$.
    Falls ein $I' \subseteq I$ mit $|I'| = \gamma \geq 1$ existiert, dass \Cref{equation:all-equal} erfüllt und alle $J' \subseteq J$ mit $|J'| = \gamma$ \Cref{equation:two-equal} erfüllen, folgt
    \begin{equation*}
        (\det M_{IJ})(a) = 0
    \end{equation*}
\end{satz}

\begin{proof}
    Zu Bestimmung der Determinante führen wir eine Zeilenentwicklung über die Indizes $I'$ durch. 
\begin{equation*}
    (\det M_{IJ})(a) = \left( \sum_{J':|J'| = \gamma} (-1)^{\sum I' + \sum J'} \left( \det M_{I'J'} \cdot \det M_{I\backslash I',J\backslash J'} \right) \right)(a)
\end{equation*}
Da $I'$ und alle $J'$ die \Cref{equation:all-equal,equation:two-equal} erfüllen, kann \Cref{lemma:equal-columns} auf die Terme $\det M_{I'J'}$ angewendet werden. Somit ist jeder Summand gleich $0$ und die Aussage folgt.
\end{proof}


Wir betrachten zum Beispiel die Indexmengen $I = \{0,2,3,4\}$ und $J = \{0,2,5,7\}$. Da die Teilmenge $I\backslash\{3\}$ die \Cref{equation:all-equal} erfüllt, wird entlang der Zeile zum Exponenten $3$ entwickelt. Die für die Berechnung relevanten Submatrizen ergeben sich aus den Paarungen

\begin{align*}
    I_1=\{0,2,4\} \text{ und } J_1=\{2,5,7\}, \\
    I_2=\{0,2,4\} \text{ und } J_2=\{0,5,7\}, \\
    I_3=\{0,2,4\} \text{ und } J_3=\{0,2,7\}, \\
    I_4=\{0,2,4\} \text{ und } J_4=\{0,2,5\}.
\end{align*}

Diese erfüllen jeweils \Cref{equation:all-equal,equation:two-equal}, sodass \Cref{satz:equal-columns-subs} angewendet werden kann und $(\det M_{IJ})(a) = 0$ folgt.
}

Anstatt alle Teilmengen von $J$ einer bestimmten Kardinalität auf die Bedingung zu prüfen, kann auch das minimale $\gamma$ gesucht werden

\section{Spezielle Indexmengen}

\newcolumntype{C}{>{$}c<{$}}

In diesem Kapitel wollen wir spezielle Indexmengen betrachten und Bedingungen aufstellen, wann die Submatrix $M_{IJ}$ für ein $a \in \field{p}[n]$ invertierbar ist. Wir betrachten zunächst den allgemeinen Fall für $|I| = |J| = 3$. Seien $I = \{0, \alpha, \beta\}$ und $J = \{0, \gamma, \delta\}$. Dann besitzt $M_{IJ}$ die Form 
\begin{equation*}
    \begin{pmatrix}
        1 & 1 & 1 \\
        1 & x^{\alpha\gamma} & x^{\beta\gamma} \\
        1 & x^{\alpha\delta} & x^{\beta\delta}
    \end{pmatrix}.
\end{equation*}

Mithilfe von Zeilenumformungen und anschließender Spaltenentwicklung ergibt sich die Determinante wie folgt:

\begin{align*}
    \det \begin{pmatrix}
        1 & 1 & 1 \\
        1 & x^{\alpha\gamma} & x^{\beta\gamma} \\
        1 & x^{\alpha\delta} & x^{\beta\delta}
    \end{pmatrix} 
    &= \det \begin{pmatrix}
        1 & 1 & 1 \\
        0 & x^{\alpha\gamma} -1 & x^{\beta\gamma} -1 \\
        0 & x^{\alpha\delta} -1 & x^{\beta\delta} -1
    \end{pmatrix} \\
    &= \det \begin{pmatrix}
        x^{\alpha\gamma} -1 & x^{\beta\gamma} -1 \\
        x^{\alpha\delta} -1 & x^{\beta\delta} -1
    \end{pmatrix} \\
    &= (x^{\alpha\gamma} -1)(x^{\beta\delta} -1) - (x^{\alpha\delta} -1)(x^{\beta\gamma} -1)
\end{align*}

Für ein $a \in \field{p}[n]$ ist die Submatrix $M_{IJ}$ somit genau dann nicht invertierbar, falls $a$ die Gleichung $(x^{\alpha\gamma} -1)(x^{\beta\delta} -1) = (x^{\alpha\delta} -1)(x^{\beta\gamma} -1)$ löst. Da diese Gleichung noch sehr allgemein ist, setzen wir noch konkreter $I = \{0,1,3\}$. Dadurch reduziert sich die Bedingung zu

\begin{align}
        & (x^{\gamma} -1)(x^{3\delta} -1) = (x^{\delta} -1)(x^{3\gamma} -1) \nonumber \\
    \iff & \frac{(x^{3\delta} -1)}{(x^{\delta} -1)} = \frac{(x^{3\gamma} -1)}{(x^{\gamma} -1)} \nonumber \\
    \iff & {x^\delta}^2 + x^{\delta} + 1 = {x^{\gamma}}^2 + x^{\gamma} + 1 \nonumber \\
    \iff & {x^\delta}^2 + x^{\delta} = {x^{\gamma}}^2 + x^{\gamma}. \label{equation:013}
\end{align}

Anders formuliert sind nun zwei Elemente $v_1 = a^\delta$ und $v_2 = a^\gamma$ gesucht, die $x^2 + x = m$ für ein festes $m \in \field{p}[n]$ lösen. Abhängig von der Charakteristik des Körpers existieren zwei unterschiedliche Lösungsformeln.

\begin{equation*}
    v_{1,2} = \begin{cases}
        -2^{-1} \pm \sqrt{2^{-2} + m}                                   & \text{falls } \mathrm{char}(\field{p}[n]) \neq 2 \\  
        \sum_{k=1}^{n-1} m^{2^j}(\sum_{l=0}^{k-1} u^{2^k}),\quad v_1 + 1 & \text{sonst}
    \end{cases} 
\end{equation*}

Sofern $v_1,v_2 \neq 0$ sind, existieren für jeden Erzeuger $a \in \field{p}[n]$ Exponenten $\gamma$ und $\delta$ mit $v_1 = a^\gamma$ und $v_2 = a^\delta$. Für diese Erzeuger ist dann die \Cref{equation:013} erfüllt und $M_{IJ}$ nicht invertierbar. 

Im Folgenden betrachten wir ein Beispiel für den Körper $\field{11}$, wobei $-2^{-1} = -6 = 5$ und $2^{-2} = 3$ gilt. In \Cref{table:sol_013F11} sind für jedes 


{\renewcommand{\arraystretch}{1.5}
\begin{table}
    \centering
    \begin{tabular}{|C|C|C|C|C|C|C|C|C|C|C|C|C}
    \hline
    m          & 0    & 1   & 2   & 3 & 4 & 5 & 6   & 7  & 8   & 9   & 10 \\
    \hline
    \sqrt{3+m} & 5    & 2   & 4   & - & - & - & 3   & -  & 0   & 1   & -  \\
    \hline
    v_{1,2} = 5 \pm \sqrt{3+m}   & 10,0 & 3,7 & 9,1 & - & - & - & 8,2 & -  & 5,5 & 6,4 & -  \\
    \hline
    \end{tabular}
    \caption{Lösungen $v_{1,2}$ zu $x^2 + x = m$ für festes $m \in \field{11}$.} \label{table:sol_013F11}
\end{table}
}

\begin{table}
    \centering
    \begin{tabular}{|C|C|C|C|C|C|C|C|C|C|C|C|}
    \hline
    \text{\backslashbox{a}{i}} & 0 & 1  & 2 & 3 & 4 & 5  & 6 & 7 & 8 & 9 & 10 \\ \hline
    1  & 1 & 1  &   &   &   &    &   &   &   &   &   \\ \hline
    2  & 1 & 2  & 4 & 8 & 5 & 10 & 9 & 7 & 3 & 6 & 1 \\ \hline
    3  & 1 & 3  & 9 & 5 & 4 & 1  &   &   &   &   &   \\ \hline
    4  & 1 & 4  & 5 & 9 & 3 & 1  &   &   &   &   &   \\ \hline
    5  & 1 & 5  & 3 & 4 & 9 & 1  &   &   &   &   &   \\ \hline
    6  & 1 & 6  & 3 & 7 & 9 & 10 & 5 & 8 & 4 & 2 & 1 \\ \hline
    7  & 1 & 7  & 5 & 2 & 3 & 10 & 4 & 6 & 9 & 8 & 1 \\ \hline
    8  & 1 & 8  & 9 & 6 & 4 & 10 & 3 & 2 & 5 & 7 & 1 \\ \hline
    9  & 1 & 9  & 4 & 3 & 5 & 1  &   &   &   &   &   \\ \hline
    10 & 1 & 10 & 1 &   &   &    &   &   &   &   &   \\ \hline
    \end{tabular}
    \caption{Alle von einem $a \in \field{11}$ erzeugten Untergruppen.} \label{table:subgroupsF11}
\end{table}

{\renewcommand{\arraystretch}{1.5}
\begin{table}
    \centering
    \begin{tabular}{|C|C|C|C|C|C|C|C|C|}
    \hline
    m               & 0 & 1 & o & o + 1 & o^2 & o^2 + 1 & o^2 + o  & o^2 + o + 1 \\
    \hline
    \mathrm{Tr}(m)  & 0 & 1 & 0 & 1     & 0   & 1       & 0        & 1 \\
    \hline
    v_1         & 0 & - & o^2 & - & o^2 + o & - & o & - \\
    \hline
    \end{tabular}
    \caption{Lösungen $v_1$ zu $x^2 + x = m$ für festes $m \in \field{2}[3]$.} \label{table:sol_013F2_3}
\end{table}
}

\begin{table}
    \centering
    \begin{tabular}{|C|C|C|C|C|C|C|C|C|C|C|C|}
    \hline
    \text{\backslashbox{i}{a}}& 1 & o       & o+1     & o^2     & o^2 + 1 & o^2 + o & o^2 + o + 1 \\ \hline
    0 & 1 & 1       & 1       & 1       & 1       & 1       & 1           \\ \hline
    1 & 1 & o       & o+1     & o^2     & o^2+1   & o^2+o   & o^2+o+1     \\ \hline
    2 &   & o^2     & o^2+1   & o^2+o   & o^2+o+1 & o       & o+1         \\ \hline
    3 &   & o+1     & o^2     & o^2+1   & o^2+o   & o^2+o+1 & o           \\ \hline
    4 &   & o^2+o   & o^2+o+1 & o       & o+1     & o^2     & o^2+1       \\ \hline
    5 &   & o^2+o+1 & o       & o+1     & o^2     & o^2+1   & o^2+o       \\ \hline
    6 &   & o^2+1   & o^2+o   & o^2+o+1 & o       & o+1     & o^2         \\ \hline
    7 &   & 1       & 1       & 1       & 1       & 1       & 1           \\ \hline
    \end{tabular}
    \caption{Alle von einem $a \in \field{2}[3]$ erzeugten Untergruppen.} \label{table:subgroupsF2_3}
\end{table}


\end{document}